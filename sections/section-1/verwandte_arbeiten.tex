\subsection{Verwandte Arbeiten}\label{subsec:Verwandte-Arbeiten}
Den Ausgangspunkt dieser Arbeit stellt die Publikation \cite{tecuci1992} von G. Tecuci dar, die die Automatisierung der Wissenserfassung als ein Konzept der Erweiterung, Aktualisierung und Verbesserung der Wissensbasis beschreibt \cite[S.1444]{tecuci1992}. In diesem Zusammenhang wird ein lernendes System vorgestellt, das eine Auswahl an generischen Ans�tzen des maschinellen Lernens bei der Wissenserfassung umsetzt \cite[S.1445]{tecuci1992}. Ferner wird ein Framework entwickelt, das die Wissenserfassung durch maschinelles Lernen automatisiert. Dar�ber hinaus werden die erlernten Daten von einem Experten auf Korrektheit �berpr�ft. Die Entwicklung der Wissensbasis wird in drei Phasen durchgef�hrt. In der ersten Phase wird die Anfangswissensbasis aufgebaut, die unvollst�ndig und teilweise widerspr�chliche Daten enthalten. Die zweite Phase umfasst die inkrementelle Erweiterung und Verbesserung der Wissensbasis. Schlie�lich wird in der dritten Phase die Wissensbasis in Bezug auf Effizienz optimiert \cite[S.1444]{tecuci1992}. Die Kernaussage der Arbeit besteht darin, dass die Kooperation zwischen dem Experten und dem Expertensystem in jeder Phase die Automatisierung der Datenerfassung deutlich erleichert. Beispielsweise k�nnen die Daten von einem Algorithmus generiert werden. Daraufhin werden sie vom fachlichen Experten auf formale und semantische Korrekteit �berpr�ft und in der Wissensbasis gespeichert \cite[S.1445]{tecuci1992}. Die Weiterentwicklung dieses Frameworks wird in \cite{tecuci1994} fortgef�hrt.\\
Neben \cite{tecuci1992} und \cite{tecuci1994} gibt es eine Reihe weiterer Ans�tze, die das maschinelle Lernen bei der Automatisierung der Datenerfassung zur Hilfe nehmen. Einige Beispiele sind \cite{castro1999}, \cite{castro2001}, \cite{webb1996}. Dabei ist die Idee der Zusammenarbeit zwischen dem Experten und dem Lernalgorithmus durchwegs verbreitet. Ein Beispiel stellt die Arbeit von Castro et al. \cite{castro2001} dar. Ihr Ansatz beruht auf der Arbeit von \cite{tecuci1992}. Als Startpunkt wird eine unvollst�ndige Anfangswissensbasis betrachtet, die schrittweise verbessert wird, indem der Experte die Fragen des Systems beantwortet. Bei der Frageerstellung wird ein Lernalgorithmus eingesetzt, der aus einer Trainingsmenge die Regeln zur Fragenbildung lernt und kontinuierlich die Qualit�t der Fragen verbessert. Dabei betonen Castro et al. \cite{castro2001}, dass der Lernalgorithmus keineswegs den Wissensingenieur ersetzen kann. Vielmehr soll er diesem die Routinearbeit abnehmen und bei den schwierigeren Aufgaben unterst�tzen, indem verschiedene Varianten des Interviews vom Algorithmus vorgeschlagen werden \cite[S.308]{castro2001}.\\
Ein praxisorientierter Ansatz f�r die Automatisierung der Wissenserfassung wird in \cite{gebus2009} thematisiert. Allgemein handelt es sich um die Transformation eines datenbasierten Sys\-tems in ein wissensbasiertes System, um die Effizienz des Produktionsprozesses zu steigern. Im Hinblick auf die Automatisierung der Aktualisierung und Erweiterung der unvollst�ndigen Wissensbasis beziehen sich Gebus und Leivisk{\"a} auf die Erkenntnisse aus \cite{tecuci1992}, \cite{winter1992} und \cite{su2002}. Bez�glich der Erfassung von Erfahrungswissen nehmen die Autoren Bezug auf die Arbeit von Okamura et al. \cite{okamura1991}, die Heuristik bei Probleml�sungen einsetzen. Im praktischen Teil wird ein bereits bestehendes datenbasiertes \acf{DSS} betrachtet, das die Unternehmensf�hrung bei den Entscheidungen in Bezug auf die Produktionsoptimierung unterst�tzen soll. Allerdings werden St�rungen in der Produktion von Anlagenbedienern (Experten) mithilfe von Erfahrungswissen intern behoben. Als Folge hat die Unternehmensf�hrung kein umfassendes Bild der Produktion, was sich langfristig negativ auf die Produktion auswirkten kann. Aus diesem Grund erweitern Gebus und Leivisk{\"a} das System um eine Wissenstr�gerschnittstelle, um das Expertenwissen in die Datenbank zu integrieren \cite[S.94]{gebus2009}. Gebus und Leivisk{\"a} veranschaulichen damit, wie die Idee der Zusammenarbeit zwischen dem Experten und dem wissensbasierten System zur Automatisierung der Wissenserfassung im Kontext eines Unternehmens umgesetzt werden kann.\\
Mit der rasanten Entwicklung des World Wide Web hat sich eine Forschungsrichtung entwickelt, die sich mit der Daten- und Wissenserfassung aus Online-Ressourcen befasst. Ein fundierter �berblick �ber die Webdatenerfassung, sowie aktuelle Ans�tze und An\-wen\-dung\-en in Bereichen wie Business Intelligence, Web-Crawling etc. wird in \cite{ferrara2014} vorgestellt. Dabei werden sowohl theoretische als auch praktische Aspekte umfassend thematisiert.  In theoretischer Hinsicht werden zuerst allgemeine Probleme wie Automatisierungsgrad, Skalierbarkeit, Datenschutz, Instabilit�t der Ressourcenstruktur und die Trainingsmenge angesprochen \cite[S.301-302]{ferrara2014}. In Bezug auf die praktischen Ans�tze wurden das Baumparadigma, Web-Wrapper und hybride Systeme durch die Analyse zahlreicher Publikationen systematisiert. Bei Anwendungen zur Webdatenerfassung w�chst der Trend in Richtung freier Open-Source Projekte, die mit kommerziellen Anwendungen im Wettbewerb stehen \cite[S.310]{ferrara2014}. Die Autoren nennen als Beispiel die Pipes\footnote{https://en.wikipedia.org/wiki/Yahoo!\_Pipes} von Yahoo \cite[S.315]{ferrara2014}. Allerdings ist das Beispiel schon veraltet, da die Plattform von Yahoo nicht mehr unterst�tzt wird. Diese Tatsache best�tigt jedoch die Aussage �ber die hohe Dynamik im Webbereich. Nichtsdestotrotz gibt es einige Technologien, die bereits �ber mehrere Jahre hinweg bestehen. Ein Beispiel stellen Feed-Services wie RSS und Atom dar, die Pull-Benachrichtigungen �ber �nderungen auf Webseiten, Blogs usw. erm�glichen (mehr dazu in \cite{hammersley2005} und \cite{tilkov2015}). Bezogen auf An\-wen\-dung\-en, die aus dem kommerziellen Bereich stammen, f�hren die Autoren in \cite{ferrara2014} Lixto Web-Wrapper als Beispiel an. Urspr�nglich entstand Lixto aus einem Forschungsprojekt und wurde sp�ter als kommerzielle Anwendung implementiert \cite{ferrara2011}. Die Idee von Lixto in \cite{ferrara2011} spricht das Problem der Stabilit�t des Web-Wrappers an. Die Anwendung erkennt automatisch �nderungen einer Webseite und passt sich an die neue Struktur an \cite[S.309]{ferrara2014}.