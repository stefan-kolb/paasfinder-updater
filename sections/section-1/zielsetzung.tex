\subsection{Zielsetzung}\label{subsec:Zielsetzung}
Das Ziel der vorliegenden Arbeit ist die Erarbeitung eines allgemeinen Konzeptes zur Automatisierung der Datenerfassung. Da die komplett automatisierte Datenerfassung sehr schwierig umzusetzen ist, liegt der Schwerpunkt dieser Arbeit auf der Kombination zwischen den manuellen und maschinellen Vorgehensweisen.\\
Beim technischen Kontext wird angedeutet, dass die Umsetzung im Rahmen eines bestimmten Anwendungsbereichs erfolgt. Es wird also kein Allgemeinwissen wie in \cite{tandon2016}, sondern ein anwendungsbezogenes Wissen betrachtet. In diesem Zusammenhang wird das Konzept auf Basis eines Expertensystems entwickelt, da die Wissensbasis eines Expertensystems spezifischer ist als bei einem wissensbasierten System.\\
Die praktische Umsetzung soll auf Basis von \textit{PaaSfinder}\footnote{https://paasfinder.org} erfolgen. Bei \textit{PaaSfinder} handelt es sich um eine Web-Anwendung, die eine Wissensdatenbank im Bereich \ac{PaaS} verwaltet. 
Zur Einordnung geh�rt \ac{PaaS} zum Bereich von Cloud Computing \cite{nist2011}. Das Ziel von \ac{PaaS} besteht in der Erleichterung der Anwendungsentwicklung, indem Entwicklungsumgebung von einem \ac{PaaS}-Anbieter als ein konfigurierbarer Service angeboten wird \cite[S.14]{lawton2008}. Aufgrund der hohen Anzahl von \ac{PaaS}-Anbieter, Vielzahl von Einstellungsm�glichkeiten und potentiellen Inkompatibilit�t zwischen den unterschiedlichen Anbietern gibt es einen Bedarf an einen systematische Analyse des Marktes, die aufgrund von Daten �ber \ac{PaaS}-Anbieter  erfolgt. Die Daten von \textit{PaaSfinder} wurden bisher haupts�chlich manuell erfasst, was m�hsam, zeit- und kostenintensiv ist. Das Ziel dieser Arbeit besteht darin, die f�r den Anwendungsfall von \textit{PaaSfinder} Technologien zur Automatisierung der Datenerfassung zu erforschen und umzusetzen. 