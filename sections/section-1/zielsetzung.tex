\subsection{Zielsetzung}\label{subsec:Zielsetzung}
Das Ziel der vorliegenden Arbeit ist die Erarbeitung eines allgemeinen Konzeptes zur Automatisierung der Datenerfassung. Da die ausschlie�lich automatisierte Datenerfassung sehr schwierig umzusetzen ist, liegt der Schwerpunkt dieser Arbeit auf der Kombination von manuellen und maschinellen Vorgehensweisen.\\
Die praktische Umsetzung soll am Beispiel von \textit{PaaSfinder}\footnote{https://paasfinder.org} erfolgen. Bei \textit{PaaSfinder} handelt es sich um eine Web-Anwendung, die eine Wissensdatenbank im Bereich \ac{PaaS} verwaltet. Das Ziel von \ac{PaaS} besteht in der Erleichterung der Anwendungsentwicklung, indem eine Entwicklungsumgebung von einem \ac{PaaS}-Anbieter als ein konfigurierbarer Service angeboten wird \cite[S.14]{lawton2008}. Aufgrund der hohen Anzahl von \ac{PaaS}-Anbietern, der Vielzahl von Einstellungsm�glichkeiten und potentiellen Inkompatibilit�ten zwischen den unterschiedlichen Anbietern, gibt es einen realen Bedarf an einem systematischen Marktvergleich, der mithilfe von Daten der \ac{PaaS}-Anbieter erfolgt. Die im \textit{PaaSfinder} enthaltenen Daten wurden bisher haupts�chlich manuell erfasst, was m�hsam, zeit- und kostenintensiv ist. Das Ziel der Arbeit besteht darin, die Methoden zur Automatisierung der Datenerfassung hinsichtlich des Anwendungsbeispiels \textit{PaaSfinder} zu erforschen und umzusetzen. Hierzu werden folgende Aspekte bearbeitet: Als Erstes soll eine Benutzerschnittstelle zur Korrektur der bestehenden Daten entwickelt werden. Momentan erfordert eine Aktualisierung der Daten fachspezifische Vorkenntnisse, was viele Nutzer daran hindert, ein Update zu erstellen. Als N�chstes wird die Erstellung eines Pull Requests mit den Daten der Benutzerschnittstelle automatisiert. Ferner werden weitere M�glichkeiten erforscht, die Daten automatisch zu erfassen. Es k�nnen beispielsweise soziale Netzwerke eingesetzt werden. Schlie�lich werden die automatischen Tests erweitert, um die Konsistenz der Daten sicherzustellen. 