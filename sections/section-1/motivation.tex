\section{Einf�hrung}\label{sec:Einf�hrung}
\subsection{Motivation}\label{subsec:Motivation}
Die Idee von wissensbasierten Systemen entstand aus dem Anliegen, ein intelligentes System zu schaffen, das mithilfe des spezifischen Wissens die Menschen bei den Probleml�sungen unterst�tzt \cite[S.18]{akerkar2010}. Darin besteht der Unterschied zwischen einem konventionellen Informationssystem und einem wissensbasierten System. Informationssysteme sind datenbasiert aufgebaut und konzentrieren sich auf Datenverarbeitung \cite[S.19]{akerkar2010}. Eine Spezialisierung der wissensbasierten Systeme stellen Expertensysteme dar, in denen das Wissen letztendlich von Experten stammt. Ein wissensbasiertes System bzw. ein Expertensystem ist durch die Trennung zwischen der Wissensdarstellung eines Problembereichs und der Wissensverarbeitung gekennzeichnet \cite[S.11]{beierle2014}.\\ 
Die Beschreibung des Wissens von einem wissensbasierten bzw. Expertensystem erfolgt in der Wissensbasis, die in Form einer Wissensdatenbank realisiert wird. Die Anfangswissensbasis wird in der Regel mithilfe manueller Datenerfassung aufgebaut \cite[S.70]{kurbel1992}. Ein Interview zwischen einem Wissensingenieurs und einem Wissenstr�ger wird beispielsweise oft eingesetzt \cite[S.76]{gottlob1990}. Allerdings sind die manuellen Wissenserwerbsmethoden f�r die Weiterentwicklung der Wissensbasis hinsichtlich der Erweiterung und Aktualisierung der Wissensdatenbank eher schlecht geeignet, da sie fehleranf�llig, kosten- und zeitintensiv sind. Aus diesem Grund gibt es Bestrebungen, den Prozess der Datenerfassung zu automatisieren. Eine Auswahl an bestehenden Ans�tzen, die bez�glich der vorliegenden Arbeit relevant sind, wird im Abschnitt zu den verwandten Arbeiten gegeben. Trotz der vielen Herangehensweisen gibt es kein einheitliches Konzept, das einen allgemeinen Rahmen f�r die Automatisierung der Datenerfassung bildet.