\section{Einf�hrung}\label{sec:Einf�hrung}
\subsection{Motivation}\label{subsec:Motivation}
Die Idee von wissensbasierten Systemen entstand aus dem Bed�rfnis, ein intelligentes System zu schaffen, das mittels spezifischen Wissens die Fachexperten bei den Problem\-l�\-sung\-en unterst�tzt \cite[S.18]{akerkar2010}. Eine Spezialisierung wissensbasierter Systeme stellen Expertensysteme dar, in denen das Wissen letztendlich von Experten entsprechender Dom�nen stammt. Ein wissensbasiertes System, bzw. ein Expertensystem, ist durch die Trennung der Wissensdarstellung eines Problembereichs und der Wissensverarbeitung gekennzeichnet \cite[S.11]{beierle2014}.\\ 
Die Beschreibung des Wissens eines wissensbasierten bzw. eines Expertensystems erfolgt in der Wissensbasis, die in Form einer Wissensdatenbank realisiert wird. Die Anfangswissensbasis wird in der Regel mithilfe manueller Datenerfassungen aufgebaut \cite[S.70]{kurbel1992}. Hierzu wird beispielsweise h�ufig die Methode des Interviews zwischen einem Wissensingenieur und einem Wissenstr�ger eingesetzt \cite[S.76]{gottlob1990}. Allerdings sind die manuellen Wissenserwerbsmethoden f�r die Weiterentwicklung der Wissensbasis hinsichtlich der Erweiterung und Aktualisierung der Wissensdatenbank eher schlecht geeignet, da sie fehleranf�llig sowie kosten- und zeitintensiv sind. Aus diesen Gr�nden gibt es Bestrebungen, den Prozess der Datenerfassung zu automatisieren. Eine Auswahl an bestehenden Ans�tzen, die f�r die vorliegende Arbeit relevant sind, wird im Abschnitt \ref{subsec:Verwandte-Arbeiten} gegeben. Trotz der vielen Herangehensweisen gibt es kein einheitliches Konzept, das einen allgemeinen Rahmen f�r die Automatisierung der Datenerfassung bildet.