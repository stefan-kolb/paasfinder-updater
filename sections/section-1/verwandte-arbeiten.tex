\subsection{Verwandte Arbeiten}\label{subsec:Verwandte-Arbeiten}
Der Ausgangspunkt der vorliegenden Forschungsfrage stammt aus der Arbeit \cite{tecuci1992} von Gheorghe Tecuci. Dabei handelt es sich um die Automatisierung der Wissenserfassung als ein Konzept der Erweiterung, Aktualisierung und Verbesserung der Wissensbasis \cite[S.1444]{tecuci1992}. Die Entwicklung der Wissensbasis umfasst dabei drei Phasen. In der ersten Phase wird die Anfangswissensbasis aufgebaut, die unvollst�ndig und teilweise widerspr�chlich sein kann. Die zweite Phase befasst sich mit der inkrementellen Erweiterung und Verbesserung der Wissensbasis, wobei der Bezug auf das maschinelle Lernen genommen wird. Schlie�lich wird in Z�ge der dritten Phase die Wissensbasis in Bezug auf Effizienz optimiert \cite[S.1444]{tecuci1992}. Die Kernaussage der Arbeit besteht darin, dass jede Phase von der Kooperation zwischen dem fachlichen Experten und dem System profitieren kann. So k�nnen die Inhalte maschinell vorbeitetet werdne und anschlie�end vom fachlichen Experten auf die formale und semantische Korrekteit gepr�ft werden \cite[S.1445]{tecuci1992}. In \cite{tecuci1994}. wird dieser Ansatz fortgef�hrt, wobei der Schwerpunkt auf das maschinelle Lernen gelegt wird. Es handelt sich dabei um die Entwicklung eines Frameworks der Verbesserung und Erweiterung der Wissensbasis unter Verwendung von multi-strategischem Lernen (multistrategy leraning) \cite[S.137]{tecuci1994}.\\
Ein praxisorientierter Ansatz f�r die Autormatisierung der Wissenserfassung wird in \cite{gebus2009} thematisiert. Dabei besteht das Forschungsziel in der Transformation eines datenbasierten Systems in ein wissensbasiertes System, um die Effizient der Produktion zu steigern. Im Hinblick auf die Automatisierung der Aktualisierung und Erweiterung der unvollst�ndigen Wissensbasis beziehen sich Gebus und Leivisk{\"a} auf die Erkenntnisse aus \cite{tecuci1992}, \cite{winter1992} und \cite{su2002}. Bei dem Problem der automatisierten Erfassung von Erfahrungswissen nehmen die Autoren den Bezug auf die Arbeit von Okamura et al \cite{okamura1991}, die Heuristik bei der Probleml�sungen eins�tzt. Im praktischen Teil wird ein bereits bestehendes datenbasiertes \acf{DSS} betrachtet, das die Unternehmensf�hrung bei den Entscheidungen in Bezug auf die Produktionsoptimierung unterst�tzen soll. Allerdings werden die St�rungen in der Produktion von Anlagenbedienern (Experten) mittels Erfahrungswissen intern behoben, ohne dieses Wissen weiterzugeben. Als Folge hat die Unternehmensf�hrung kein umfangreiches Bild der Produktion und kann keine optimalen Entscheidungen treffen. Aus diesem Grund erweitern Gebus und Leivisk{\"a} das bestehende DSS um eine Benutzerschnittstelle, um das Expertenwissen in die Datenbank zu integrieren \cite[S.94]{gebus2009}. Das Konzept der Kooperation zwischen dem Menschen und dem wissensbasierten System zur Automatisierung der Datenerfassung wird von Gebus und Leivisk{\"a} praxisorientiert angewandt.\\
In Bezug auf die Webdatenerfassung wurde \cite{ferrara2014} betrachtet. Die Autoren gehen eine umfassende �bersicht �ber sowohl theoretische als auch praktische Aspekte im Bereich der Datenerfassung in Digitaldokumenten. In diesem Zusammenhang wurden zahlreiche Systeme in unterschiedlichen Anwendungsgerechten wie Business Intelligence, Web-Crawling oder sogar Bioinformatik ausgewertet. Allgemein wurde die Problematik der Webdatenerfassung in den Herausforderungen (Automatisierungsgrad, Skalierbarkeit, Datenschutz, �nderung der Ressourcenstruktur und Trainingsdaten) systematisch beschrieben \cite[S.301-302]{ferrara2014}. Im Hinblick auf die theoretischen Grundlagen wurden die aktuellsten Ans�tze der Webdatenerfassung erl�utert. Dazu z�hlen Baumparadigma, Web Wrapper und hybrides System, das sich aus der Kombination des Wrapper-Konzeptes und maschinellen Lernens zusammensetzt (z.B. automatische Wrapper-Generierung) \cite[S.303-310]{ferrara2014}. \textit{Praktische Anwendungen erw�hnen}