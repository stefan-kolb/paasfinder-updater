\subsection{Verwandte Arbeiten}\label{subsec:Verwandte-Arbeiten}
Ein Konzept der Automatisierung der Wissenserfassung bei Expertensystemen wird in \cite{tecuci1992} vorgestellt. Die Entwicklung der Wissensbasis umfasst dabei drei Phasen. In der ersten Phase wird die Anfangswissensbasis aufgebaut, die unvollst�ndig und teilweise widerspr�chlich sein kann. Die zweite Phase befasst sich mit der inkrementellen Erweiterung und Verbesserung der Wissensbasis. Dabei werden Verfahren des maschinellen Lernens eingesetzt. Im Laufe der dritten Phase wird die Wissensbasis bez�glich Effizienzsteigerung optimiert \cite[S.1444]{tecuci1992}. Gheorghe Tecuci betont, dass in jeder Phase eine Kooperation zwischen dem fachlichen Experten und dem System notwendig ist. Nach der maschinellen Datenerschlie�ung werden die Ergebnisse manuell �berpr�ft und anschlie�end in die Wissensbasis �bertragen. Auf diese Weise lassen sich die Vorteile maschineller und manueller Datenerfassung optimal kombinieren \cite[S.137]{tecuci1994}, \cite[S.1445]{tecuci1992}.\\
Ein weiterer Ansatz wird in \cite{gebus2009} thematisiert. Dabei handelt es sich um ein datenbasiertes \acf{DSS}, das die Unternehmensf�hrung bei den Entscheidungen in Bezug auf die Produktionsoptimierung unterst�tzen soll. Allerdings werden die St�rungen in der Produktion von Anlagenbedienern (Experten) mittels Erfahrungswissen intern behoben, ohne dieses Wissen weiterzugeben. Als Folge hat die Unternehmensf�hrung kein umfangreiches Bild der Produktion und kann keine optimalen Entscheidungen treffen. Aus diesem Grund erweitern Gebus und Leivisk{\"a} das bestehende DSS mit einer Schnittstelle, um das Expertenwissen in die Datenbank zu integrieren \cite[S.94]{gebus2009}. Generell l�sst sich sagen, dass es sich die Transformation eines datenbasiertes System in ein wissensbasiertes System handelt. Auch in diesem Fall wird das Konzept der Kooperation zwischen dem Menschen und dem System verwendet.\\
In Bezug auf die Webdatenerfassung wurde \cite{ferrara2014} betrachtet. Die Autoren gehen eine umfassende �bersicht �ber sowohl theoretische als auch praktische Aspekte im Bereich der Datenerfassung in Digitaldokumenten. In diesem Zusammenhang wurden zahlreiche Systeme in unterschiedlichen Anwendungsgerechten wie Business Intelligence, Web-Crawling oder sogar Bioinformatik ausgewertet. Allgemein wurde die Problematik der Webdatenerfassung in den Herausforderungen (Automatisierungsgrad, Skalierbarkeit, Datenschutz, �nderung der Ressourcenstruktur und Trainingsdaten) systematisch beschrieben \cite[S.301-302]{ferrara2014}. Im Hinblick auf die theoretischen Grundlagen wurden die aktuellsten Ans�tze der Webdatenerfassung erl�utert. Dazu z�hlen Baumparadigma, Web Wrapper und hybrides System, das sich aus der Kombination des Wrapper-Konzeptes und maschinellen Lernens zusammensetzt (z.B. automatische Wrapper-Generierung) \cite[S.303-310]{ferrara2014}. \textit{Praktische Anwendungen erw�hnen}