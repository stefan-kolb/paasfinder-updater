\subsection{Verwandte Arbeiten}\label{subsec:Verwandte-Arbeiten}
Der Ausgangspunkt dieser Arbeit stellt die Publikation \cite{tecuci1992} von G. Tecuci dar, die die Automatisierung der Wissenserfassung als ein Konzept der Erweiterung, Aktualisierung und Verbesserung der Wissensbasis beschreibt \cite[S.1444]{tecuci1992}. In diesem Zusammenhang wurde ein lernendes System vorgestellt, das eine Auswahl an generischen Ans�tzen des maschinellen Lernens bei der Wissenserfassung umsetzt \cite[S.1445]{tecuci1992}. Ferner wird ein Framework gebildet, das die Wissenserfassung durch maschinelles Lernen automatisiert. Dar�ber hinaus werden die gelernten Daten von einem Experten auf die Korrektheit �berpr�ft. Die Entwicklung der Wissensbasis wird in drei Phasen durchgef�hrt. In der ersten Phase wird die Anfangswissensbasis aufgebaut, die unvollst�ndig und teilweise widerspr�chlich sein kann. Die zweite Phase umfasst die inkrementelle Erweiterung und Verbesserung der Wissensbasis. Schlie�lich wird in der dritten Phase die Wissensbasis in Bezug auf Effizienz optimiert \cite[S.1444]{tecuci1992}. Die Kernaussage der Arbeit besteht darin, dass die Kooperation zwischen dem fachlichen Experten und dem Expertensystem in jeder Phase die Automatisierung der Datenerfassung deutlich erleichert. Beispielsweise k�nnen die Daten von einem Algorithmus generiert werden. Daraufforgend werden sie vom fachlichen Experten auf die formale und semantische Korrekteit �berpr�ft und in die Wissensbasis gespeichert \cite[S.1445]{tecuci1992}. In \cite{tecuci1994} wird die Entwicklung des Frameworks fortgef�hrt, wobei der Schwerpunkt im Bereich vom multi-strategischen Lernen (multistrategy leraning) liegt \cite[S.137]{tecuci1994}.\\
Neben \cite{tecuci1992} und \cite{tecuci1994} gibt es eine Reihe unterschiedlicher Ans�tze, die das maschinelle Lernen zur Automatisierung der Datenerfassung benutzen. Einige Beispiele sind \cite{castro1999}, \cite{castro2001}, \cite{webb1996}. Dabei ist die Idee der Zusammenarbeit zwischen dem Experten und dem Lernalgorithmus durchaus verbreitet. Ein Beispiel stellt die Arbeit von Castro et al. \cite{castro2001} dar. In der grundlegenden Idee beziehen sich die Autoren auf die Arbeit von \cite{tecuci1992}. Als Startpunkt wird eine unvollst�ndige Anfangswissensbasis betrachtet, die schrittweise verbessert wird, indem der Experte die Fragen vom System beantwortet. Bei der Frageerstellung wird ein Lernalgorithmus eingesetzt, der aus einer Trainingsmenge die Regeln lernt und kontinuierlich die Qualit�t der Fragen verbessert. Dabei betonen Castro et al. \cite{castro2001}, dass der Lernalgorithmus keineswegs den Wissensingenieur ersetzen kann. Vielmehr soll der Algorithmus den Wissensingenieur die Routinearbeit abnehmen und bei den schwierigeren Aufgaben unterst�tzen, indem verschiedene Varianten der Interviews vom Algorithmus vorgeschlagen werden \cite[S.308]{castro2001}.\\
Ein praxisorientierter Ansatz f�r die Autormatisierung der Wissenserfassung wird in \cite{gebus2009} thematisiert. Allgemein handelt es sich um die Transformation eines datenbasierten Systems in ein wissensbasiertes System, um die Effizient der Produktion zu steigern. Im Hinblick auf die Automatisierung der Aktualisierung und Erweiterung der unvollst�ndigen Wissensbasis beziehen sich Gebus und Leivisk{\"a} auf die Erkenntnisse aus \cite{tecuci1992}, \cite{winter1992} und \cite{su2002}. Bez�glich der Erfassung von Erfahrungswissen nehmen die Autoren den Bezug auf die Arbeit von Okamura et al \cite{okamura1991}, die Heuristik bei Probleml�sungen eingesetzt. Im praktischen Teil wird ein bereits bestehendes datenbasiertes \acf{DSS} betrachtet, das die Unternehmensf�hrung bei den Entscheidungen in Bezug auf die Produktionsoptimierung unterst�tzen soll. Allerdings werden die St�rungen in der Produktion von Anlagenbedienern (Experten) mithilfe von Erfahrungswissen intern behoben, ohne dieses Wissen weiterzugeben. Als Folge hat die Unternehmensf�hrung kein umfangreiches Bild der Produktion, was sich negativ auf die Produktion auswirkt. Aus diesem Grund erweitern Gebus und Leivisk{\"a} das System um eine Wissenstr�gerschnittstelle, um das Expertenwissen in die Datenbank zu integrieren \cite[S.94]{gebus2009}. Gebus und Leivisk{\"a} veranschaulichen damit, wie die Idee der Zusammenarbeit zwischen dem Experten und dem wissensbasierten System zur Automatisierung der Wissenserfassung im Kontext eines Unternehmens umgesetzt werden kann.\\
Mit der rasanten Entwicklung des Word Wide Web hat sich eine Forschungsrichtung ergeben, die sich mit der Daten- und Wissenserfassung aus Online-Ressourcen befasst. Ein fundierter �berblick �ber die Webdatenerfassung sowie aktuelle Ans�tze und Anwendungen in Anwendungsbereichen wie Business Intelligence, Web-Crawling etc. wurde in \cite{ferrara2014} vorgestellt. Dabei werden sowohl theoretische als auch praktische Aspekte umfassend thematisiert. In der theoretischen Hinsicht wurden zuerst allgemeine Probleme wie Automatisierungsgrad, Skalierbarkeit, Datenschutz, Unstabilit�t der Ressourcenstruktur und Trainingsmenge angesprochen \cite[S.301-302]{ferrara2014}. In Bezug auf die Ans�tze wurden Baumparadigma (z.B. \cite{dave2003} und \cite{wang1998}), Web Wrapper (z.B. \cite{sahuguet1999}) und hybrides System (z.B. \cite{crescenzi2001}) durch die Analyse zahlreicher Publikationen systematisiert. Bei der Anwendungen zur Webdatenerfassung w�chst der Trend Richtung freier Open-Source Projekte, die mit kommerziellen Anwendungen im Wettbewerb stehen \cite[S.310]{ferrara2014}. Die Autoren nennen als Beispiel die Pipes\footnote{https://en.wikipedia.org/wiki/Yahoo!\_Pipes} von Yahoo \cite[S.315]{ferrara2014}. Bedauerlicherweise ist das Beispiel schon veraltet, da die Plattform nicht mehr unterst�tzt wird. Diese Tatsache best�tigt nun die Aussage �ber die hohe Dynamik im Webbereich. Nichtsdestotrotz gibt es einige Technologien, die �ber mehrere Jahre hinweg bestehen. Ein Beispiel stellen Feed-Services RSS/Atom das, die die Pull-Benachrichtigen �ber die �nderungen auf der Webseite, Blogs usw. erm�glichen (mehr dazu in \cite{hammersley2005} und \cite{tilkov2015}). Bezogen auf Anwendungen, die aus dem kommerziellen Bereich stammen, nehmen die Autoren in \cite{ferrara2014} Lixito Web-Wrapper als Beispiel. Urspr�nglich enstand Lixto aus einem Forschungsprojekt in \cite{baumgartner2001} und wurde sp�ter als kommerzielle Anwendung implementiert \cite{ferrara2011}. Die Implementierung von Lixto in \cite{ferrara2011} spricht das Problem der Stabilit�t des Wrappers. Die Anwendung erkennt automatisch �nderungen einer Webseite und passt sich an die neue Struktur an \cite[S.309]{ferrara2014}.