\subsection{Verwandte Arbeiten}\label{subsec:Verwandte-Arbeiten}
Ein Konzept der Automatisierung der Wissenserfassung wird in \cite{gebus2009} vorgestellt. Die Entwicklung der Wissensbasis umfasst dabei drei Phasen. In der ersten Phase wird die initiale Wissensdatenbank aufgebaut, die unvollst�ndig und teilweise widerspr�chlich sein kann. Die zweite Phase umfasst inkrementelle Erweiterung und Verbesserung der Wissensbasis. Dabei werden Methoden des maschinellen Lernens eingesetzt. Im Laufe der dritten Phase wird die Wissensbasis in Bezug auf Effizienz optimiert \cite[S.1444]{tecuci1992}. Gheorghe Tecuci betont, dass bei jeder Phase eine Kooperation zwischen dem fachlichen Experten und dem System notwendig ist. Nach der maschinellen Datenerschlie�ung werden die Ergebnisse manuell kontrolliert und anschlie�end in der Wissensbasis gespeichert. In dieser Weise lassen sich die Vorteile maschineller und manueller Datenerfassung optimal kombinieren \cite[S.137]{tecuci1994}, \cite[S.1445]{tecuci1992}.\\
Ein weiterer Ansatz wird in \cite{gebus2009} thematisiert. Dabei handelt es sich um ein datenbasiertes \acf{DSS}, das die Unternehmensf�hrung bei den Entscheidungen in Bezug auf die Produktionsoptimierung unterst�tzen soll. Allerdings werden die St�rungen in der Produktion von Anlagenbedienern (Experten) mittels Erfahrungswissen intern behoben, ohne dieses Wissen weiterzugeben. Als Folge hat die Unternehmensf�hrung kein umfangreiches Bild der Produktion und kann keine optimalen Entscheidungen treffen. Aus diesem Grund erweitern Gebus und Leivisk{\"a} das bestehende DSS mit einer Schnittstelle, um das Expertenwissen in die Datenbank zu integrieren \cite[S.94]{gebus2009}. Allgemeiner betrachtet geht es um die Transformation eines datenbasiertes System in ein wissensbasiertes System. Auch hier wird versucht, die Daten mithilfe der engen Kooperation zwischen dem Menschen und dem System zu erfassen. 