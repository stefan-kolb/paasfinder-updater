\subsection{Die Wissenstr�gerschnittstelle}
Bei der Wissenstr�gerschnittstelle handelt es sich um eine semi-automatisierte Methode zur Datenerfassung, die einen direkten Wissenstransfer zwischen einem Wissenstr�ger und dem Expertensystem erm�glicht. Im folgenden Abschnitt wird das allgemeine Konzept der Wissenstr�gerschnittstelle erl�utert. Darauffolgend wird ein relevantes Beispiel aus der verwandten Forschung vorgestellt und analysiert. Anschlie�end werden die grundlegenden Anforderungen zusammengefasst, die bei der Entwicklung der Wissenstr�gerschnittstelle ber�cksichtigt werden sollen. \\
Die Idee der Wissenstr�gerschnittstelle besteht darin, dass ein Wissenstr�ger sein Wissen selbst und jederzeit ins System eingeben kann, ohne die Hilfe des Wissensingenieurs zu ben�tigen. Das Verfahren wird als semi-automatisiert bezeichnet, da die Eingabe manuell vom Wissenstr�ger ausgel�st und automatisiert vom System bearbeitet wird. \\
Eine verwandte Arbeit bez�glich der Anwendung der Wissenstr�gerschnittstelle wird in \cite{gebus2009} vorgestellt. In der Arbeit handelt es sich um ein wissensbasiertes \ac{DSS} f�r ein Unternehmen im Bereich der Elektrotechnik. Gebus und Leivisk{\"a} definieren ein DSS als ein Informationssystem, das die F�hrungskr�fte bei den Entscheidungen von nicht-strukturieren Problemen unterst�tzten soll \cite[S.94]{gebus2009}. Das Ziel des Systems besteht in der Datenerfassung (z.B. Ger�teausfall) aus der Produktionsableitung, um die Produktionsprozesse zu optimieren. Da es sich grunds�tzlich um ein wissensbasiertes System handelt, kann die Studie als ein relevantes Beispiel im Rahmen dieser Arbeit betrachtet werden.

Stichpunkte f�r weitere Analyse von \cite{gebus2009}
\begin{itemize}
\item \ac{DSS} war schon im Unternehmen, allerdings nur als reine Datenbank
\item Prozesse werden von einem Anlagenbediener kontrolliert ausgehend von seiner Erfahrung und Kenntnisse
\item Entscheidungen bei Problemen im Produktionsprozess werden von einzelnen Personen getroffen, die �ber Expertenwissen verf�gen. Das Ziel: Wissen zu erfassen. 
\item Wissensrepr�sentation
\end{itemize}