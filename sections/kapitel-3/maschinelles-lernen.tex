\newpage
\subsection{Maschinelles Lernen beim Wissenserwerb}
Eine Anwendung des maschinellen Lernens bei der Automatisierung des Wissenserwerbs wird in \cite{castro2001} vorgestellt. Die grundlegende Idee dabei orientiert sich an \cite{tecuci1992}, indem die Wissensakquisition als ein Prozess der der Erweiterung, Aktualisierung und Verbesserung der unvollst�ndigen Wissensbasis betrachtet wird. Der vorliegende Algorithmus verwendet die Induktion und den Algorithmus mit Fuzzylogik, um neues Wissen aus einer Trainingsmenge zu erschlie�en. Wie bereits im Kapitel \ref{subsec:Wissensakquisitionskomponente} erw�hnt wurde, werden bei der Induktion allgemeine Schlussfolgerungen auf Basis von Einzelf�llen erzielt. Bei der Fuzzylogik handelt es sich um graduelle Aussagen, die mithilfe der vagen Pr�dikaten beschrieben werden \cite[S.27]{beierle2014}. Castro et al betonen, dass die Induktion und die Fuzzylogik effektiv mit dem unvollst�ndigen Wissen arbeiten, das bei der Erfassung des Expertenwissens oft der Fall ist \cite[S.308]{castro2001}.\\
Weitere Punkte:
\begin{itemize}
\item Wissensrepr�sentation  
\item Algorithmus
\item Beispiel
\end{itemize}