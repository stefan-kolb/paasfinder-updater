\section{Grundlagen von Expertensystemen}\label{sec:Grundlagen}
\subsection{Begriffsdefinition}\label{subsec:Begriffsdefinition}
Urspr�nglich waren Expertensysteme Anwendungsprogramme, die logische Schlussfolgerungen aus einer Wissensbasis ziehen konnten. Au�erdem konnten sie �berpr�fen, ob eine Aussage aus einer vorhandenen Wissensbasis abgeleitet werden kann \cite[S.75]{greer2010}. Daher handelt es sich in der fr�heren Literatur meist um Anwendungen, die ihr Wissen in Form von logischen Ausdr�cken darstellen und in der Lage waren, neue Erkenntnisse von bestehendem Wissen abzuleiten \cite{tecuci1992}. Im Laufe der Zeit hat sich das Konzept eines Expertensystems auf andere Anwendungsbereiche ausgeweitet. Aus diesem Grund gibt es mehrere Definitionen, die im Allgemeinen �hnlich sind und im Spezifischen Merkmale des zugeh�rigen Anwendungsbereichs beinhalten.\\
Allgemein l�sst sich sagen, dass \glqq{}ein Expertensystem ein Computersystem (Hardware und Software) ist, das in einem bestimmten Bereich Wissen und Schlussfolgerungsf�higkeit \-eines menschlichen Experten nachbildet\grqq{} \cite[S.12]{beierle2014}. Aus Sicht der Wirtschaftsinformatik zielen Expertensysteme darauf ab, \glqq{}das Expertenwissen menschlicher Fachleute in der Wissensbasis eines Computers abzuspeichern und f�r eine Vielzahl von Pro\-b\-lem\-l�\-sung\-en zu nutzen\grqq{} \cite[S.59]{mertens2012}. Beierle und Kern-Isberner gehen auf die Eigenschaften ein, die Expertensystemen aufweisen sollen \cite[S.12]{beierle2014}. Im Rahmen dieser Arbeit sind folgende Eigenschaften besonders relevant:
\begin{itemize}
\item Nutzung des Wissens eines oder mehrerer Experten, um Probleme in einem bestimmten Anwendungsbereich zu l�sen,
\item Leicht lesbare Wissensdarstellung,
\item M�glichst anschauliche und intuitive Benutzerschnittstellen,
\item Leichte Wartbarkeit und Erweiterbarkeit des Wissens im Expertensystem,
\item Unterst�tzung beim Wissenstransfer vom Experten zum System.
\end{itemize}
Hier ist es au�erdem wichtig anzumerken, dass die Begriffe \glqq{}K�nstliche Intelligenz\grqq{}, \glqq{}wissensbasiertes System\grqq{} und \glqq{}Expertensystem\grqq{} in einer engen Beziehung zueinander stehen. Haun \cite[S.30]{haun2000} stellt eine systematische Abgrenzung dieser Begriffe vor, die sich folgenderma�en beschreiben l�sst:
\begin{itemize}
\item \textit{K�nstliche Intelligenz} stellt den Oberbegriff dar und bildet den theoretischen Rahmen f�r die Entwicklung von wissensbasierten Systemen und Expertensystemen.
\item \textit{Wissensbasierte Systeme} sind eine Teilmenge der Anwendungen innerhalb des Bereichs der k�nstlichen Intelligenz. Sie wenden die Wissensverarbeitung auf ein konkretes Aufgabengebiet an und verwalten Allgemeinwissen explizit und getrennt vom Rest des Systems.
\item \textit{Expertensysteme}, die ein Teilbereich der wissensbasierten Systeme sind, stellen eine Spezialisierung von wissensbasierten Systemen dar. Sie verwalten spezifisches Expertenwissen, das auf praxisbezogene Probleme angewandt wird.
\end{itemize}
Graphisch l�sst sich die vorliegende Abgrenzung wie in Abbildung \ref{Abgrenzung} darstellen:
\begin{figure}[H] 
	\centering
	\includegraphics[width=0.55\textwidth]{images/abgrenzung.png}
	\caption{Begriffsabgrenzung, \cite[S.30]{haun2000}}
	\label{Abgrenzung}
\end{figure} 
Es l�sst sich feststellen, dass der Unterschied eines Expertensystems zu einem wissensbasierten System darin besteht, dass das Wissen immer von einem Experten stammt. Allerdings ist dieses Kriterium nicht besonders aussagekr�ftig. Beierle und Kern-Isberner weisen darauf hin, dass nach diesem Kriterium viele der existierenden wissensbasierten Systeme als Expertensysteme bezeichnet werden k�nnten \cite[S.11]{beierle2014}. Als Reaktion auf fehlende Kriterien stellen die Autoren die Eigenschaften eines Experten dar, die sich folgenderma�en zusammenfassen lassen:
\begin{itemize}
\item Experten sind selten und teuer.
\item Experten sind nicht immer verf�gbar.
\item Die Leistungsf�higkeit der Experten ist nicht konstant, sondern kann im Tagesverlauf schwanken.
\item Expertenwissen kann oft nicht als solches weitergegeben werden.
\item Expertenwissen kann verloren gehen.
\end{itemize}
Ein gutes Beispiel hinsichtlich der Gefahr, dass Expertenwissen verloren gehen kann, wird in \cite[S.94]{gebus2009} vorgestellt. Gebus nimmt hier Bezug auf die Mitarbeiter, die als Experten gelten und aus Altersgr�nden das Unternehmen verlassen. Somit geht auch das Erfahrungswissen aus dem Unternehmen verloren.\\
Zusammenfassend l�sst sich sagen, dass die Entwicklung eines Expertensystems ein hohes Potenzial besitzt. Allerdings kann ein Expertensystem nicht als Ersatz f�r einen menschlichen Experten betrachtet werden. Vielmehr geht es um die Erfassung, Darstellung und Pflege des Expertenwissens in einem Expertensystem, um die Arbeitsprozesse effizienter zu gestalten und sowohl erfahrene als auch neue Anwender in einem bestimmten Wissensbereich bei der Aufgabenabwicklung zu unterst�tzen.