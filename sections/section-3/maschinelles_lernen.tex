\subsection{Maschinelles Lernen beim Wissenserwerb}\label{subsec: Maschinelles-Lernen}
In diesem Abschnitt wird ein Ausblick in weitere Wissenserwerbsmethoden gegeben, und zwar die Anwendung des maschinellen Lernens. In der Literatur gibt es zahlreiche Ans�tze, die dieses Themenfeld bearbeiten. Im Rahmen dieser Arbeit wurden die Ans�tze aus \cite{castro2001} und \cite{webb1996} betrachtet. Trotz den Unterschiede in der Umsetzung wird die Zusammenarbeit zwischen dem Experten und dem Lernalgorithmus betont. Im Folgenden wird der Ansatz von \cite{castro2001} genauer betrachtet.\\
In der Arbeit von \cite{castro2001} handelt es sich um die Anwendung des maschinellen Lernens bei der Erfassung von Expertenwissen im Medizinbereich. Die Idee des Lernverfahrens besteht in der Erschlie�ung des naheliegenden Expertenwissens aus den Trainingsdaten der Schlussfolgerungen eines Experten. Das Konzept der Wissensbasiserweiterung orientiert sich an \cite{tecuci1992} und besteht darin, dass anfangs eine initiale Wissensbasis aufgebaut und inkrementell verbessert wird, indem der Experte die Fragen vom System beantwortet. Bei der Erstellung der Fragen wird ein Lernverfahren eingesetzt.\\
Der Kern des Verfahrens besteht aus dem induktiven Lernen und dem generischen Algorithmus von \cite{castro1999}, der auf der Fuzzylogik basiert. Die Fuzzylogik besch�ftigt sich mit den graduellen Aussagen, die nicht eindeutig als falsch oder wahr bezeichnet werden k�nnen. Diese Aussagen werden mithilfe der vagen Pr�dikate beschrieben. Ein typisches Beispiel ist das Pr�dikat \glqq{}gro�\grqq{}. Eine 1,80 m gro�e Person wird als \glqq{}gro�\grqq{} bezeichnet, w�hrend eine 1,75 m Person nicht genau so \glqq{}gro�\grqq{}, aber auch nicht \glqq{}klein\grqq{} ist \cite[S.27]{beierle2014}.\\
Das Lernverfahren umfasst folgenden Schritte \cite[S.316]{castro2001}:
\begin{itemize}
\item[1.] Eingabe der Trainingsmenge \( \theta \).
\item[2.] Entfernen von Noise aus den Trainingsdaten.
\item[3.] Ermittlung der Gene, die sich nicht �ndern k�nnen.
\item[4.] Ermittlung der Menge der initialen Regeln mithilfe des Algorithmus von \cite{castro1999}.
\item[5.] Anwendung des Algorithmus von \cite{castro1999} zur Ermittlung der Menge der Regeln, die das naheliegende Expertenwissen aus der Trainingsmenge beschreiben.
\item[6.] Ausschluss der Regeln, die ein Bestandteil anderer Regeln sind.
\end{itemize}
Aufgrund der hohen Komplexit�t und des spezifischen Anwendungsbereichs des Verfahrens wird sich im Weiteren auf den ersten Schritt beschr�nkt, der das grundlegende Verst�ndnis zu dem vorliegenden Ansatz liefert.\\
Als erstes definieren Castro et al.\cite[S.309]{castro2001} die Wissensrepr�sentation. Es werden folgende Informationstypen unterschieden:
\begin{itemize}
\item \textit{Numerische Information}, die aus einer empirischen Beobachtung stammt und als eine Beispielmenge von Input-Output-Beziehungen erfasst wird.
\item \textit{Sprachinformation}, die von einem Experten stammt und in Form von IF-THEN-Regeln beschrieben wird.
\end{itemize}
Die Trainingsmenge wird folgenderma�en definiert (siehe Formeln \ref{eq:trainingsmenge} und \ref{eq:training}):
\begin{equation}
	\theta = \lbrace e_1 \dots e_m \rbrace
	\label{eq:trainingsmenge}
\end{equation}
\begin{equation}
	e_i = ((x_{i0}, \dots , x_{in}), y_j)
	\label{eq:training}
\end{equation}
wobei:
\begin{itemize}
\item[\( e_i \)] = eine Schlussfolgerung,
\item[\( m \)] = Anzahl der vorhandenen Schlussfolgerungen,
\item[\( x_{in} \)] = Input-Variable,
\item[\( n \)] = Anzahl der Input-Variablen in der Schlussfolgerung,
\item[\( y_j \)] = Output-Wert (Expertenschlussfolgerung).
\end{itemize}
Die Output-Werte nach der Formel \ref{eq:training} stellen die Teilmenge des kartesischen Produkts aller Input- und Output-Werte \(X^n \times Y \) dar. Das Ziel ist die Approximation der Funktion \( \Omega: X^n \rightarrow Y \) durch die Ermittlung von Fuzzy-IF-THEN-Regeln, um das naheliegende Expertenwissen aus der Trainingsmenge zu erschlie�en. Fuzzy-IF-THEN-Regel \( R_j \) wird wie folgt definiert (siehe Formel \ref{eq:fuzzy-regel}):
\begin{equation}
	R_j : \mbox{ IF } X \mbox{ is } E \mbox{ THEN } Y \mbox{ is } y_i 
	\label{eq:fuzzy-regel}
\end{equation}
wobei:
\begin{itemize}
\item[\( X \)] = Menge der Variablen in den Schlussfolgerungen,
\item[\( E \)] = Teilmengen der Fuzzy-Labels,
\item[\( Y \)] = Expertenschlussfolgerung.
\end{itemize}
Castro et al. \cite[S.309]{castro2001} treffen die Annahme, dass die Teilmengen der Fuzzy-Labels \( E \) aus einer endlichen Menge der Labels $\mathcal{L}$ entnommen sind (siehe Formel \ref{eq:fuzzy-labels}):
\begin{equation}
	\mathcal{L} = \lbrace L_{i1}, \dots , L_{ik} \rbrace
	\label{eq:fuzzy-labels}
\end{equation}
Dabei ist \( k \) die Anzahl der Labels und \( E_i \) ist eine Expertenschlussfolgerung, die ein Element der Funktionsmenge \( \mathcal{P}(\mathcal{L}_i) \) ist, oder zusammenfassend \( E_i \in \mathcal{P}(\mathcal{L}_i) \).\\
Das Ergebnis der Durchf�hrung des Lernverfahrens ist die Menge von Regeln, die bei der Erstellung der Fragen verwendet wird. Dabei kann die Differenzstrategie eingesetzt werden \cite[S.317]{castro2001}. Bei der Differenzstrategie werden die Regeln gesucht, die zu einem Konflikt f�hren, wenn sie gleichzeitig angewandt werden. Um den Konflikt zu l�sen, kann der Experte eine neue Variable zulassen. Alternativ kann der Experte eine neue Meta-Regel definieren, wie z.B wenn \( R_i \) und \( R_j \) sich widersprechen, soll \( R_i \) bevorzugt werden, da es die allgemeinere Klasse repr�sentiert \cite[S.318]{castro2001}.