\section{Fazit}\label{sec:Fazit}
Im folgenden Abschnitt handelt es sich um die Zusammenfassung der Ergebnisse, die im Zuge dieser Arbeit entstanden sind. Allgemein lassen sich die Ergebnisse aus der theoretischen und der praktischen Sicht betrachten.\\
Im theoretischen Teil wurden zun�chst die Grundlagen von wissensbasierten bzw. Expertensystemen umfassend analysiert. 
Im Weiteren wurde der Schwerpunkt auf die genaue Untersuchung der Wissenserwerbskomponente gelegt. Daraus ist ein Modell der Wissenserwerbs entstanden (siehe Abbildung \ref{fig:wissenserwerbskomponente}), das drei Kategorien des Wissenserwerbs unterscheidet, n�mlich indirekter, direkter und automatisierter Wissenserwerb. Dar�ber hinaus wurde die Komponente zur Daten�bermittlung definiert, die als Bindeglied zwischen den Wissenserwerbsmethoden und der Wissensbasis auftritt.\\
Bez�glich der Wissenserwerbsmethoden wurden zahlreiche Arbeiten untersucht, die sich mit direkten und automatisierten Wissenserwerbsmethoden besch�ftigen. Meistens handelte es sich um den Einsatz vom maschinellen Lernen beim automatisieren Wissenserwerb. In \cite{tecuci1992} und \cite{castro2001} wurde allerdings betont, dass das maschinelle Lernen in Kooperation mit einem fachlichen Experten am besten funktioniert. Ein anderer Ansatz wird in \cite{gebus2009} vorgestellt. In dieser Fallstudie liegt der Fokus auf der Mensch-Computer-Interaktion. Um Erfahrungswissen aus der Produktionsabteilung zu erfassen, verwenden die Autoren eine graphische Benutzerschnittstelle zur Dateneingabe. Ein anderer Bereich, in dem die Datenerfassung eine zunehmende Bedeutung hat, stellt das Web dar. Dieses Interesse ist in der ersten Linie durch die enorm schnelle Entwicklung des Web und damit verbundenen Datenmengen begr�ndet. Einen umfassend �berblick kann man aus der Abrbeit von Ferrara et al \cite{ferrara2014} entnehmen.\\
Im praktischen Teil geht es um die Anwendung der theoretischen Erkenntnisse auf \textit{PaaSfinder}. Als erstes wurde die Idee der Wissenstr�gerschnittstelle umgesetzt, sodass der Nutzer der Wissensbasis von \textit{PaaSfinder} ohne Informatikvorkenntnisse beitragen kann. Die Implementierung umfasst zwei Teile. Im ersten Teil wurde das Frontend entwickelt, das die Daten dynamisch aktualisieren l�sst und im lokalen Speicher f�r sp�tere �bermittlung verwaltet. Im zweiten Teil wurde ein Service implementiert, der f�r die Weiterleitung der Daten an die Wissensbasis (das Git-Repository) zust�ndig ist. Der Service verwendet wiederum die API von Github, um die Daten abzuschicken.\\
Im Weiteren wurden die M�glichkeiten untersucht, die potentiell bei der Automatisierung der Datenerfassung bei \textit{PaaSfinder} benutzt werden k�nnen. Allgemein wurden drei Informationsquellen betrachtet, n�mlich die Webseiten von PaaS-Anbietern, Web-Feeds und soziale Netzwerke. Die Webseiten bieten einen Informationsmehrwert, wenn es sich um einen nicht erfassten PaaS-Anbieter handelt. F�r die Aktualisierungen auf der Webseite sind eher zeitlich geordnete Web-Feeds geeignet. Alternativ k�nnen die Updates aus sozialen Netzwerken (z.B. Facebook, Twitter etc.) bezogen werden.