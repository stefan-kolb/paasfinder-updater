\section{Fazit}\label{sec:Fazit}
Zuletzt werden die Ergebnisse der Arbeit zusammengefasst. Des Weiteren werden die Probleme angesprochen, die in zuk�nftige Arbeiten behandelt werden k�nnen. \\
Im theoretischen Teil der Arbeit wurde ein Modell des automatisierten Wissenserwerbs erarbeitet (siehe Abbildung \ref{fig:wissenserwerbskomponente}). Das Modell sieht vor, dass die Datenerfassung je nach Anwendungsbereich unterschiedlich automatisierbar ist. Aus diesem Grund wurden drei Kategorien des Wissenserwerbs unterschieden: indirekter, direkter und automatisierter Wissenserwerb. Dar�ber hinaus wurde die Komponente zur Daten�bermittlung definiert, die als Bindeglied zwischen den Wissenserwerbsmethoden und der Wissensbasis auftritt. \\
Die Anwendung des theoretischen Modells wurde am Beispiel von \textit{PaaSfinder} behandelt. Gem�� der Zielsetzung wurde im ersten Schritt die Benutzerschnittstelle zur Aktualisierung eines Vendors entwickelt. Als N�chstes wurde ein Service implementiert, der f�r die automatische Erstellung von Pull Requests zust�ndig ist. Des Weiteren wurden verschiedene Datenquellen untersucht, die bei der Automatisierung der Datenerfassung in zuk�nftiger Arbeit eingesetzt werden k�nnen.\\
In Bezug auf zuk�nftige Arbeit gibt es folgende Herausforderungen: Erstens sollen Verfahren entwickelt werden, die die Daten aus unterschiedlichen Quellen sinnvoll zusammenfassen und persistent speichern. Zweitens soll es erm�glicht werden, relevante Informationen aus den erfassten Daten maschinell zu ziehen. Dabei besteht Potential f�r die Anwendung von maschinellem Lernen. Drittens soll die Erweiterung der automatischen Tests in Zukunft st�rker ber�cksichtigt werden, um die Datenbank konsistent und fehlerfrei zu halten.