\section{Fazit}\label{sec:Fazit}
Im folgenden Abschnitt werden die Ergebnisse der Arbeit zusammengefasst und ein Ausblick auf Richtungen in Future Work gegeben.\\
In Bezug auf den theoretischen Teil der Arbeit wurde ein Modell des automatisierten Wissenserwerbs erarbeitet (siehe Abbildung \ref{fig:wissenserwerbskomponente}). Das Modell sieht vor, dass die Datenerfassung je nach dem Anwendungsbereich unterschiedlich automatisierbar ist. Aus diesem Grund wurden drei Kategorien des Wissenserwerbs unterschieden, n�mlich indirekter, direkter und automatisierter Wissenserwerb. Dar�ber hinaus wurde die Komponente zur Daten�bermittlung definiert, die als Bindeglied zwischen den Wissenserwerbsmethoden und der Wissensbasis auftritt. \\
Im praktischen Teil wurde das Modell auf \textit{PaaSfinder} angewandt. Hinsichtlich der Zielsetzung wurde im ersten Schritt die Benutzerschnittstelle zur Aktualisierung eines Vendors entwickelt. Als n�chstes wurde ein Service implementiert, der f�r die automatische Erstellung von Pull Requests zust�ndig ist. \\
Potentielle Forschungsrichtungen wurden in Abschnitt \ref{subsec:future_work} angesprochen. Im Hinblick auf die Updates spielt der Kurznachrichtendienst Twitter eine wichtige Rolle. Twitter bietet au�erdem eine REST API an, die f�r den Zugriff auf die Daten benutzt werden kann. Im Weiteren verf�gen Blogs �ber Informationen, die f�r \textit{PaaSfinder} relevant sein k�nnen. Die Datenerfassung kann bei Blogs mithilfe von Web-Crawling erfolgen. Dabei besteht der Spielraum f�r die Anwendung des maschinellen Lernens auf die von Twitter und Blogs erfassten Daten.\\
Die Erweiterung der automatischen Tests soll in Zukunft ebenso ber�cksichtigt werden, um die Datenbank konsistent und fehlerfrei zu halten.