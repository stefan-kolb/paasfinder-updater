\subsection{Zielsetzung}\label{subsec:Zielsetzung}
Das Ziel der vorliegenden Arbeit ist die Erarbeitung eines allgemeinen Konzeptes zur Automatisierung der Datenerfassung. Da die komplett automatisierte Datenerfassung sehr schwierig umzusetzen ist, liegt der Schwerpunkt dieser Arbeit auf der Kombination zwischen den manuellen und maschinellen Vorgehensweisen.\\
Beim technischen Kontext wird angedeutet, dass die Umsetzung im Rahmen eines bestimmten Anwendungsbereichs erfolgt. Es wird also kein Allgemeinwissen wie in \cite{tandon2016}, sondern ein anwendungsbezogenes Wissen betrachtet. In diesem Zusammenhang wird das Konzept auf Basis eines Expertensystems entwickelt, da die Wissensbasis eines Expertensystems spezifischer ist als bei einem wissensbasierten System.\\
Die praktische Umsetzung soll auf Basis von \textit{PaaSfinder}\footnote{https://paasfinder.org} erfolgen. Bei \textit{PaaSfinder} handelt es sich um eine Web-Anwendung, die �ber eine Wissensdatenbank im Bereich \ac{PaaS} verf�gt. \ac{PaaS} geh�rt neben \ac{IaaS} und \ac{SaaS} zum Themengebiet von Cloud Computing \cite{nist2011} und soll die Anwendungsentwicklung erleichtern, indem die Laufzeit- ober Entwicklungsumgebungen von einem \ac{PaaS}-Anbieter dem Kunden vorkonfiguriert angeboten werden \cite[S.14]{lawton2008}. Da \textit{PaaSfinder} ein Open-Source Projekt ist, kann jeder der Datenbank von \textit{PaaSfinder} beitragen. Allerdings ist die Mitwirkung mit einem hohen Aufwand verbunden und setzt Informatikkenntnisse voraus. Au�erdem werden die Daten von \textit{PaaSfinder} haupts�chlich auf den Webseiten von \ac{PaaS}-Anbietern manuell erfasst. Dies stellt ein hohes Entwicklungspotential, die Erfassung von solchen Daten zu automatisieren.