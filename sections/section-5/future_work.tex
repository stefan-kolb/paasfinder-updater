\subsection{Future Work}\label{subsec:future_work}
Im folgenden Abschnitt werden zuk�nftig m�gliche Arbeitsrichtungen in Bezug auf die Automatisierung der Datenerfassung von \textit{PaaSfinder} erl�utert, die auf den Erkenntnissen aus \ref{subsec:Wissenstr�gerschnittstelle}, \ref{subsec:Worker} und \ref{subsec:vergleich} basieren. \\
Als Weiterentwicklung der Arbeit aus \ref{subsec:Wissenstr�gerschnittstelle} soll Hilfestellung in das Interface eingebaut werden. Da es nicht immer f�r alle Nutzer ersichtlich ist, was ein bestimmtes Feld bedeutet, sollte neben jedem Feld ein Element (z.B. Fragezeichen Tooltip) hinzugef�gt werden, das eine kurze Erkl�rung enth�lt.\\
Zun�chst soll die REST API von Twitter genauer untersucht werden. Da Twitter enorm verbreitet ist, lassen sich Nachrichten vieler Vendors auf einmal erfassen. Die Aufgabe besteht darin, die Daten nach Vendors zu sortieren. Als N�chstes k�nnten die Daten nach einem Muster ausgewertet werden. Wenn beispielsweise das Wort \glqq{}Ruby\grqq{} vorkommt, ist die Wahrscheinlichkeit hoch, dass es sich um ein Update der unterst�tzen Version von Ruby handelt. Hier besteht auch Potential f�r das Anwenden von maschinellem Lernen, indem die Trainingsmengen von Twitter ausgewertet werden. \\
Im Weiteren soll das Potential von Blogs ausgenutzt werden. Die Artikel, die in Blogs pub\-liziert werden, k�nnen zus�tzliche Informationen und externe Links auf wichtige Ressourcen enthalten. Dabei bietet es sich an, einen Web-Crawler zu benutzen. Wenn ein Newsletter allgemein oder im Kontext eines Blogs angeboten wird, kann diese M�glichkeit in Anspruch genommen werden. Ansonsten kann ein Service entwickelt werden, der den Update Reminder an verantwortliche Personen der Firmen via E-Mail nach einem bestimmten Zeitraum nach dem letzten Update verschickt.\\
Schlie�lich sollen die automatischen Tests von \textit{PaaSfinder} erweitert werden, um die Konsistenz der Daten sicherzustellen. Bei der Implementierung der Wissenstr�gerschnittstelle stellte sich als h�ufiges Problem heraus, dass einige Felder auf null gesetzt werden, was zu Problemen bei der Darstellung der Daten f�hrt.