\subsection{Future Work}\label{subsec:future_work}
Im folgenden Abschnitt werden zuk�nftige Arbeitsrichtungen in Bezug auf die Automatisierung der Datenerfassung von \textit{PaaSfinder} erl�utert, die auf den Erkenntnissen der Arbeit aus \ref{subsec:Wissenstr�gerschnittstelle}, \ref{subsec:Worker} und \ref{subsec:vergleich} basieren. \\
Als Weiterentwicklung der Arbeit aus \ref{subsec:Wissenstr�gerschnittstelle} sollen die Hilfestellungen in das Interface eingebaut werden. Da es nicht immer f�r alle ersichtlich ist, was ein bestimmtes Feld bedeutet, sollte neben einem Feld ein Element (z.B. Fragezeichen Tooltip) hinzugef�gt werden, das eine kurze Erkl�rung zum Feld enth�lt.\\
Zun�chst soll die REST API von Twitter genauer untersucht werden. Da Twitter enorm verbreitet ist, lassen sich Nachrichten vieler Vendors auf einmal erfassen. Die Aufgabe besteht darin, die Daten nach Vendor zusammenzustellen. Als n�chstes k�nnen die Daten nach einem Pattern ausgewertet werden. Wenn beispielsweise das Wort \glqq{}Ruby\grqq{} vorkommt, ist die Wahrscheinlichkeit hoch, dass es sich um ein Update der unterst�tzen Version von Ruby handelt. Hier besteht auch Potential f�r das Anwenden des maschinellen Lernens, indem die Trainingsmengen von Twitter ausgewertet werden. \\
Im Weiteren soll das Potential von Blogs ausgenutzt werden. Die Artikel, die in Blogs publiziert werden, k�nnen zus�tzliche Information und externe Links auf wichtige Ressourcen enthalten. Dabei bietet es sich an, einen Web-Crawler zu benutzen. Wenn ein Newsletter allgemein oder im Kontext eines Blogs angeboten wird, soll diese M�glichkeit in Anspruch genommen werden. Ansonsten kann ein Service entwickelt werden, der den Update Reminder an verantwortliche Personen der Firmen via E-Mail nach einem bestimmten Zeitraum nach dem letzten Update verschickt.\\
Schlie�lich sollen die automatischen Tests von \textit{PaaSfinder} erweitert werden, um die Konsistenz der Daten sicherzustellen. Bei der Implementierung der Wissenstr�gerschnittstelle stellte sich als h�ufiges Problem heraus, dass einige Felder auf null gesetzt werden, was zu Problemen bei der Darstellung der Daten f�hrt.