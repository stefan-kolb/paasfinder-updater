\subsection{Abgrenzung der Datenquellen}\label{subsec:Abgrenzung der Datenquellen}
Die Auswahl der Datenquellen erfolgt unter der Annahme, dass die Datenerfassung sich auf die Aktualisierung der bestehenden Vendors beschr�nkt. Der Grund daf�r besteht darin, dass die maschinelle Erfassung von einem neuen Vendor eine fachliche Expertise eines Experten erfordert. \\
Insgesamt k�nnen folgende Datenquellen relevant sein:
\begin{itemize}
\item Webseiten der Vendors 
\item Web-Feeds
\item Soziale Netzwerke 
\item Newsletters
\item Blogs
\end{itemize}
Die erste Quelle, die bei der Datenerfassung eines Vendors in Frage kommt, ist der Webauftritt des betrachteten Vendors. Zur Datenerfassung kann ein Web-Crawler benutzt werden, der ausgehend von einem Startlink (Seed) und einem Schl�sselwort die Webseite durchsucht \cite[S.32]{croft2010}. Weitere Seiten ergeben sich durch Verfolgen der Links auf der besuchen Seite. Im diesem Anwendungsfalls ist der Ansatz weniger sinnvoll, da die Mehrheit der Daten bereits erfasst wurde. Es kann auch nicht ausgeschlossen werden, dass neue Daten dazu kommen. Allerdings sind zu diesem Zweck daf�r vorgesehene Services mehr geeignet wie Web-Feeds.\\
Bei Web-Feeds handelt es sich um \glqq{}Content Syndication\grqq{}, was als Bereitstellung von Daten f�r �bertragung, Aggregierung und Online-Publikation\footnote{http://web.resource.org/rss/1.0/} bezeichnet werden kann. Ein verbreitetes Beispiel f�r Web-Feeds ist RSS. Die Abk�rzung steht je nach Quelle f�r \glqq{}RDF Site Summary\grqq{}\footnote{http://web.resource.org/rss/1.0/spec}, \glqq{}Really Simple Syndikation\grqq{}\footnote{https://validator.w3.org/feed/docs/rss2.html} und \glqq{}Rich Site Summary\grqq{}. Es wird zwischen Push und Pull RSS unterschieden. Bei Push RSS werden die Benachrichtigung vom Absender angesto�en. Bei Pull RSS soll man die Benachrichtigen manuell abrufen. In der Praxis werden Pull RSS am meisten eingesetzt. Ein RSS-Dokument wird in XML beschrieben, was sich positiv auf das Parsen der Inhalte auswirkt. Au�erdem enthalten RSS Eintr�ge das Erstellungsdatum, das f�r die Zuordnung der Aktualisierung hilfreich ist. Ein RSS Beispiel von Heroku\footnote{https://www.heroku.com/} wird in Listing \ref{rss} dargestellt.
\begin{lstlisting}[basicstyle=\ttfamily, breaklines=true, label=rss,
					captionpos=b, caption={Ein RSS Beispiel von Heroku}]
<rss xmlns:dc="http://purl.org/dc/elements/1.1/" version="2.0">
 <channel>
  <title>Heroku</title>
  <link>http://blog.heroku.com</link>
  <description>The Heroku Blog</description>
  <item>
   <title>The Heroku-16 Stack is Now Available</title>
   <link>https://blog.heroku.com/heroku-16-is-generally-available</link>
   <pubDate>Thu, 20 Apr 2017 15:06:00 GMT</pubDate>
   <guid>https://blog.heroku.com/heroku-16-is-generally-available</guid>
   <description>
    <p>Your Heroku applications run on top of ...</p> 
   </description>
   <author>Jon Byrum</author>
  </item>
 </channel>
</rss>
\end{lstlisting}
Neben RSS k�nnen soziale Netzwerke f�r das Beziehen von Updates verwendet werden. Twitter\footnote{https://twitter.com/} ist ein Beispiel daf�r. Der Vorteil von diesem Kurznachrichtendienst besteht darin, dass es f�r den Vendor keinen Entwicklungsaufwand mitbringt. Zum Abrufen der Daten bietet Twitter eigene REST API\footnote{https://dev.twitter.com/rest/public} an.\\
Newsletters und Blogs k�nnen ebenso als Datenquellen benutzt werden. Beim Abonnieren von Newsletter werden die Updates an die E-Mail Adresse regelm��ig vom Vendor verschickt. Ein Blog ist eher f�r die Nutzer gedacht, die das Informationsbedarf im technischen Bereich haben. Ein Blog Post ist meist ein kurzer Artikel, der einen Update enthalten kann. Aus diesem Grund werden die Blogs ebenso in die Betrachtung genommen.