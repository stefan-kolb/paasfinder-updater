\subsection{Abgrenzung der Informationsquellen}\label{subsec:Informationsquellen}
Allgemein beschr�nkt sich die Datenerfassung wie im Kapitel \ref{sec:Umsetzung} auf die Aktualisierung der bestehenden Eintr�gen der Wissensbasis von \textit{PaaSfinder}. Dabei lassen sich folgende Informationsquellen identifizieren: 
\begin{itemize}
\item Webseiten von PaaS-Anbietern
\item Web-Feeds (z.B. News-Feeds)
\item Soziale Netzwerke 
\end{itemize}
Die Analyse der Webseiten w�re die naheliegende Vorgehensweise, neue Daten �ber einen PaaS-Anbieter zu erschlie�en. Die Wissensbasis von \textit{PaaSfinder} beinhaltet bereits die Homepage-URLs, die als Ausgangspunkt (Seed) bei einem Web-Crawler benutzt werden k�nnen. Auch die Fachliteratur bietet zahlreiche Informationen bez�glich der Umsetzung eines Web-Crawlers (z.B. \cite[S.35-50]{croft2010}). Allerdings macht das Webseiten-Crawling aus folgenden Gr�nden wenig Sinn:
\begin{itemize}
\item \textit{Webseitenstruktur}. Unter den PaaS-Anbietern gibt es keine einheitliche Sicht, wie die Informationen auf der Webseiten strukturiert werden k�nnen.
\item \textit{Datenverarbeitung (Parsen)}. Bei der kleinsten �nderung der Seitenstruktur kann die gesamte Anfrage (meist HTML-Tag-Selektoren) zusammengebrochen werden.
\item \textit{Neuheit der Information}. Im Kontext von bestehenden Daten ist es h�chstwahrscheinlich, dass die Daten bereits erfasst wurden. 
\end{itemize}
In Bezug auf Parsen der Webseiten wurde im Abschnitt \ref{subsec:webdaten} das Konzept von Wrapper angesprochen. Allerdings ist an der Stelle die Verwendung der REST-API einfacher zu nutzen und zu warten. Das Argument mit der Neuheit der Information kann umstritten werden, wenn es um die Erfassung eines noch nicht in der Wissensbasis vorhandenen PaaS-Anbieters geht. Aber selbst in diesem Fall stellt sich die Frage der Glaubw�rdigkeit eines PaaS-Profils, das auf maschinelle Weise erstellt wurde.\\
Eine weitere Kategorie der Informationsquelle stellen Web-Feeds und soziale Netzwerke dar. Aus Sicht der Datenaktualit�t sind Web-Feeds und soziale Netzwerke hervorragend geeignet. Die Daten werden in kleinen zeitlich geordneten Informationseinheiten dargestellt. Au�erdem bieten einige PaaS-Anbieter einen Service, der sich auf ein bestimmtes Thema spezialisiert. Aufgrund dessen wird im Weiteren der Bereich von Web-Feeds und sozialen Netzwerken genauer betrachtet.