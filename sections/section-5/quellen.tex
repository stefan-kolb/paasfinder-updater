\subsection{Abgrenzung der Datenquellen}\label{subsec:Abgrenzung der Datenquellen}
Die Auswahl der Datenquellen erfolgt unter der Annahme, dass die Datenerfassung sich auf die Aktualisierung der bestehenden Vendors beschr�nkt. Der Grund daf�r besteht darin, dass die maschinelle Erfassung eines neuen Vendors die fachliche Expertise eines Experten erfordert. \\
Insgesamt k�nnen folgende Datenquellen relevant sein:
\begin{itemize}
\item Webseite des Vendors
\item Web-Feeds
\item Soziale Netzwerke 
\item Newsletter
\item Blogs
\end{itemize}
Die erste Quelle, die bei der Datenerfassung eines Vendors in Frage kommt, ist dessen Web\-auf\-tritt. Zur Datenerfassung kann ein Web-Crawler benutzt werden, der ausgehend von einem Startlink (Seed) und nach einem Suchbegriff die Webseite durchsucht \cite[S.32]{croft2010}. Als Seed wird die Homepage des Vendors verwendet. Die weiteren Seiten ergeben sich durch Verfolgen der Links auf der besuchen Seite. Allerdings ist dieser Ansatz hier weniger sinnvoll, da die Mehrheit der Daten in \textit{PaaSfinder} bereits vorliegt.\\
Bei Web-Feeds handelt es sich um \glqq{}Content Syndication\grqq{}, was als Bereitstellung von Daten f�r �bertragung, Aggregierung und Online-Publikation\footnote{http://web.resource.org/rss/1.0/} bezeichnet werden kann. Ein verbreitetes Beispiel f�r Web-Feeds ist RSS. Die Abk�rzung steht je nach Quelle f�r \glqq{}RDF Site Summary\grqq{}\footnote{http://web.resource.org/rss/1.0/spec}, \glqq{}Really Simple Syndikation\grqq{}\footnote{https://validator.w3.org/feed/docs/rss2.html} oder \glqq{}Rich Site Summary\grqq{}. Es wird zwischen Push und Pull RSS unterschieden. Bei Push RSS werden die Benachrichtigung vom Absender angesto�en. Bei Pull RSS soll der Nutzer die Benachrichtigungen manuell abrufen. In der Praxis werden Pull RSS am meisten eingesetzt. Ein RSS-Dokument wird in XML-Format beschrieben, was ein effizientes Parsen der Inhalte erm�glicht. Au�erdem enthalten RSS-Eintr�ge das Erstellungsdatum, das f�r die Zuordnung der Aktualisierung hilfreich ist. Ein RSS Beispiel von Heroku\footnote{https://www.heroku.com/} wird in Listing \ref{rss} dargestellt.
\begin{lstlisting}[basicstyle=\ttfamily, breaklines=true, label=rss,
					captionpos=b, caption={Ein RSS-Beispiel von Heroku}]
<rss xmlns:dc="http://purl.org/dc/elements/1.1/" version="2.0">
 <channel>
  <title>Heroku</title>
  <link>http://blog.heroku.com</link>
  <description>The Heroku Blog</description>
  <item>
   <title>The Heroku-16 Stack is Now Available</title>
   <link>https://blog.heroku.com/heroku-16-is-generally-available</link>
   <pubDate>Thu, 20 Apr 2017 15:06:00 GMT</pubDate>
   <guid>https://blog.heroku.com/heroku-16-is-generally-available</guid>
   <description>
    <p>Your Heroku applications run on top of ...</p> 
   </description>
   <author>Jon Byrum</author>
  </item>
 </channel>
</rss>
\end{lstlisting}
Neben RSS k�nnen soziale Netzwerke f�r das Beziehen von Updates verwendet werden. Twitter\footnote{https://twitter.com/} ist ein Beispiel daf�r. Der Vorteil dieses Kurznachrichtendienstes besteht darin, dass er f�r den Vendor keinen Entwicklungsaufwand mitbringt. Zum Abrufen der Daten bietet Twitter eine eigene REST API\footnote{https://dev.twitter.com/rest/public} an.\\
Newsletter und Blogs k�nnen ebenso als potentielle Datenquellen benutzt werden. Beim Abonnieren von Newslettern werden vom Vendor regelm��ig die Updates an die E-Mail Adresse verschickt. Au�erdem k�nnen Vendor Blogs zur Datenerfassung durchsucht werden. In einem Blog werden meist Informationen ver�ffentlicht, die f�r Entwickler interessant sind. Ein Blogeintrag ist meist ein kurzer Artikel, der ein Update enth�lt. Aus diesem Grund werden Blogs ebenso f�r die Datenerfassung in Betracht gezogen.