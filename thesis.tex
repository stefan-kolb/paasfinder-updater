% Einbindung der Konfigurationsdatei
\input{includes/config.tex}

\begin{document}

% Titelblatt erstellen
\maketitle{Bachelor}{Wirtschaftsinformatik}{Automatisierung der Datenerfassung f�r Wissensdatenbanken im technischen Kontext}{Vasilyev Petr}{31.03.2017}

% Erstellung der Inhaltsverzeichnisses
\pagenumbering{Roman}

\tableofcontents
\newpage

\listoffigures
\newpage

%\listoftablesD
%\newpage

%\lstlistoflistings
%\newpage

\fancyhead[LO]{\footnotesize\sc\nouppercase{Abk�rzungsverzeichnis}}
\section*{Abk�rzungsverzeichnis}
% In Klammern steht das l�ngste Akronym!
\begin{acronym}[PaaS]
 \acro{IaaS}{Infrastructure-as-a-Service}
 \acro{PaaS}{Platform-as-a-Service}
 \acro{SaaS}{Software-as-a-Service}
\end{acronym}
\newpage
\fancyhead[LO]{\footnotesize\sc\nouppercase{\leftmark}}
\setcounter{page}{1}
\pagenumbering{arabic}
%
% Hier einzelne Kapitel mit \input{sections/Kapitel-File} einf�gen.
% Bei gr��eren Arbeiten empfiehlt es sich, noch genauer aufzuteilen.
%
\section{Einf�hrung}\label{sec:Einf�hrung}
\subsection{Motivation}
Das Themengebiet von Cloud Computing, das aus den Servicemodellen \ac{IaaS}, \ac{PaaS} und \ac{SaaS} besteht \cite{nist2011}, geh�rt zu den am meisten versprechenden Themen der letzten Jahre. Insbesondere PaaS ist vom gro�en Interesse f�r die zukunftsorientierte Anwendungsentwicklung. Der Hauptvorteil f�r die Entwickler besteht darin, dass die Serverkonfiguration vom PaaS-Anbieter durchgef�hrt wird. Dies erm�glicht die Effizienzsteigerung der Anwendungsentwicklung, da die Programmierer mehr Zeit f�r den eigentlichen Entwicklungsprozess haben \cite[S.218]{kolb2014}. Die Reduzierung der Entwicklungskosten geh�rt laut George Lawton ebenso zu den Vorteilen von \acs{PaaS} \cite[S.14]{lawton2008}.\\
Momentan befindet sich der PaaS-Markt in einer Wachstumsphase, die duch eine hohe Anzahl von PaaS-Anbietern gekennzeichnet ist. In \cite{kolb2014} werden bereits 68 \acs{PaaS}-Anbietern erw�hnt. Dar�ber hinaus bieten verschiedene PaaS-Dienstleister ein unterschiedliches Angebot an Laufzeitumgebungen, Kapazit�ten und Systemkonfigurationen. Um einen �berblick �ber das vorhandenes PaaS-Angebot zu schaffen, wurde ein Expertensystem namens \textit{PaaSfinder}\footnote{https://paasfinder.org} entwickelt. Dabei handelt es sich um eine Web-Anwendung, die dem Nutzer die Suche nach der PaaS-Plattform erleichtern soll. Zu den wichtigsten Funktionen von \textit{PaaSfinder} z�hlen Filtersuche mittels Merkmalen der PaaS-Plattform und 1-zu-1 Vergleich zwischen den zwei PaaS-Anbietern. \textit{PaaSfinder} ist ein Open Source Projekt und steht frei f�r die Nutzung oder Verbesserungsvorschl�gen zur Verf�gung.\\
Da sich der PaaS-Bereich sehr schnell entwickelt, besteht f�r \textit{PaaSfinder} das Problem der Aktualit�t der vorliegenden Daten. Ein PaaS-Anbieter kann beispielsweise einen neuen Service zur Plattform hinzuf�gen oder Informationen im Bezug auf bestehende Dienstleistungen �ndern. Aus diesem Grund muss die Wissensbasis von \textit{PaaSfinder} regelm��ig aktualisiert und gepflegt werden, um konsistent, vollst�ndig und aktuell zu bleiben. Momentan werden die Daten f�r \textit{PaaSfinder} manuell erfasst und aktualisiert. Dies ist vor allem zeitaufw�ndig und fehleranf�llig. Au�erdem kann ein wichtiges Update leicht �bersehen werden.

\subsection{Zielsetzung}
Das Ziel dieser Bachelorarbeit ist die Entwicklung eines Konzeptes zur Automatisierung der Datenerfassung f�r Wissensdatenbanken. Dabei ist die Sicherstellung von Aktualit�t, Konsistenz und Fehlerfreiheit einer Wissensdatenbank von Bedeutung. Die praktische Umsetzung soll am Beispiel von \textit{PaaSfinder} erfolgen.\\
S\'{e}bastien Gebus betont, dass ein System mit generischen Wissenserfassungsmethoden f�r einen konkreten Anwendungsfall kaum einsetzbar ist \cite[S.100]{gebus2009}. Aus diesem Grund beschr�nkt sich die vorliegende Arbeit auf die Auswahl der Methoden, die f�r \textit{PaaSfinder} und dazugeh�rige Wissensbasis am meisten geeignet sind. \\
Ein weiteres Problem ist der Automatisierungsgrad der Wissenserfassungsmethoden. In mehreren Studien hat sich herausgestellt, dass ein halb-automatisierter Ansatz f�r die Wissenserfassung am meisten geeignet ist \cite{fujihara1997}, \cite{gebus2009}, \cite{tecuci1992}. So sind Hiroko Fujihara et al. zum Schluss gekommen, dass Wissenserfassungsprozess prim�r zu dem Aufgabenbereich eines Wissensingenieurs geh�rt. Die Methoden, die auf automatisierte Weise Daten erfassen, sollen dabei nur als Unterst�tzung dienen, um die Produktivit�t und Effizienz des Fachexperten zu erh�hen \cite[S.219]{fujihara1997}. Daher liegt der Fokus dieser Arbeit auf einen halb-automatisierten Ansatz, der automatisierte Datenerfassung und manuelle Kontrolle durch einen Experten kombinieren soll.

\subsection{Aufbau der Arbeit}
Die vorliegende Arbeit ist folgenderma�en aufgeteilt. In Kapitel 2 wird der Begriff eines Expertensystems und das grundlegende Konzept der Datenerfassung erl�utert. Anschlie�end wird definiert, welche Informationen f�r die Wissensbasis von \textit{PaaSfinder} notwendig sind. In Kapitel 3 handelt es sich um die Auswahl und Beschreibung geeigneter Wissenserfassungsmethoden im Hinblick auf \textit{PaaSfinder}. Die konkrete Umsetzung der ausgew�hlten Methoden wird in Kapitel 4 beschrieben. In Kapitel 5 werden die implementierten Methoden evaluiert. Dabei wird die N�tzlichkeit der Methoden bestimmt, in dem die gewonnenen Daten auf die Qualit�t ausgewertet werden. In Kapitel 6 handelt es sich um die Zusammenfassung des Endergebnis der durchgef�hrten Arbeit. Kapitel 7 gibt eine m�gliche Richtung f�r zuk�nftige Forschung im betrachteten Bereich.
\newpage
\section{Grundlagen von Expertensystemen}\label{sec:Grundlagen}
\subsection{Begriffsdefinition}\label{subsec:Begriffsdefinition}
Urspr�nglich waren Expertensysteme Anwendungsprogramme, die logische Schlussfolgerungen aus einer Wissensbasis ziehen konnten. Au�erdem konnten sie �berpr�fen, ob eine Aussage aus einer vorhandenen Wissensbasis abgeleitet werden kann \cite[S.75]{greer2010}. Daher handelt es sich in der fr�heren Literatur meist um Anwendungen, die ihr Wissen in Form von logischen Ausdr�cken darstellen und in der Lage waren, neue Erkenntnisse von bestehendem Wissen abzuleiten \cite{tecuci1992}. Im Laufe der Zeit hat sich das Konzept eines Expertensystems auf andere Anwendungsbereiche ausgeweitet. Aus diesem Grund gibt es mehrere Definitionen, die im Allgemeinen �hnlich sind und im Spezifischen Merkmale des zugeh�rigen Anwendungsbereichs beinhalten.\\
Allgemein l�sst sich sagen, dass ein Expertensystem ein Computersystem (Hardware und Software) ist, das in einem bestimmten Bereich Wissen und Schlussfolgerungsf�higkeit eines menschlichen Experten nachbildet \cite[S.12]{beierle2014}. Aus Sicht der Wirtschaftsinformatik zielen Expertensysteme darauf ab, das Expertenwissen menschlicher Fachleute in der Wissensbasis eines Computers abzuspeichern und f�r eine Vielzahl von Probleml�sungen zu nutzen \cite[S.59]{mertens2012}. Im Weiteren gehen Beierle und Kern-Isberner auf die Eigenschaften ein, die ein Expertensystem aufweisen sollen \cite[S.12]{beierle2014}. Im Rahmen dieser Arbeit sind folgende Eigenschaften besonders relevant:
\begin{itemize}
\item Anwendung des Wissens eines oder mehrerer Experten, um Probleme in einem bestimmten Anwendungsbereich zu l�sen,
\item Leicht lesbare Wissensdarstellung,
\item M�glichst anschauliche und intuitive Benutzerschnittstelle,
\item Leichte Wartbarkeit und Erweiterbarkeit des Wissens im Expertensystem,
\item Unterst�tzung beim Wissenstransfer vom Experten zum System.
\end{itemize}
Hier ist es au�erdem wichtig anzumerken, dass die Begriffe \grqq{}K�nstliche Intelligenz\grqq{}, \grqq{}wissensbasiertes System\grqq{} und \grqq{}Expertensystem\grqq{} in einer engen Beziehung zueinander stehen. Haun stellt eine systematische Abgrenzung dieser Begriffe vor, die sich folgenderma�en beschreiben l�sst \cite[S.30]{haun2000}:
\begin{itemize}
\item \textit{K�nstliche Intelligenz} stellt den Oberbegriff dar und bildet den theoretischen Rahmen f�r die Entwicklung von wissensbasierten Systemen und Expertensystemen.
\item \textit{Wissensbasierte Systeme} sind eine Teilmenge der Anwendungen innerhalb des Bereichs der k�nstlichen Intelligenz. Sie wenden die Wissensverarbeitung auf ein konkretes Aufgabengebiet an und verwalten Allgemeinwissen explizit und getrennt vom Rest des Systems.
\item \textit{Expertensysteme}, die ein Teilbereich der wissensbasierten Systeme sind, stellen eine Spezialisierung von wissensbasierten Systemen dar. Sie verwalten spezifisches Expertenwissen, das von einem Experten stammt und auf praxisbezogene Probleme angewandt wird.
\end{itemize}
Graphisch l�sst sich die vorliegende Abgrenzung in Abbildung \ref{Abgrenzung} darstellen:
\begin{figure}[H] 
	\centering
	\includegraphics[width=0.55\textwidth]{images/abgrenzung.png}
	\caption{Begriffsabgrenzung, \cite[S.30]{haun2000}}
	\label{Abgrenzung}
\end{figure} 
Nach dieser Abgrenzung l�sst sich feststellen, dass der Unterschied zwischen einem wissensbasierten System und einem Expertensystem darin besteht, dass das Wissen im Endeffekt von einem Experten stammt. Allerdings ist dieses Kriterium nicht besonders aussagekr�ftig. Beierle und Kern-Isberner weisen darauf hin, dass nach diesem Kriterium viele der existierenden wissensbasierten Systeme als Expertensysteme bezeichnet werden k�nnten \cite[S.11]{beierle2014}. Als Reaktion auf fehlende Kriterien stellen die Autoren die Eigenschaften eines Experten dar, die sich folgenderma�en zusammenfassen lassen:
\begin{itemize}
\item Experten sind selten und teuer.
\item Experten sind nicht immer verf�gbar.
\item Leistungsf�higkeit der Experten ist nicht konstant, sondern kann nach Tagesverlauf schwanken.
\item Expertenwissen kann oft nicht als solches weitergegeben werden.
\item Expertenwissen kann verloren gehen.
\end{itemize}
Ein gutes Beispiel hinsichtlich der Gefahr, dass Expertenwissen verloren gehen kann, wird in \cite[S.94]{gebus2009} vorgestellt. Gebus nimmt hier Bezug auf die Mitarbeiter der sogenannten Baby-Boomgeneration. Es handelt sich um Experten, die ein umfangreiches Erfahrungswissen besitzen und bald aus Altersgr�nden das Unternehmen verlassen. Somit geht auch das Erfahrungswissen aus dem Unternehmen verloren.\\
Zusammenfassend l�sst sich sagen, dass die Entwicklung eines Expertensystems ein hohes Potenzial besitzt. Allerdings kann ein Expertensystem nicht als Ersatz f�r einen menschlichen Experten betrachtet werden. Vielmehr geht es um eine Erfassung, Darstellung und Pflege des Expertenwissens in einem Expertensystem, um die Arbeitsprozesse effizienter zu gestalten und sowohl erfahrene als auch neue Anwender in einem bestimmten Wissensbereich bei der Aufgabenabwicklung zu unterst�tzen.
\subsection{Architektur eines Expertensystems}\label{subsec:Architektur}
Beierle und Kern-Isberner betonen, dass die Trennung zwischen der Darstellung des Wissens (Wissensbasis) und der Wissensverarbeitung (Wissensverarbeitungskomponente) der wichtigste Aspekt eines Wissensbasierten Systems ist. \cite[S.11]{beierle2014}. Die Wissensbasis kann man sich als eine Art Datenstruktur vorstellen, in der das ben�tigte Wissen gespeichert wird. Die Wissensverarbeitungskomponente umfasst eine Menge von anwendungsunabh�ngigen Algorithmen, die mithilfe der Wissensbasis eine L�sung f�r ein gegebenes Problem erarbeiten. Somit stehen die Wissensbasis und die Wissensverarbeitungskomponente in einer engen Beziehung zueinander \cite[S.18]{kurbel1992}.\\
Allgemein umfasst ein Expertensystem folgende Bestandteile \cite[S.75]{greer2010}:
\begin{itemize}
\item \textit{Wissensbasis}, die Expertenwissen in Form von Fakten in einer bestimmten Sprache speichert sowie Regeln zur Wissensorganisation beinhaltet.
\item \textit{Inferenzmaschine}, die unter Ber�cksichtigung des zugrunde liegenden Wissensbedarfs die Wissensbasis
durchsucht bis das System einen Probleml�sungsvorschlag erarbeitet hat oder herausfindet, dass keiner existiert.
\item \textit{Dialogkomponente}, die eine Schnittstelle zwischen dem Nutzer und dem System darstellt. 
\item \textit{Erkl�rungskomponente}, die dem Benutzer erl�utert, warum und auf welche Weise eine bestimmte L�sung gefunden bzw. nicht gefunden wurde \cite[S.126]{haun2000}.
\item \textit{Wissensakquisitionskomponente}, die den Entwickler des Expertensystems bei der Erweiterung, �nderung und Wartung der Wissensbasis unterst�tzt.
\end{itemize}
Laut Tecuci stellen Wissensbasis und Inferenzmaschine grundlegende Bestandteile eines Expertensystems dar und bilden damit den Kern des Expertensystems \cite[S.1444]{tecuci1992}. Dialogkomponente, Erkl�rungskomponente und Wissensakquisitionskomponente geh�ren zur sogenannten Schale und sind f�r die Kommunikation zwischen dem Systemverwalter und dem Nutzer zust�ndig (siehe Abbildung \ref{expertensystem_haun}). 
\begin{figure}[H] 
	\centering
	\includegraphics[width=1.0\textwidth]{images/expertensystem_haun.png}
	\caption{Architektur eines Expertensystems nach Haun, \cite[S.126]{haun2000}}
	\label{expertensystem_haun}
\end{figure}
Im Hinblick auf die Interaktion gibt es drei Gruppen, die mit dem Expertensystem interagieren: 
\begin{itemize}
\item \textit{Nutzer}, der das Expertensystem zum L�sen eines Problems benutzt und mit der Dialogkomponente kommuniziert. Der Wissensingenieur und der Experte k�nnen ebenso als Nutzer auftreten \cite[S.758]{wachsmuth1993}.
\item \textit{Wissensingenieur}, der sich mit dem Aufbau und Wartung der Wissensbasis besch�ftigt. Unter anderem ist Wissensmodellierung ein wichtiger Aufgabenbereich eines Wissensingenieurs \cite[S.742]{wachsmuth1993}.
\item \textit{Experte}, der �ber spezifisches Erfahrungswissen verf�gt, das f�r das Expertensystem relevant ist.
\end{itemize}
Der Ablauf der Kommunikation zwischen dem Nutzer und dem Expertensystem sieht folgenderma�en aus: 
\begin{itemize}
\item Der Nutzer schickt eine Anfrage an die Dialogkomponente des Expertensystems.
\item Die Dialogkomponente �bermittelt die Anfrage an die Inferenzmaschine.
\item Die Inferenzmaschine erarbeitet eine L�sung f�r das gegebene Problem mittels der Wissensbasis und gibt das Ergebnis an die Dialogkomponente zur�ck. 
\item Anschlie�end teilt die Dialogkomponente dem Nutzer die L�sung des Problems mit. Falls keine L�sung zum Problem existiert, wird eine entsprechende Fehlermeldung angezeigt.
\end{itemize}
Auf der anderen Seite k�nnen die Inhalte der Wissensbasis von einem Wissensingenieur mithilfe der Wissensakquisitionskomponente beeinflusst werden. Der Wissenserwerb durch den Wissensingenieur ist die verbreitetste Vorgehensweise, neue Daten f�r ein wissensbasiertes System zu erschlie�en. Meistens handelt es sich um ein Interview zwischen dem Wissensingenieur und dem Experten \cite[S.76]{greer2010}, \cite[S.210]{fujihara1997}. Neben dem Interview kann der Wissensingenieur eine Recherche der verf�gbaren Wissensquellen wie Text, technische Zeichnungen oder Web-Ressourcen durchf�hren. Anschlie�end werden die Daten vom Wissensingenieur formalisiert und in die Wissensbasis gespeichert. \\
Die Wissensbasis kann in einigen F�llen von einem fachlichen Experten beeinflusst werden. Daf�r ist eine geeignete Expertenschnittstelle innerhalb der Wissensakquisitionskomponente notwendig, die den Experten erm�glicht, ihr Erfahrungswissen selbst zu formalisieren und gegebenenfalls zu warten \cite[S.743]{wachsmuth1993}. Die �berpr�fung des Dateninputs ist ebenfalls die Aufgabe der Wissensakquisitionskomponente. Dies kann mittels Durchf�hrung automatisierten Tests bei jeder �nderungsanfrage erfolgen, um die Konsistenz der Wissensbasis zu gew�hrleisten \cite[S.743]{wachsmuth1993}.\\
Um ein geeignetes Konzept der automatisierten Datenerfassung zu entwickeln, ist ein grundlegendes Verst�ndnis von der Struktur und Funktionsweise der Wissensbasis sowie der Wissensakquisitionskomponente erforderlich. Im weiteren Verlauf der Arbeit werden die Erkenntnisse �ber die Wissensbasis und die Wissensakquisitionskomponente erl�utert, die in der Forschung von Expertensystemen entstanden sind.

\subsection{Wissensbasis}\label{subsec:Wissensbasis}
Neben der Inferenzmaschine stellt die Wissensbasis den zentralen Teil eines Expertensystems, der die Daten des gesamten Systems beinhaltet \cite[S.754]{wachsmuth1993}. Im Folgenden werden der allgemeine Prozess der Wissensbasisentwicklung, der Inhalt der Wissensbasis und die M�glichkeiten der Wissensrepr�sentation thematisiert. Gheorghe Tecuci beschreibt folgende Phasen bei der Entwicklung der Wissensbasis \cite[S.1444]{tecuci1992}: 
\begin{itemize}
\item[1.] Systematische Erfassung vom Expertenwissen
\item[2.] Verfeinerung der Wissensbasis
\item[3.] Reorganisation der Wissensbasis
\end{itemize}
In der ersten Phase werden das Vokabular und die geeignete Wissensrepr�sentation festgelegt. Gebus und Leivisk{\"a} betonen, dass die Wissensrepr�sentation den entscheidenden Einfluss auf die Generierung und sp�tere Handhabung der Wissensbasis hat \cite[S.95]{gebus2009}. Die initialen Daten werden meistens im Rahmen eines Interviews zwischen dem Wissensingenieur und dem Experten erfasst \cite[S.1444]{tecuci1992}. Das Ergebnis der ersten Phase ist eine initiale Wissensbasis, die unvollst�ndig und teilweise widerspr�chlich ist. In der zweiten Phase wird die initiale Wissensbasis mithilfe der geeigneten Datenerfassungsmethoden solange erweitert und verbessert, bis sie vollst�ndig und korrekt genug ist, um ein gegebenes Problem richtig zu l�sen. In der dritten Phase wird die vollst�ndige und korrekte Wissensbasis reorganisiert, um die Effizienz der L�sungsberechnung zu steigern \cite[S.1445]{tecuci1992}. Zusammenfassend werden die Phasen in der Abbildung \ref{drei_phasen} dargestellt.  
\begin{figure}[H] 
	\centering
	\includegraphics[width=0.7\textwidth]{images/drei_phasen.png}
	\caption{Phasen der Expertensystementwicklung, \cite[S.138]{tecuci1994}}
	\label{drei_phasen}
\end{figure}
In der Abbildung \ref{drei_phasen} sieht man, dass der Autor dem Experten die gesamte Kontrolle �ber die Entwicklung der Wissensbasis zuweist. Allerdings ist diese Sichtweise nicht vollst�ndig, da im Entwicklungsprozess der Wissensingenieur und der Systementwickler beteiligt sind und dementsprechend ber�cksichtigt werden sollen.\\
In Bezug auf den Inhalt der Wissensbasis unterscheiden Beierle und Kern-Isberner folgende Wissensarten \cite[S.5]{beierle2014}:
\begin{itemize}
\item \textit{Fachspezifisches Wissen}. Dabei handelt es sich um das spezifischste Wissen, das sich
nur auf den gerade betrachteten Problemfall bezieht. Das sind z.B. Fakten, die von Beobachtungen oder Untersuchungsergebnissen stammen.
\item \textit{Regelhaftes Wissen}, das den eigentlichen Kern der Wissensbasis darstellt. Dieses Wissen kann noch genauer differenziert werden: 
	\begin{itemize}
	\item \textit{Bereichsbezogenes Wissen}, das sich auf den gesamten Problembereich beziehen. Das kann sowohl theoretisches Fachwissen als auch Erfahrungswissen sein. Anders gesagt handelt es sich um generisches Wissen.
	\item \textit{Allgemeinwissen}, das z.B. um generelle Probleml�sungsheuristiken, Optimierungsregeln oder auch allgemeines Wissen �ber Objekte und Beziehungen in der realen Welt beinhaltet.
	\end{itemize}
\end{itemize}
Unter Ber�cksichtigung der Differenzierung der Wissensarten innerhalb der Wissensbasis beschreiben die Autoren in \cite[S.18]{beierle2014} auf eigene Weise die Architektur des Expertensystems, die in der Abbildung \ref{expertensystem_beierle}  dargestellt wird.
\begin{figure}[H] 
	\centering
	\includegraphics[width=0.7\textwidth]{images/expertensystem_beierle.png}
	\caption{Expertensystem nach Beierle und Kern-Isberner, \cite[S.18]{beierle2014}}
	\label{expertensystem_beierle}
\end{figure} 
Laut Beierle und Kern-Isberner k�nnen verschiedene Wissensarten in einem wissensbasierten System je nach dem Anwendungsbereich unterschiedlich umfangreich auftreten. Ein hochspezialisiertes System kann beispielsweise �ber sehr wenig oder gar kein Allgemeinwissen verf�gen. Auf der anderen Seite kann ein wissensbasiertes System den Schwerpunkt auf das gew�hnliche Alltagswissen legen \cite[S.5-6]{beierle2014}.\\
Ein weiterer Aspekt beim Aufbau der Wissensbasis ist die Wissensrepr�sentation. Die grundlegende Aufgabe der Wissensrepr�sentation ist die Formularisierung von Wissen, um eine maschinelle Verarbeitung  erst zu erm�glichen \cite[S.22]{haun2000}. Sinz und Ferstl unterscheiden folgende Formen der Wissensrepr�sentation \cite[S.366]{sinz2013}:
\begin{itemize}
\item \textit{Regelorientierte Darstellung}, in der das Wissen in Form von WENN-DANN-Regeln beschrieben wird. Diese Darstellungsform wird beispielsweise bei Prolog-Regeln eingesetzt.
\item \textit{Objektorientierte Darstellung}, die das Konzept der Objekttypen �bernimmt und mit deklarativen Operatorbeschreibungen verbindet.
\item \textit{Constraints Darstellung}, die Modellbeschreibungen aus dem Operations Research benutzt. Dabei handelt es sich um L�sungsr�ume durch Nebenbedingungen und Zielvorgaben.   
\end{itemize}
Hinsichtlich der Wissensrepr�sentation stellen Ferstl und Sinz imperative und deklarative Paradigmen gegen�ber \cite[S.366]{sinz2013}. Ein Programm, das dem imperativen Paradigma folgt, besteht aus einer Folge von Befehlen, die nacheinander ausgef�hrt werden \cite[S.341]{sinz2013}. Bei einem deklarativen Programm handelt es sich um eine Beschreibung der Aufgabenau�ensicht. Ein deklaratives Programm hat keine festgelegten L�sungsverfahren je Aufgaben. Stattdessen wird eine L�sung zum Zeitpunkt der Aufgabendurchf�hrung mittels Inferenzmaschine abgeleitet \cite[S.361]{sinz2013}.\\
Allgemein beziehen sich die Autoren darauf, dass an ein wissensbasiertes System nur geringe Anforderungen bez�glich Vollst�ndigkeit, Widerspruchsfreiheit und Eindeutigkeit gestellt werden k�nnen. Aus diesem Grund ist das deklarative Paradigma f�r die Wissensrepr�sentation besser geeignet. Folgende Gr�nde nennen die Autoren f�r die deklarative Umsetzung der Wissensbasis \cite[S.366]{sinz2013}: 
\begin{itemize}
\item \textit{Wissensdarstellung}: Da ein Mensch das Erfahrungswissen durch assoziative Beziehungsmuster aufbaut, ist die deklarative Wissensdarstellung eher geeignet.
\item \textit{Wissensauswertung}: �nderungen von Erfahrungswissen werden normalerweise in deklarativen Form erfasst.
\item \textit{Wissensverf�gbarkeit}: Die Codewartung von einem imperativen Programm ist fehleranf�llig, kosten- und zeitintensiv, da das Erfahrungswissen h�ufig ge�ndert und aktualisiert werden muss.
\end{itemize}
Der objektorientierte Ansatz ist eine weitere M�glichkeit, das Wissen zu beschreiben. Ein Beispiel f�r die objektorientierte Implementierung wird in \cite{leung1990} vorgestellt. Die Wissensbasis wird dabei als eine Sammlung von Klassen, Objekten und Methoden definiert \cite[S.40]{leung1990}. Der gro�e Vorteil solcher Umsetzung besteht in der Modularit�t des Wissens. Das Wissen in unabh�ngige Module aufgeteilt wird. Da die einzelnen Module unabh�ngig voneinander sind, k�nnen sie getrennt getestet und modifiziert werden, ohne den Rest der Wissensbasis zu beeintr�chtigen. Dies erm�glicht hohe Flexibilit�t bei der Wissensbasiserweiterung \cite[S.43]{leung1990}.\\
Neben der Implementierung der Wissensbasis ist eine geeignete Umsetzung der Wissenserwerbskomponente erforderlich, um die Wissensbasis aktuell, m�glichst fehlerfrei und konsistent zu halten. Im Folgenden wird die Wissenserwerbskomponente in Hinsicht auf den allgemeinen Aufbau und Funktionen thematisiert.  
\input{sections/kapitel-2/wissensakquisitionskomponente.tex}
\newpage
\section{Automatisierung der Datenerfassung}\label{sec:Datenerfassung}
Allgemein l�sst sich sagen, dass bei der Wissensbasiserweiterung ein hybrides Modell sinnvoll dass eine Kooperation zwischen einem menschlichen Experten und automatisierten Methoden darstellt. Die Autoren in \cite{tecuci1994} weisen ebenso darauf hin, dass manuelle und maschinelle Wissenserschlie�ung jeweils eigene St�rke haben, die in einem hybriden Ansatz kombiniert werden k�nnen\cite[S.137]{tecuci1994}.\\
In Bezug auf die Wissensbasisentwicklung k�nnen folgende Phasen unterschieden werden \cite[S.1444]{tecuci1992}: 
\begin{itemize}
\item[1.] Systematische Erfassung vom Expertenwissen
\item[2.] Verfeinerung der Wissensbasis
\item[3.] Reorganisation der Wissensbasis
\end{itemize}
Die erste Phase umfasst die Festlegung der grunds�tzlichen Struktur der Wissensbasis, die im Rahmen eines indirekten Wissenserwerbs stattfindet. Das strukturierte Interview wird dabei oft eingesetzt \cite[S.1444]{tecuci1992}. Das Ergebnis der ersten Phase ist eine initiale Wissensbasis, die unvollst�ndig und teilweise falsch ist. In der zweiten Phase wird die initiale Wissensbasis mithilfe der geeigneten Datenerfassungsmethoden solange erweitert und verbessert, bis sie vollst�ndig und korrekt genug ist, um ein gegebenes Problem richtig zu l�sen. In der dritten Phase wird die vollst�ndige und korrekte Wissensbasis reorganisiert, um die Effizienz der L�sungsberechnung zu steigern \cite[S.1445]{tecuci1992}. Zusammenfassend lassen sich die Phasen in der Abbildung \ref{drei_phasen} darstellen: 
\begin{figure}[H] 
	\centering
	\includegraphics[width=0.65\textwidth]{images/drei_phasen.png}
	\caption{Phasen der Expertensystementwicklung, \cite[S.138]{tecuci1994}}
	\label{drei_phasen}
\end{figure}
Die Methoden der Automatisierung der Datenerfassung, die im weiteren Verlauf behandelt werden, beziehen sich auf die zweite Phase der Wissensbasisentwicklung.

\subsection{Die Wissenstr�gerschnittstelle}
Stark an die Arbeit von \cite{gebus2009} orientiert.

\subsection{Maschineller Wissenserwerb}
Eine Auswahl an generellen Methoden

\subsection{Schnittstelle f�r Validierung und Speicherung}
Zweck, Konzept
\section{Implemetierung}\label{sec:Implementierung}
\subsection{Ansatz 1}
\subsection{Ansatz 2}
\subsection{Ansatz 3}
\section{Evaluation}\label{sec:Evaluation}
\subsection{Test 1}
\subsection{Test 2}
\subsection{Test 3}
\section{Fazit und Ausblick}\label{sec:Fazit und Ausblick}
\input{sections/kapitel-7.tex}
\newpage

% Einstellungen f�r Literaturverzeichnis
\addcontentsline{toc}{section}{\bibname}
\bibliographystyle{geralpha}
\selectlanguage{german}

% Hier das Literaturverzeichnis einbinden
\bibliography{bibliography/references}
\newpage

% Eigenst�ndigkeitserkl�rung
\makedeclaration{Bachelorarbeit}{31.03.2017}{Petr Vasilyev}

\end{document}