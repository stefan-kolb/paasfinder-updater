% Einbindung der Konfigurationsdatei
\input{includes/config.tex}

\begin{document}

% Titelblatt erstellen
\maketitle{Bachelorarbeit}{Wirtschaftsinformatik}{Automatisierung der Datenerfassung f�r Wissensdatenbanken im technischen Kontext}{Vasilyev Petr}{17.05.2017}

% Erstellung der Inhaltsverzeichnisses
\pagenumbering{Roman}

\tableofcontents
\newpage

\listoffigures
\newpage

%\listoftablesD
%\newpage

\lstlistoflistings
\newpage

\fancyhead[LO]{\footnotesize\sc\nouppercase{Abk�rzungsverzeichnis}}
\section*{Abk�rzungsverzeichnis}
% In Klammern steht das l�ngste Akronym!
\begin{acronym}[PaaS]
 \acro{IaaS}{Infrastructure-as-a-Service}
 \acro{PaaS}{Platform-as-a-Service}
 \acro{SaaS}{Software-as-a-Service}
\end{acronym}
\newpage
\fancyhead[LO]{\footnotesize\sc\nouppercase{\leftmark}}
\setcounter{page}{1}
\pagenumbering{arabic}

%
% Einzelne Kapitel mit \input{sections/Kapitel-File} einf�gen.
%
\section{Einf�hrung}\label{sec:Einf�hrung}
\subsection{Motivation}\label{subsec:Motivation}
Die Idee von wissensbasierten Systemen entstand aus dem Bed�rfnis, ein intelligentes System zu schaffen, das mittels spezifischen Wissens die Fachexperten bei den Problem\-l�\-sung\-en unterst�tzt \cite[S.18]{akerkar2010}. Eine Spezialisierung wissensbasierter Systeme stellen Expertensysteme dar, in denen das Wissen letztendlich von Experten entsprechender Dom�nen stammt. Ein wissensbasiertes System, bzw. ein Expertensystem, ist durch die Trennung der Wissensdarstellung eines Problembereichs und der Wissensverarbeitung gekennzeichnet \cite[S.11]{beierle2014}.\\ 
Die Beschreibung des Wissens eines wissensbasierten bzw. eines Expertensystems erfolgt in der Wissensbasis, die in Form einer Wissensdatenbank realisiert wird. Die Anfangswissensbasis wird in der Regel mithilfe manueller Datenerfassungen aufgebaut \cite[S.70]{kurbel1992}. Hierzu wird beispielsweise h�ufig die Methode des Interviews zwischen einem Wissens\-ingenieur und einem Wissenstr�ger eingesetzt \cite[S.76]{gottlob1990}. Allerdings sind die ma\-nu\-ell\-en Wissenserwerbsmethoden f�r die Weiterentwicklung der Wissensbasis hinsichtlich der Erweiterung und Aktualisierung der Wissensdatenbank eher schlecht geeignet, da sie fehleranf�llig sowie kosten- und zeitintensiv sind. Aus diesen Gr�nden gibt es Bestrebungen, den Prozess der Datenerfassung zu automatisieren. Eine Auswahl an bestehenden An\-s�tzen, die f�r die vorliegende Arbeit relevant sind, wird im Abschnitt \ref{subsec:Verwandte-Arbeiten} gegeben. Trotz der vielen Herangehensweisen gibt es kein einheitliches Konzept, das einen allgemeinen Rahmen f�r die Automatisierung der Datenerfassung bildet.
\subsection{Verwandte Arbeiten}\label{subsec:Verwandte-Arbeiten}
Der Ausgangspunkt dieser Arbeit stellt die Publikation \cite{tecuci1992} von G. Tecuci dar, die die Automatisierung der Wissenserfassung als ein Konzept der Erweiterung, Aktualisierung und Verbesserung der Wissensbasis beschreibt \cite[S.1444]{tecuci1992}. In diesem Zusammenhang wurde ein lernendes System vorgestellt, das eine Auswahl an generischen Ans�tzen des maschinellen Lernens bei der Wissenserfassung umsetzt \cite[S.1445]{tecuci1992}. Ferner wird ein Framework gebildet, das die Wissenserfassung durch maschinelles Lernen automatisiert. Dar�ber hinaus werden die gelernten Daten von einem Experten auf die Korrektheit �berpr�ft. Die Entwicklung der Wissensbasis wird in drei Phasen durchgef�hrt. In der ersten Phase wird die Anfangswissensbasis aufgebaut, die unvollst�ndig und teilweise widerspr�chlich sein kann. Die zweite Phase umfasst die inkrementelle Erweiterung und Verbesserung der Wissensbasis. Schlie�lich wird in der dritten Phase die Wissensbasis in Bezug auf Effizienz optimiert \cite[S.1444]{tecuci1992}. Die Kernaussage der Arbeit besteht darin, dass die Kooperation zwischen dem fachlichen Experten und dem Expertensystem in jeder Phase die Automatisierung der Datenerfassung deutlich erleichert. Beispielsweise k�nnen die Daten von einem Algorithmus generiert werden. Daraufforgend werden sie vom fachlichen Experten auf die formale und semantische Korrekteit �berpr�ft und in die Wissensbasis gespeichert \cite[S.1445]{tecuci1992}. In \cite{tecuci1994} wird die Entwicklung des Frameworks fortgef�hrt, wobei der Schwerpunkt im Bereich vom multi-strategischen Lernen (multistrategy leraning) liegt \cite[S.137]{tecuci1994}.\\
Neben \cite{tecuci1992} und \cite{tecuci1994} gibt es eine Reihe unterschiedlicher Ans�tze, die das maschinelle Lernen zur Automatisierung der Datenerfassung benutzen. Einige Beispiele sind \cite{castro1999}, \cite{castro2001}, \cite{webb1996}. Dabei ist die Idee der Zusammenarbeit zwischen dem Experten und dem Lernalgorithmus durchaus verbreitet. Ein Beispiel stellt die Arbeit von Castro et al. \cite{castro2001} dar. In der grundlegenden Idee beziehen sich die Autoren auf die Arbeit von \cite{tecuci1992}. Als Startpunkt wird eine unvollst�ndige Anfangswissensbasis betrachtet, die schrittweise verbessert wird, indem der Experte die Fragen vom System beantwortet. Bei der Frageerstellung wird ein Lernalgorithmus eingesetzt, der aus einer Trainingsmenge die Regeln lernt und kontinuierlich die Qualit�t der Fragen verbessert. Dabei betonen Castro et al. \cite{castro2001}, dass der Lernalgorithmus keineswegs den Wissensingenieur ersetzen kann. Vielmehr soll der Algorithmus den Wissensingenieur die Routinearbeit abnehmen und bei den schwierigeren Aufgaben unterst�tzen, indem verschiedene Varianten der Interviews vom Algorithmus vorgeschlagen werden \cite[S.308]{castro2001}.\\
Ein praxisorientierter Ansatz f�r die Autormatisierung der Wissenserfassung wird in \cite{gebus2009} thematisiert. Allgemein handelt es sich um die Transformation eines datenbasierten Systems in ein wissensbasiertes System, um die Effizient der Produktion zu steigern. Im Hinblick auf die Automatisierung der Aktualisierung und Erweiterung der unvollst�ndigen Wissensbasis beziehen sich Gebus und Leivisk{\"a} auf die Erkenntnisse aus \cite{tecuci1992}, \cite{winter1992} und \cite{su2002}. Bez�glich der Erfassung von Erfahrungswissen nehmen die Autoren den Bezug auf die Arbeit von Okamura et al \cite{okamura1991}, die Heuristik bei Probleml�sungen eingesetzt. Im praktischen Teil wird ein bereits bestehendes datenbasiertes \acf{DSS} betrachtet, das die Unternehmensf�hrung bei den Entscheidungen in Bezug auf die Produktionsoptimierung unterst�tzen soll. Allerdings werden die St�rungen in der Produktion von Anlagenbedienern (Experten) mithilfe von Erfahrungswissen intern behoben, ohne dieses Wissen weiterzugeben. Als Folge hat die Unternehmensf�hrung kein umfangreiches Bild der Produktion, was sich negativ auf die Produktion auswirkt. Aus diesem Grund erweitern Gebus und Leivisk{\"a} das System um eine Wissenstr�gerschnittstelle, um das Expertenwissen in die Datenbank zu integrieren \cite[S.94]{gebus2009}. Gebus und Leivisk{\"a} veranschaulichen damit, wie die Idee der Zusammenarbeit zwischen dem Experten und dem wissensbasierten System zur Automatisierung der Wissenserfassung im Kontext eines Unternehmens umgesetzt werden kann.\\
Mit der rasanten Entwicklung des Word Wide Web hat sich eine Forschungsrichtung ergeben, die sich mit der Daten- und Wissenserfassung aus Netzwerken unterschiedlicher Art befasst. Ein fundierter �berblick �ber die Webdatenerfassung sowie aktuelle Ans�tze und Anwendungen wurde in \cite{ferrara2014} vorgestellt. Dabei werden sowohl theoretische als auch praktische Aspekte umfassend thematisiert. In der theoretischen Hinsicht wurden zuerst allgemeine Probleme wie Automatisierungsgrad, Skalierbarkeit, Datenschutz, Unstabilit�t der Ressourcenstruktur und Trainingsmenge angesprochen \cite[S.301-302]{ferrara2014}. In Bezug auf die Ans�tze wurden Baumparadigma (z.B. \cite{dave2003} und \cite{wang1998}), Web Wrapper (z.B. \cite{sahuguet1999}) und hybrides System (z.B. \cite{crescenzi2001}) durch die Analyse zahlreicher Publikationen systematisiert. Bei der Anwendungen zur Webdatenerfassung w�chst der Trend Richtung freier Open-Source Projekte, die mit kommerziellen Anwendungen im Wettbewerb stehen \cite[S.310]{ferrara2014}. Die Autoren nennen als Beispiel die Pipes\footnote{https://en.wikipedia.org/wiki/Yahoo!\_Pipes} von Yahoo. Bedauerlicherweise ist das ein veraltetes Beispiel, da die Plattform nicht mehr unterst�tzt wird. Die Tatsache best�tigt nun die Aussage �ber die hohe Dynamik im Webtechnologiebereich. Nichtsdestotrotz gibt es einige Anwendungen, die �ber mehrere Jahre hinweg �berleben und enorm beliebt sind. Ein Beispiel stellen Feed-Services RSS/Atom das, die die Pull-Benachrichtigen �ber die �nderungen auf der Webseite, Blogs usw. erm�glichen (mehr dazu in \cite{hammersley2005} und \cite{tilkov2015}). Bezogen auf Anwendungen, die im Forschungsrahmen entstanden sind, wird nehmen die Autoren in \cite{ferrara2014} Lixito aus \cite{baumgartner2001} als Beispiel.


%Die Autoren thematisieren die Datenextraktion in unterschiedlichen Anwendungsbereichen.
%In diesem Zusammenhang wurden zahlreiche Systeme in unterschiedlichen Anwendungsgerechten wie Business Intelligence, Web-Crawling oder sogar Bioinformatik ausgewertet.
\subsection{Zielsetzung}\label{subsec:Zielsetzung}
Das Ziel der vorliegenden Arbeit ist die Erarbeitung eines allgemeinen Konzeptes zur Automatisierung der Datenerfassung. Da die komplett automatisierte Datenerfassung sehr schwierig umzusetzen ist, liegt der Schwerpunkt dieser Arbeit auf der Kombination zwischen den manuellen und maschinellen Vorgehensweisen.\\
Beim technischen Kontext wird angedeutet, dass die Umsetzung im Rahmen eines bestimmten Anwendungsbereichs erfolgt. Es wird also kein Allgemeinwissen wie in \cite{tandon2016}, sondern ein anwendungsbezogenes Wissen betrachtet. In diesem Zusammenhang wird das Konzept auf Basis eines Expertensystems entwickelt, da die Wissensbasis eines Expertensystems spezifischer ist als bei einem wissensbasierten System.\\
Die praktische Umsetzung soll am Beispiel von \textit{PaaSfinder}\footnote{https://paasfinder.org} erfolgen. Bei \textit{PaaSfinder} handelt es sich um eine Web-Anwendung, die eine Wissensdatenbank im Bereich \ac{PaaS} verwaltet. Das Ziel von \ac{PaaS} besteht in der Erleichterung der Anwendungsentwicklung, indem die Entwicklungsumgebung von einem \ac{PaaS}-Anbieter als ein konfigurierbarer Service angeboten wird \cite[S.14]{lawton2008}. Aufgrund der hohen Anzahl von \ac{PaaS}-Anbietern, Vielzahl von Einstellungsm�glichkeiten und potentiellen Inkompatibilit�ten zwischen den unterschiedlichen Anbietern gibt es einen Bedarf an einen systematischen Marktvergleich, der mithilfe von Daten der \ac{PaaS}-Anbieter erfolgt. Die Daten von \textit{PaaSfinder} wurden bisher haupts�chlich manuell erfasst, was m�hsam, zeit- und kostenintensiv ist. Das Ziel dieser Arbeit besteht darin, die f�r den Anwendungsfall von \textit{PaaSfinder} Methoden zur Automatisierung der Datenerfassung zu erforschen und umzusetzen. Es werden folgende Ideen vorgenommen. Als erstes soll eine Benutzerschnittstelle zur Korrektur der bestehenden Daten entwickelt werden. Momentan erfordert eine Aktualisierung der Daten Informatikvorkenntnisse, was viele Leute daran hindert, einen Update zu erstellen. Als n�chstes soll die Erstellung eines Pull Requests mit den Daten von der Benutzerschnittstelle automatisiert werden. Ferner sollen weitere M�glichkeiten erforscht werden, die Daten automatisch zu erfassen. Als Beispiel k�nnen soziale Netzwerke eingesetzt werden. Schlie�lich sollen die automatischen Tests erweitert werden, um die Konsistenz der Daten sicherzustellen. 
\subsection{Aufbau der Arbeit}\label{subsec:Aufbau-der-Arbeit}
Die vorliegende Arbeit ist wie folgt aufgebaut. In Kapitel 2 wird der Begriff und die grundlegende Architektur eines Expertensystems erl�utert. Darauffolgend werden die Bestandteile, die f�r das Konzept relevant sind, n�her betrachtet. Anschlie�end wird schematisch das Kontext der Datenerfassung dargestellt und Automatisierungsm�glichkeiten angesprochen. In Kapitel 3 wird genauer auf die Automatisierungsmethoden bei der Datenerfassung eingegangen. Dabei werden bestehende Forschungsergebnisse vorgestellt und konzeptuell verallgemeinert. Kapitel 4 umfasst die praktische Umsetzung der Wissenstr�gerschnittstelle am Beispiel von \textit{PaaSfinder}. In Kapitel 5 wird ein Ausblick in weitere M�glichkeiten bei der Automatisierung der Datenerfassung von \textit{PaaSfinder} gegeben. Anschlie�end wird das Ergebnis der Arbeit im Fazit zusammengefasst.
\newpage
\section{Grundlagen von Expertensystemen}\label{sec:Grundlagen}
\subsection{Begriffsdefinition}\label{subsec:Begriffsdefinition}
Urspr�nglich waren Expertensysteme Anwendungsprogramme, die logische Schlussfolgerungen aus einer Wissensbasis ziehen konnten. Au�erdem konnten sie �berpr�fen, ob eine Aussage aus einer vorhandenen Wissensbasis abgeleitet werden kann \cite[S.75]{greer2010}. Daher handelt es sich in der fr�heren Literatur meist um Anwendungen, die ihr Wissen in Form von logischen Ausdr�cken darstellen und in der Lage waren, neue Erkenntnisse von bestehendem Wissen abzuleiten \cite{tecuci1992}. Im Laufe der Zeit hat sich das Konzept eines Expertensystems auf andere Anwendungsbereiche ausgeweitet. Aus diesem Grund gibt es mehrere Definitionen, die im Allgemeinen �hnlich sind und im Spezifischen Merkmale des zugeh�rigen Anwendungsbereichs beinhalten.\\
Allgemein l�sst sich sagen, dass ein Expertensystem ein Computersystem (Hardware und Software) ist, das in einem bestimmten Bereich Wissen und Schlussfolgerungsf�higkeit eines menschlichen Experten nachbildet \cite[S.12]{beierle2014}. Aus Sicht der Wirtschaftsinformatik zielen Expertensysteme darauf ab, das Expertenwissen menschlicher Fachleute in der Wissensbasis eines Computers abzuspeichern und f�r eine Vielzahl von Probleml�sungen zu nutzen \cite[S.59]{mertens2012}. Im Weiteren gehen Beierle und Kern-Isberner auf die Eigenschaften ein, die ein Expertensystem aufweisen sollen \cite[S.12]{beierle2014}. Im Rahmen dieser Arbeit sind folgende Eigenschaften besonders relevant:
\begin{itemize}
\item Anwendung des Wissens eines oder mehrerer Experten, um Probleme in einem bestimmten Anwendungsbereich zu l�sen,
\item Leicht lesbare Wissensdarstellung,
\item M�glichst anschauliche und intuitive Benutzerschnittstelle,
\item Leichte Wartbarkeit und Erweiterbarkeit des Wissens im Expertensystem,
\item Unterst�tzung beim Wissenstransfer vom Experten zum System.
\end{itemize}
Hier ist es au�erdem wichtig anzumerken, dass die Begriffe \grqq{}K�nstliche Intelligenz\grqq{}, \grqq{}wissensbasiertes System\grqq{} und \grqq{}Expertensystem\grqq{} in einer engen Beziehung zueinander stehen. Haun stellt eine systematische Abgrenzung dieser Begriffe vor, die sich folgenderma�en beschreiben l�sst \cite[S.30]{haun2000}:
\begin{itemize}
\item \textit{K�nstliche Intelligenz} stellt den Oberbegriff dar und bildet den theoretischen Rahmen f�r die Entwicklung von wissensbasierten Systemen und Expertensystemen.
\item \textit{Wissensbasierte Systeme} sind eine Teilmenge der Anwendungen innerhalb des Bereichs der k�nstlichen Intelligenz. Sie wenden die Wissensverarbeitung auf ein konkretes Aufgabengebiet an und verwalten Allgemeinwissen explizit und getrennt vom Rest des Systems.
\item \textit{Expertensysteme}, die ein Teilbereich der wissensbasierten Systeme sind, stellen eine Spezialisierung von wissensbasierten Systemen dar. Sie verwalten spezifisches Expertenwissen, das von einem Experten stammt und auf praxisbezogene Probleme angewandt wird.
\end{itemize}
Graphisch l�sst sich die vorliegende Abgrenzung in Abbildung \ref{Abgrenzung} darstellen:
\begin{figure}[H] 
	\centering
	\includegraphics[width=0.55\textwidth]{images/abgrenzung.png}
	\caption{Begriffsabgrenzung, \cite[S.30]{haun2000}}
	\label{Abgrenzung}
\end{figure} 
Nach dieser Abgrenzung l�sst sich feststellen, dass der Unterschied zwischen einem wissensbasierten System und einem Expertensystem darin besteht, dass das Wissen im Endeffekt von einem Experten stammt. Allerdings ist dieses Kriterium nicht besonders aussagekr�ftig. Beierle und Kern-Isberner weisen darauf hin, dass nach diesem Kriterium viele der existierenden wissensbasierten Systeme als Expertensysteme bezeichnet werden k�nnten \cite[S.11]{beierle2014}. Als Reaktion auf fehlende Kriterien stellen die Autoren die Eigenschaften eines Experten dar, die sich folgenderma�en zusammenfassen lassen:
\begin{itemize}
\item Experten sind selten und teuer.
\item Experten sind nicht immer verf�gbar.
\item Leistungsf�higkeit der Experten ist nicht konstant, sondern kann nach Tagesverlauf schwanken.
\item Expertenwissen kann oft nicht als solches weitergegeben werden.
\item Expertenwissen kann verloren gehen.
\end{itemize}
Ein gutes Beispiel hinsichtlich der Gefahr, dass Expertenwissen verloren gehen kann, wird in \cite[S.94]{gebus2009} vorgestellt. Gebus nimmt hier Bezug auf die Mitarbeiter der sogenannten Baby-Boomgeneration. Es handelt sich um Experten, die ein umfangreiches Erfahrungswissen besitzen und bald aus Altersgr�nden das Unternehmen verlassen. Somit geht auch das Erfahrungswissen aus dem Unternehmen verloren.\\
Zusammenfassend l�sst sich sagen, dass die Entwicklung eines Expertensystems ein hohes Potenzial besitzt. Allerdings kann ein Expertensystem nicht als Ersatz f�r einen menschlichen Experten betrachtet werden. Vielmehr geht es um eine Erfassung, Darstellung und Pflege des Expertenwissens in einem Expertensystem, um die Arbeitsprozesse effizienter zu gestalten und sowohl erfahrene als auch neue Anwender in einem bestimmten Wissensbereich bei der Aufgabenabwicklung zu unterst�tzen.
\subsection{Architektur eines Expertensystems}\label{subsec:Architektur}
Beierle und Kern-Isberner betonen, dass die Trennung zwischen der Darstellung des Wissens (Wissensbasis) und der Wissensverarbeitung (Wissensverarbeitungskomponente) der wichtigste Aspekt eines Wissensbasierten Systems ist. \cite[S.11]{beierle2014}. Die Wissensbasis kann man sich als eine Art Datenstruktur vorstellen, in der das ben�tigte Wissen gespeichert wird. Die Wissensverarbeitungskomponente umfasst eine Menge von anwendungsunabh�ngigen Algorithmen, die mithilfe der Wissensbasis eine L�sung f�r ein gegebenes Problem erarbeiten. Somit stehen die Wissensbasis und die Wissensverarbeitungskomponente in einer engen Beziehung zueinander \cite[S.18]{kurbel1992}.\\
Allgemein umfasst ein Expertensystem folgende Bestandteile \cite[S.75]{greer2010}:
\begin{itemize}
\item \textit{Wissensbasis}, die Expertenwissen in Form von Fakten in einer bestimmten Sprache speichert sowie Regeln zur Wissensorganisation beinhaltet.
\item \textit{Inferenzmaschine}, die unter Ber�cksichtigung des zugrunde liegenden Wissensbedarfs die Wissensbasis
durchsucht bis das System einen Probleml�sungsvorschlag erarbeitet hat oder herausfindet, dass keiner existiert.
\item \textit{Dialogkomponente}, die eine Schnittstelle zwischen dem Nutzer und dem System darstellt. 
\item \textit{Erkl�rungskomponente}, die dem Benutzer erl�utert, warum und auf welche Weise eine bestimmte L�sung gefunden bzw. nicht gefunden wurde \cite[S.126]{haun2000}.
\item \textit{Wissensakquisitionskomponente}, die den Entwickler des Expertensystems bei der Erweiterung, �nderung und Wartung der Wissensbasis unterst�tzt.
\end{itemize}
Laut Tecuci stellen Wissensbasis und Inferenzmaschine grundlegende Bestandteile eines Expertensystems dar und bilden damit den Kern des Expertensystems \cite[S.1444]{tecuci1992}. Dialogkomponente, Erkl�rungskomponente und Wissensakquisitionskomponente geh�ren zur sogenannten Schale und sind f�r die Kommunikation zwischen dem Systemverwalter und dem Nutzer zust�ndig (siehe Abbildung \ref{expertensystem_haun}). 
\begin{figure}[H] 
	\centering
	\includegraphics[width=1.0\textwidth]{images/expertensystem_haun.png}
	\caption{Architektur eines Expertensystems nach Haun, \cite[S.126]{haun2000}}
	\label{expertensystem_haun}
\end{figure}
Im Hinblick auf die Interaktion gibt es drei Gruppen, die mit dem Expertensystem interagieren: 
\begin{itemize}
\item \textit{Nutzer}, der das Expertensystem zum L�sen eines Problems benutzt und mit der Dialogkomponente kommuniziert. Der Wissensingenieur und der Experte k�nnen ebenso als Nutzer auftreten \cite[S.758]{wachsmuth1993}.
\item \textit{Wissensingenieur}, der sich mit dem Aufbau und Wartung der Wissensbasis besch�ftigt. Unter anderem ist Wissensmodellierung ein wichtiger Aufgabenbereich eines Wissensingenieurs \cite[S.742]{wachsmuth1993}.
\item \textit{Experte}, der �ber spezifisches Erfahrungswissen verf�gt, das f�r das Expertensystem relevant ist.
\end{itemize}
Der Ablauf der Kommunikation zwischen dem Nutzer und dem Expertensystem sieht folgenderma�en aus: 
\begin{itemize}
\item Der Nutzer schickt eine Anfrage an die Dialogkomponente des Expertensystems.
\item Die Dialogkomponente �bermittelt die Anfrage an die Inferenzmaschine.
\item Die Inferenzmaschine erarbeitet eine L�sung f�r das gegebene Problem mittels der Wissensbasis und gibt das Ergebnis an die Dialogkomponente zur�ck. 
\item Anschlie�end teilt die Dialogkomponente dem Nutzer die L�sung des Problems mit. Falls keine L�sung zum Problem existiert, wird eine entsprechende Fehlermeldung angezeigt.
\end{itemize}
Auf der anderen Seite k�nnen die Inhalte der Wissensbasis von einem Wissensingenieur mithilfe der Wissensakquisitionskomponente beeinflusst werden. Der Wissenserwerb durch den Wissensingenieur ist die verbreitetste Vorgehensweise, neue Daten f�r ein wissensbasiertes System zu erschlie�en. Meistens handelt es sich um ein Interview zwischen dem Wissensingenieur und dem Experten \cite[S.76]{greer2010}, \cite[S.210]{fujihara1997}. Neben dem Interview kann der Wissensingenieur eine Recherche der verf�gbaren Wissensquellen wie Text, technische Zeichnungen oder Web-Ressourcen durchf�hren. Anschlie�end werden die Daten vom Wissensingenieur formalisiert und in die Wissensbasis gespeichert. \\
Die Wissensbasis kann in einigen F�llen von einem fachlichen Experten beeinflusst werden. Daf�r ist eine geeignete Expertenschnittstelle innerhalb der Wissensakquisitionskomponente notwendig, die den Experten erm�glicht, ihr Erfahrungswissen selbst zu formalisieren und gegebenenfalls zu warten \cite[S.743]{wachsmuth1993}. Die �berpr�fung des Dateninputs ist ebenfalls die Aufgabe der Wissensakquisitionskomponente. Dies kann mittels Durchf�hrung automatisierten Tests bei jeder �nderungsanfrage erfolgen, um die Konsistenz der Wissensbasis zu gew�hrleisten \cite[S.743]{wachsmuth1993}.\\
Um ein geeignetes Konzept der automatisierten Datenerfassung zu entwickeln, ist ein grundlegendes Verst�ndnis von der Struktur und Funktionsweise der Wissensbasis sowie der Wissensakquisitionskomponente erforderlich. Im weiteren Verlauf der Arbeit werden die Erkenntnisse �ber die Wissensbasis und die Wissensakquisitionskomponente erl�utert, die in der Forschung von Expertensystemen entstanden sind.

\subsection{Wissensbasis}\label{subsec:Wissensbasis}
Neben der Inferenzmaschine stellt die Wissensbasis den zentralen Teil eines Ex\-per\-ten\-sys\-tems dar, der die Wissensdaten des gesamten Systems beinhaltet \cite[S.754]{wachsmuth1993}. Im Folgenden werden der allgemeine Prozess der Wissensbasisentwicklung, der Inhalt der Wissensbasis und die M�glichkeiten der Wissensrepr�sentation thematisiert. Gheorghe Tecuci beschreibt folgende Phasen bei der Entwicklung der Wissensbasis \cite[S.1444]{tecuci1992}: 
\begin{itemize}
\item Systematische Erfassung des Expertenwissens
\item Verfeinerung der Wissensbasis
\item Reorganisation der Wissensbasis
\end{itemize}
In der ersten Phase werden das Vokabular und die geeignete Wissensrepr�sentation festgelegt. Gebus und Leivisk{\"a} betonen, dass die Wissensrepr�sentation einen entscheidenden Einfluss auf die Generierung und sp�tere Handhabung der Wissensbasis hat \cite[S.95]{gebus2009}. Die initialen Daten werden meistens im Rahmen eines Interviews mit einem Experten erfasst \cite[S.1444]{tecuci1992}. Das Ergebnis der ersten Phase ist eine initiale Wissensbasis, die unvollst�ndig und teilweise widerspr�chlich sein kann. In der zweiten Phase wird die initiale Wissensbasis mithilfe geeigneter Datenerfassungsmethoden solange erweitert und verbessert, bis sie vollst�ndig und korrekt genug ist, um ein gegebenes Problem richtig zu l�sen. In der dritten Phase wird die vollst�ndige und korrekte Wissensbasis reorganisiert, um die Effizienz der L�sungsberechnung zu steigern \cite[S.1445]{tecuci1992}. Zusammenfassend werden die Phasen in Abbildung \ref{drei_phasen} dargestellt.  
\begin{figure}[H] 
	\centering
	\includegraphics[width=0.7\textwidth]{images/drei_phasen.png}
	\caption{Phasen der Expertensystementwicklung, \cite[S.138]{tecuci1994}}
	\label{drei_phasen}
\end{figure}
Aus Abbildung \ref{drei_phasen} wird deutlich, dass Tecuci dem Experten die gesamte Kontrolle �ber die Entwicklung der Wissensbasis zuweist. Allerdings ist diese Sichtweise nicht vollst�ndig, da im Entwicklungsprozess der Wissensingenieur und der Systementwickler beteiligt sind und dementsprechend ber�cksichtigt werden m�ssen.\\
In Bezug auf den Inhalt der Wissensbasis unterscheiden Beierle und Kern-Isberner fol\-gende Wissensarten \cite[S.5]{beierle2014}:
\begin{itemize}
\item \textit{Fachspezifisches Wissen}: Dabei handelt es sich um das spezifische Wissen, das sich
nur auf den gerade betrachteten Problemfall bezieht. Das sind z.B. Fakten, die von Beobachtungen oder Untersuchungsergebnissen stammen.
\item \textit{Regelhaftes Wissen}: Dieses Wissen stellt den eigentlichen Kern der Wissensbasis dar und kann noch genauer differenziert werden: 
	\begin{itemize}
	\item \textit{Bereichsbezogenes Wissen}, das sich auf den gesamten Problembereich bezieht. Das kann sowohl theoretisches Fachwissen, als auch Erfahrungswissen sein.
	\item \textit{Allgemeinwissen}, das z.B. generelle Probleml�sungsheuristiken, Optimierungsregeln oder auch allgemeines Wissen �ber Objekte und Beziehungen in der realen Welt beinhaltet.
	\end{itemize}
\end{itemize}
Unter Ber�cksichtigung der Differenzierung der Wissensarten innerhalb der Wissensbasis beschreiben die Autoren in \cite[S.18]{beierle2014} auf eigene Weise die Architektur des Expertensystems, die in Abbildung \ref{expertensystem_beierle}  dargestellt wird.
\begin{figure}[H] 
	\centering
	\includegraphics[width=0.70\textwidth]{images/expertensystem_beierle.png}
	\caption{Expertensystem nach Beierle und Kern-Isberner, \cite[S.18]{beierle2014}}
	\label{expertensystem_beierle}
\end{figure} 
Laut Beierle und Kern-Isberner k�nnen verschiedene Wissensarten in einem wissensbasierten System je nach dem Anwendungsbereich in unterschiedlichem Umfang auftreten. Ein hochspezialisiertes System kann beispielsweise �ber sehr wenig bis gar kein Allgemeinwissen verf�gen. Auf der anderen Seite kann ein wissensbasiertes System den Schwerpunkt auf gew�hnliches Alltagswissen legen \cite[S.5-6]{beierle2014}.\\
Ein weiterer Aspekt beim Aufbau der Wissensbasis ist die Wissensrepr�sentation. Deren grundlegende Aufgabe ist die Formalisierung von Wissen, um eine maschinelle Verarbeitung zu erm�glichen \cite[S.22]{haun2000}. Sinz und Ferstl unterscheiden folgende Formen der Wissensrepr�sentation \cite[S.366]{sinz2013}:
\begin{itemize}
\item \textit{Regelorientierte Darstellung}, in der das Wissen in Form von WENN-DANN-Regeln beschrieben wird. Diese Darstellungsform wird beispielsweise bei Prolog\footnote{F�r weitere Informationen siehe z.B. http://www.swi-prolog.org/}-Regeln eingesetzt.
\item \textit{Objektorientierte Darstellung}, die das Konzept der Objekttypen �bernimmt und mit deklarativen Operatorbeschreibungen verbindet.
\item \textit{Constraints Darstellung}, die Modellbeschreibungen aus dem Bereich des Operation Researchs benutzt. Dabei handelt es sich um L�sungsr�ume durch Neben\-be\-din\-gung\-en und Zielvorgaben.   
\end{itemize}
Hinsichtlich der Wissensrepr�sentation stellen Ferstl und Sinz imperative und deklarative Paradigmen gegen�ber \cite[S.366]{sinz2013}. Ein Programm, das dem imperativen Paradigma folgt, besteht aus einer Folge von Befehlen, die nacheinander ausgef�hrt werden \cite[S.341]{sinz2013}. Bei einem deklarativen Programm handelt es sich um eine Beschreibung der Aufgabenau�ensicht. Ein deklaratives Programm hat keine festgelegten L�sungsverfahren f�r jede Aufgabe. Stattdessen wird eine L�sung zum Zeitpunkt der Aufgabendurchf�hrung mittels Inferenzmaschine abgeleitet \cite[S.361]{sinz2013}.\\
Im Allgemeinen beziehen sich die Autoren darauf, dass an ein wissensbasiertes System nur geringe Anforderungen bez�glich Vollst�ndigkeit, Widerspruchsfreiheit und Eindeutigkeit gestellt werden k�nnen. Aus diesem Grund ist das deklarative Paradigma f�r die Wissensrepr�sentation besser geeignet. Folgende Gr�nde nennen die Autoren f�r die deklarative Umsetzung der Wissensbasis \cite[S.366]{sinz2013}: 
\begin{itemize}
\item \textit{Wissensdarstellung}: Der ein Mensch baut das Erfahrungswissen durch assoziative Beziehungsmuster auf.
\item \textit{Wissensauswertung}: �nderungen von Erfahrungswissen werden normalerweise in deklarativer Form erfasst.
\item \textit{Wissensverf�gbarkeit}: Die Codewartung eines imperativen Programms ist fehleranf�llig, sowie kosten- und zeitintensiv, da das Erfahrungswissen h�ufig ge�ndert und aktualisiert werden muss.
\end{itemize}
Der objektorientierte Ansatz ist eine weitere M�glichkeit, das Wissen zu beschreiben. Ein Beispiel f�r die objektorientierte Implementierung wird in \cite{leung1990} vorgestellt. Die Wissensbasis wird dabei als eine Sammlung von Klassen, Objekten und Methoden definiert \cite[S.40]{leung1990}. Der gro�e Vorteil einer solchen Umsetzung besteht in der Modularit�t des Wissens, d.h. das Wissen wird in unabh�ngige Module aufgeteilt. Da die einzelnen Module voneinander unabh�ngig sind, k�nnen sie getrennt getestet und modifiziert werden, ohne den Rest der Wissensbasis zu beeintr�chtigen. Dies erm�glicht eine hohe Flexibilit�t bei der Wissensbasiserweiterung \cite[S.43]{leung1990}.\\
Neben der Implementierung der Wissensbasis ist eine geeignete Umsetzung der Wissens\-erwerbskomponente erforderlich, um die Wissensbasis aktuell, m�glichst fehlerfrei und konsistent zu halten. Im Folgenden wird die Wissenserwerbskomponente in Hinsicht auf den allgemeinen Aufbau und ihre Funktionen thematisiert.  
\subsection{Wissenserwerbskomponente}\label{subsec:Wissenserwerbskomponente}
Bei der Wissenserwerbskomponente handelt es sich um einen Bestandteil eines Expertensystems, der den Wissensingenieur oder einen Experten beim Aufbau und sp�terer Erweiterung der Wissensbasis unterst�tzt \cite[S.18]{gottlob1990}. Allgemein umfasst die Wissenserwerbskomponente zwei grunds�tzliche Aufgaben, n�mlich den Wissenserwerb und die Pr�fung des Dateninputs auf Konsistenz, Vollst�ndigkeit und Einschr�nkungen des Expertensystems \cite[S.759]{wachsmuth1993}.\\
Unter dem Wissenserwerb wird eine �bertragung sowie Eingliederung von Wissen �ber Probleml�sungsverfahren in ein Computerprogramm verstanden \cite[S.178]{gottlob1990}. Es werden folgende Grundarten des Wissenserwerbs unterschieden \cite[S.742]{wachsmuth1993}:
\begin{itemize}
\item \textit{Indirekter Wissenserwerb}: Ein Wissensingenieur f�hrt ein Interview mit einem Experten, oder allgemein mit einem Wissenstr�ger durch. Die Analyse der Dokumente, die f�r das System relevant sind, geh�rt ebenso zur Aufgabe des Wissensingenieurs.
\item \textit{Direkter Wissenserwerb}: Ein Wissenstr�ger gibt sein Wissen selbst mittels einer Schnittstelle ins Expertensystem ein. 
\item \textit{Automatisierter Wissenserwerb:} Die Wissensbasis wird entweder mithilfe der automatisierten Datenerschlie�ung aus verf�gbaren Dokumente oder Methoden des maschinellen Lernens erweitert.
\end{itemize}
Die Methoden des indirekten Wissenserwerbs lassen sich grunds�tzlich in unstrukturierte und strukturierte Verfahren unterteilen. Das unstrukturierte Interview ist die am h�ufigsten verwendete Methode \cite[S.76]{gottlob1990}. Dabei stellt der Wissensingenieur dem Experten problembezogene Zwischenfragen, um ein m�glichst vollst�ndiges Bild des zur Probleml�sung erforderlichen Wissens zu bekommen. Die Hauptschwierigkeit bei der Wissenserhebung durch ein Interview ist die Formulierung der Fragen. Wenn die Fragen zu spezifisch sind, k�nnen wichtige Informationen ausgelassen werden \cite[S.95]{gebus2009}. Eine strukturierte Vorgehensweise der Wissenserhebung ist die Protokollanalyse, wobei der Experte beim L�sen eines Problems aufgezeichnet wird. Um den L�sungsweg nachvollziehbar zu machen, kann der Experte die Aufgabe gezielt langsamer durchf�hren oder die Aufzeichnung mit Kommentaren versehen. Die aufgabenbezogenen L�sungen werden mithilfe der Induktion generalisiert. Bei der Induktion wird eine allgemeine Regel aus den Einzelf�llen abgeleitet \cite[S.308]{castro2001}. Anschlie�end werden die erzielten Ergebnisse vom Wissensingenieur formalisiert und ins Expertensystem eingetragen.\\
Beim direkten Wissenserwerb soll eine geeignete Schnittstelle im Rahmen der Wissenserwerbskomponente zur Verf�gung gestellt werden, die es dem Wissenstr�ger erm�glicht, sein Wissen ins System einzugeben. Die Schnittstelle soll eine dem Wissenstr�ger bekannte Wissensrepr�sentation verwenden und benutzerfreundlich bei der Dateneingabe sein \cite[S.743]{wachsmuth1993}. Ein Beispiel der Benutzerfreundlichkeit ist die gleichzeitige Validierung der Benutzereingaben sowie eine R�ckmeldung bei unzul�ssigen Aktionen. Der direkte Wissenserwerb hat den Vorteil, dass die Wissensbasis ohne den Wissensingenieur erweitert werden kann. Allerdings betonen die Autoren in \cite[S.765]{wachsmuth1993}, dass das nur in gut verstandenen und strukturierten Anwendungsbereichen m�glich sei.\\
Zu dem automatisierten Wissenserwerb gibt es am wenigsten Erkenntnisse, die allgemein anwendbar sind. Meistens handelt es sich um L�sungen, die nur innerhalb eines spezifischen Anwendungsbereichs funktionieren. Nichtsdestotrotz l�sst sich sagen, dass dass sich das Wissen entweder aus vorhandenen Daten (maschinelles Lernen) oder aus verf�gbaren Dokumenten (automatisierte Dokumentenanalyse) generieren l�sst \cite[S.78]{gottlob1990}. Die Implementierung h�ngt jedoch vom Anwendungsgebiet ab. Ein Beispiel f�r den Fall gro�er Datenmengen und Anwendung des maschinellen Lernens ist ein Diagnosesystem, das f�r m�gliche Diagnosef�lle die zugeh�rigen Symptome enth�lt. F�r die Textanalyse k�nnen beispielsweise Bedienungsanleitungen analysiert werden, wobei diese Vorgehensweise gewisse Einschr�nkungen ausweist. Die Schwierigkeit besteht darin, dass das Erfahrungswissen nicht in den Textdokumenten zu finden ist \cite[S.79]{gottlob1990}.\\
In allen F�llen des Wissenserwerbs werden die Daten an die zentrale Schnittstelle weitergereicht. Diese Schnittstelle besch�ftigt sich mit der Zwischenspeicherung und der Pr�fung der Datens�tze auf Korrektheit, bevor die Daten endg�ltig in der Wissensbasis gespeichert werden. Zum Teil kann der Wertebereich direkt im einzelnen Modul des Wissenserwerbs eingeschr�nkt werden. Beispielsweise kann ein Eingabefeld in der Wissenstr�gerschnittstelle nur positive Zahlen zulassen. F�r den restlichen Teil werden automatisierte Tests durchgef�hrt, die sicherstellen, dass neue Daten keine Inkonsistenzen in Bezug auf die Einschr�nkungen der Wissensbasis erzeugen \cite[S.765]{wachsmuth1993}.\\
Zusammenfassend l�sst sich die Wissenserwerbskomponente schematisch in Abbildung \ref{fig:wissenserwerbskomponente} wie folgt darstellen:
\begin{figure}[H] 
	\centering
	\includegraphics[width=1.0\textwidth]{images/wissenserwerbskomponente.png}
	\caption{Wissenserwerbskomponente}
	\label{fig:wissenserwerbskomponente}
\end{figure}
In Abbildung \ref{fig:wissenserwerbskomponente} sieht man deutlich, dass die Wissenserwerbskomponente modular aufgebaut ist. Allgemein kann dieses Modell in jedem Expertensystem eingesetzt werden, wobei die konkrete Umsetzung in Bezug auf den Anwendungsbereich spezifiziert wird. Dabei l�sst sich der Automatisierungsgrad der Datenerfassung leicht anpassen, indem der Schwerpunkt auf gew�nschte Bestandteile der Wissenserwerbskomponente gelegt wird. Beispielsweise kann sich ein Expertensystem mit umfangreichen Falldaten auf das maschinelle Lernen konzentrieren. Ein System, dass auf der Datenerfassung mithilfe der Wissenstr�gerschnittstelle basiert ist, wird in \cite[S.97]{gebus2009} vorgestellt.\\
Im weiteren Verlauf der Arbeit werden die Wissenstr�gerschnittstelle, die Komponente mit Methoden des automatisierten Wissenserwerbs und die Schnittstelle f�r Validierung und Speicherung der Daten thematisiert, da sie ein Potenzial f�r die Automatisierung der Datenerfassung aufweisen.
\newpage
\section{Methoden der Wissens- und Datenerfassung}\label{sec:Wissenserfassung}
Mit der Wissensakquisitionskomponente wurde bereits angedeutet, dass die Erweiterung bzw. Aktualisierung der Wissensbasis eines Expertensystemes meistens nur teilweise automatisierbar ist. Die Autoren in \cite{tecuci1994} weisen ebenso darauf hin, dass manuelle und maschinelle Wissenserschlie�ung jeweils eigene St�rke haben, die sich gegenseitig erg�nzen  \cite[S.137]{tecuci1994}. Aus diesem Grund ist bei der Datenerfassung ein hybrides Modell sinnvoll, das die Vorteile manueller und maschineller Verfahren kombiniert. Aufgrund der Fragestellung dieser Arbeit wird es im Weiteren auf automatisierte und halb-automatisiere Bestandteile der Wissensakquisitionskomponente beschr�nkt.
\input{sections/section-3/schnittstelle-zur-dateneingabe.tex}
\subsection{Datenerfassung aus dem Web}\label{subsec:webdaten}
Der Prozess der Webdatenerfassung umfasst die Datenextraktion aus den unstrukturierten bwz. semi-strukturierten Webdokumenten (z.B. eine Webseite oder eine E-Mail) und die Transformation der erfassten Daten in eine strukturierte Form f�r sp�tere Verwendung \cite[S.301]{ferrara2014}. Im Folgenden werden die allgemeinen Probleme bei der Erfassung der Webdaten angesprochen. Darauffolgend werden generelle Paradigmen der Datenerfassung erl�utert, n�mlich Baumparadigma, Web Wrapper und hybrides System.\\
Bei der Erfassung der Webdaten gibt es mehrere Faktoren, die ber�cksichtigt werden sollen. Ferrara et al identifizieren folgende Herausforderungen \cite[S.302]{ferrara2014}:
\newpage
\begin{itemize}
\item \textit{Automatisierungsgrad}: Die Erfassung der Webdaten soll oft von einem menschlichen Experten �berwacht werden, um die Genauigkeit der Daten zu gew�hrleisten.
\item \textit{Skalierbarkeit}: Bei den umfangreichen Webressourcen soll innerhalb k�rzer Zeit schnell eine gro�e Datenmenge bearbeitet werden.
\item \textit{Datenschutz}: Wenn es um die Erfassung der personenbezogenen Daten geht (bei sozialen Netzwerken wie Facebook), soll die Privatsph�re des Individuums nicht beeintr�chtigt werden.
\item \textit{�nderung der Ressourcenstruktur}: Die Struktur der Webressourcen �ndert sich oft. Die Datenerfassungsmethoden f�r das Web sollen eine gewisse Flexibilit�t besitzen, um weiterhin korrekt zu funktionieren.
\item \textit{Trainingsdaten}: Bei der Einsetzung des maschinellen Lernen ist eine ausreichende Trainingsmenge an Webseiten erforderlich, die manuell vorbereitet wird. Dies ist eine schwierige und fehleranf�llige Aufgabe.
\end{itemize}
Das Baumparadigma nutzt die Baumstruktur einer Webseite aus, um die gew�nschten Daten zu erfassen. Dabei handelt es sich um Dokumente, die in \acf{HTML} beschrieben sind. Eine HTML-Seite wird als \acf{DOM}\footnote{https://www.w3.org/DOM} definiert. Die Idee vom DOM besteht darin, dass die HTML-Webseite ein Baum darstellt, der mittels HTML-Tags (z.B. Button-Tag) ausgezeichnet wird. Tags k�nnen weitere Tags beinhalten und bilden somit eine hierarchische Struktur. Diese hierarchische Baumstruktur erm�glicht effiziente Datensuche in einer HTML-Seite \cite[S.303]{ferrara2014}.\\
Da HTML ein Dialekt von \acf{XML} ist, kann \acf{XPath}\footnote{https://www.w3.org/TR/xpath} f�r die Navigation in DOM eingesetzt werden. In einem XPath-Ausdruck k�nnen beliebige Elemente einer HTML-Webseite ausgew�hlt werden. In Abbildung \ref{xpath} werden zwei Beispiele dargestellt. Im ersten Fall (A) wird genau ein Element (die erste Zelle in der ersten Reihe) ausgew�hlt. Im Beispiel (B) werden mehrere Elemente (alle Zellen der zweiten Reihe) angesprochen \cite[S.303]{ferrara2014}.
\begin{figure}[H] 
	\centering
	\includegraphics[width=1.0\textwidth]{images/xpath.png}
	\caption{XPath im Dokumentenbaum, \cite[S.304]{ferrara2014}}
	\label{xpath}
\end{figure} 
Der Hauptnachteil von XPath besteht darin, dass XPath-Ausdr�cke strikt an der DOM-Struktur gebunden sind. Wenn eine �nderung im DOM stattfindet, funktioniert der von der �nderung betroffene Ausdruck nicht mehr. Aus diesem Grund m�ssen die XPath-Ausdr�cke nach jeder Ver�nderung der HTML-Webseite manuell angepasst werden. In Bezug auf dieses Problem wurde im letzten Release von XPath\footnote{https://www.w3.org/TR/xpath20} relative XPath-Ausdr�cke eingef�hrt \cite[S.304]{ferrara2014}.\\
Ein weiterer Ansatz, die Webdaten zu erfassen, stellt ein Web Wrapper dar. Das Web Wrapper umfasst in der Regel einen oder mehreren Algorithmen, die zur Datenerfassung aus den Webdokumenten eingesetzt werden. Anschlie�end werden die erfassten Daten in eine strukturierte Form transformiert und f�r weitere Nutzung gespeichert. Ein Web Wrapper umfasst folgende Schritte \cite[S.305]{ferrara2014}:
\begin{itemize}
\item[1.]\textit{Generierung}: Definition des Wrappers.
\item[2.]\textit{Ausf�hrung}: Datenerfassung mithilfe des Wrappers.
\item[3.]\textit{Wartung}: Anpassung des Wrappers bei der �nderung der DOM-Struktur.
\end{itemize}
Nach Ferrara et al kann ein Web Wrapper mittels folgender Ans�tze generiert und ausgef�hrt werden \cite[S.306]{ferrara2014}:
\begin{itemize}
\item \textit{Regul�re Ausdr�cke}: Daten werden gem�� Regeln (expressions) gewonnen. Im Rahmen dieser Arbeit wird es im Weiteren aus regul�re Ausdr�cke beschr�nkt.
\item \textit{Logikbasierter Ansatz}: Zur Datenerfassung wird eine Wrapper Programmiersprache eingesetzt (wrapper programming language).
\item \textit{Baumbasierter Ansatz}: Dabei wird die Annahme getroffen, dass bestimmte Bereiche im DOM generell f�r Daten zust�ndig sind. Die Identifikation und Datenextraktion aus diesen Bereichen ist der Gegenstand des baumbasierten Ansatzes.  
\item \textit{Maschinelles Lernen}: Daten werden mithilfe eines Lernalgorithmus und einer Trainingsmenge erfasst.
\end{itemize} 
Regul�re Ausdr�cke erm�glichen die Erkennung von Patterns in der unstrukturierten bzw. semi-strukturierten Dokumenten unter Verwendung der Regeln, die z.B. in Form von Wortgrenzen oder HTML-Tags definiert werden. Der Vorteil der regul�ren Ausdr�cken besteht in der M�glichkeit, beliebiges Element auf einer Webseite anzusprechen. Au�erdem bieten einige Implementierungen ein grafisches Benutzerinterface, sodass der Benutzer die Elemente auf einfache Weise ausw�hlen kann. Die Regeln werden dann automatisch generiert. Eine m�gliche Umsetzung des Wrappers wird in \cite{sahuguet1999} mit W4F vorgestellt. Das Tool verf�gt �ber eine Hilfsmethode, die den Benutzer bei der Auswahl der Elementen unterst�tzt. Auf Basis der ausgew�hlten Elementen werden die Regeln erstellt. Allerdings sind die Regeln in Bezug auf DOM-�nderungen nicht flexibel und k�nnen sehr schnell verletzt werden \cite[S.306]{ferrara2014}.\\
In der dritten Phase geht es um die Anpassung des Web Wrappers im Hinblick auf die Ver�nderungen der DOM-Struktur. Die Anpassung erfolgt entweder manuell oder in gewisser Weise automatisiert. Ferrara et al betonen, dass der Automatisierungsgrad der Wartung besonders kritisch ist \cite[S.308]{ferrara2014}. Bei einer kleinen Dokumentenanzahl ist die manuelle Anpassung noch akzeptabel. Allerdings ist die manuelle Wartung bei einer gro�en Menge der Dokumente nicht mehr denkbar.\\
Ein Beispiel der automatisierter Wrapper-Wartung wird in \cite{meng2003} mit dem System namens SG-WRAM (Schema-Guided Wrapper Maintenance for Web-Data Extraction) vorgestellt. Die Architektur vom SG-WRAM wird in der Abbildung \ref{wram} dargestellt.
\begin{figure}[H] 
	\centering
	\includegraphics[width=0.95\textwidth]{images/wram.png}
	\caption{Die Architektur vom SG-WRAM, \cite[S.2]{meng2003}}
	\label{wram}
\end{figure} 
Im ersten Schritt werden die HTML-Seite, XML-Schemata und das Mapping zwischen diesen beim Wrapper-Generator eingegeben. XML-Schema wird in \acf{DTD} beschrieben. Darauffolgend generiert das System die Regeln (Wrapper) f�r die gegebene Webseite, um die Daten zu erfassen und in einer XML-Datei gem�� der DTD-Datei zu speichern. Neben der Datenextraktion werden zus�tzlich die Probleme bei der Extraktion erfasst, um ein Wiederherstellungsprotokoll f�r die fehlgeschlagenen Regeln zu erzeugen. Wenn das Protokoll die Fehler beseitigt, wird die Datenerfassung fortgesetzt. Beim Fehlschlag werden Warnungen und Benachrichtigungen angezeigt, dass die Regeln nicht mehr funktionieren. In diesem Fall sollen die Regeln manuell von einem Experten angepasst werden \cite[S.308]{ferrara2014}.\\
Als ein Ausblick wird ein hybrides System der Webdatenerfassung angesprochen. Ein Beispiel des hybriden Ansatzes stellt RoadRunner von \cite{crescenzi2001} dar. RoadRunner kann sowohl mit der Trainingsmenge von einem Nutzer als auch mit selbst erstellten Trainingsdaten ausgef�hrt werden. Das Verfahren arbeitet gleichzeitig mit zwei HTML-Seiten und analysiert die Gemeinsamkeiten und die Unterschiede zwischen diesen Seiten, um die Patterns zu finden. Allgemein kann das Verfahren die Daten aus beliebiger Quelle erfassen, die mindestens zwei Seiten mit �hnlicher Struktur sind. Da Webseiten normalerweise dynamisch auf Basis eines Templates generiert werden, befinden sich relevante Daten in gleichen oder �hnlichen Bereichen \cite[S.309]{ferrara2014}.
\newpage
\subsection{Maschinelles Lernen beim Wissenserwerb}
Eine Anwendung des maschinellen Lernens bei der Automatisierung des Wissenserwerbs wird in \cite{castro2001} vorgestellt. Die grundlegende Idee dabei orientiert sich an \cite{tecuci1992}, indem die Wissensakquisition als ein Prozess der der Erweiterung, Aktualisierung und Verbesserung der unvollst�ndigen Wissensbasis betrachtet wird. Der vorliegende Algorithmus verwendet die Induktion und den Algorithmus mit Fuzzylogik, um neues Wissen aus einer Trainingsmenge zu erschlie�en. Wie bereits im Kapitel \ref{subsec:Wissensakquisitionskomponente} erw�hnt wurde, werden bei der Induktion allgemeine Schlussfolgerungen auf Basis von Einzelf�llen erzielt. Bei der Fuzzylogik handelt es sich um graduelle Aussagen, die mithilfe der vagen Pr�dikaten beschrieben werden \cite[S.27]{beierle2014}. Castro et al betonen, dass die Induktion und die Fuzzylogik effektiv mit dem unvollst�ndigen Wissen arbeiten, das bei der Erfassung des Expertenwissens oft der Fall ist \cite[S.308]{castro2001}.\\
Weitere Punkte:
\begin{itemize}
\item Wissensrepr�sentation  
\item Algorithmus
\item Beispiel
\end{itemize}
\newpage
\section{Umsetzung der Wissenstr�gerschnittstelle}\label{sec:Umsetzung}
Nachdem die Grundlagen der wissensbasierten Systeme und Wissenserfassung betrachtet wurden, handelt es sich in diesem Kapitel um die Umsetzung der Wissenstr�gerschnittstelle am Beispiel von \textit{PaaSfinder}. Am Anfang wird ein kurzer Bezug auf das \textit{PaaSfinder} genommen. Danach wird das Konzept und die Umsetzung der Wissenstr�gerschnittstelle zum Aktualisieren der bestehenden Daten erl�utert. Anschlie�end wird die Schnittstelle zur Daten�bermittlung betrachtet. 
\subsection{Systembeschreibung}
Die Umsetzung der Wissenserwerbskomponente erfolgt am Beispiel von \textit{PaaSfinder}\footnote{{\label{foot:paasfinder}}https://paasfinder.org}, einem Expertensystem und Open-Source-Projekt im Bereich von \acf{PaaS}. Das Ziel des Systems besteht darin, zahlreiche PaaS-Anbieter (im Weiteren Vendors) vergleichbar zu machen. \textit{PaaSfinder} verf�gt bereits �ber eine umfangreiche Wissensdatenbank, die aufgrund der h�ufigen �nderungen im gegebenen Anwendungsbereich regelm��ig aktualisiert werden muss. Die Struktur von \textit{PaaSfinder} l�sst sich wie in Abbildung \ref{fig:paasfinder} beschreiben.
\begin{figure}[H] 
	\centering
	\includegraphics[width=0.9\textwidth]{images/paasfinder.png}
	\caption{Die Struktur von \textit{PaaSfinder}}
	\label{fig:paasfinder}
\end{figure} 
\textit{PaaSfinder} setzt sich aus dem Programmcode der Wissensdatenbank und der Webanwendung zusammen. Die Webanwendung wird wiederum auf einem Server betrieben. Ein Vendor wird in einem Profil\footnote{https://github.com/stefan-kolb/paas-profiles\#profile-specification} beschrieben, das wie folgt aufgebaut ist:
\begin{enumerate}
\item \textbf{General Properties}: allgemeine Eigenschaften des Vendors (z.B.\ Name, \acs{URL} oder \acs{URI}, Status, Typ etc.)
\item \textbf{Extensible}: generelle Erweiterungsm�glichkeit nach Kundenbedarf (z.B.\ spezielle Laufzeitumgebung)
\item \textbf{Pricing}: verf�gbare Preismodelle
\item \textbf{Quality of Service}: die Servicequalit�t (z.B.\ Verf�gbarkeit)
\item \textbf{Hosting}: Art der Bereitstellung (z.B.\ privat)
\item \textbf{Scaling}: Skalierbarkeit (z.B.\ Speichererweiterung)
\item \textbf{Runtimes}: unterst�tzte Laufzeitumgebungen (z.B.\ Java\footnote{https://java.com})
\item \textbf{Middleware} (z.B.\ Tomcat\footnote{https://tomcat.apache.org})
\item \textbf{Frameworks} (z.B.\ Play\footnote{https://www.playframework.com})
\item \textbf{Services} (z.B.\ Datenspeicher)
\item \textbf{Infrastructures} (z.B.\ Informationen zum Standort)
\end{enumerate}
Ein Vendor wird als \acs{JSON}\footnote{\label{foot:json}http://json.org}-Eintrag gespeichert. JSON ist ein Datenaustauschformat und basiert auf zwei Datenstrukturen: 
\begin{itemize}
\item \textit{Name/Wert Paar}, das ein Objekt darstellt (siehe Abbildung \ref{fig:json-object}).
\item \textit{Eine geordnete Liste von Werten (Array)}, die kein oder mehrere durch Komma getrennte Objekte enth�lt (siehe Abbildung \ref{fig:json-array}).
\end{itemize}
\begin{figure}[H] 
	\centering
	\includegraphics[width=0.85\textwidth]{images/json-object.png}
	\caption{JSON-Objekt Spezifikation}
	\label{fig:json-object}
\end{figure} 
\begin{figure}[H] 
	\centering
	\includegraphics[width=0.75\textwidth]{images/json-array.png}
	\caption{JSON-Array Spezifikation}
	\label{fig:json-array}
\end{figure} 
Die Wissensdatenbank und der Programmcode von \textit{PaaSfinder} befinden sich in einem Git-Repository. Bei Git\footnote{https://git-scm.com} handelt es sich um eine Anwendung zur verteilten Projektverwaltung. Da \textit{PaaSfinder} ein Open-Source-Projekt ist, kann im Prinzip jeder die Wissensdatenbank erweitern bzw. modifizieren. Momentan kann das auf zwei Weisen erfolgen (siehe Abbildung \ref{fig:paasfinder}):
\begin{itemize}
\item \textit{Pull Request\footnote{https://git-scm.com/docs/git-request-pull}}
\item \textit{Direkter Kontakt mit dem Systementwickler} 
\end{itemize}
Beim Pull Request wird zuerst das gesamte Git-Repository lokal gespeichert. Danach werden �nderungen vorgenommen und anschlie�end eine Anfrage beim Projektmaster gestellt. Diese wird vom Projektmaster �berpr�ft und zugeh�rige �nderungen �bernommen (\glqq{}gemerged\grqq{}), falls keine Konflikte mit vorhandenen Daten vorliegen. Wenn der Wissenstr�ger mit dem Git-System nicht vertraut ist, kann direkter Kontakt (beispielsweise per E-Mail) mit dem Systementwickler aufgenommen werden. Die Daten werden dann manuell vom Systemverwalter in die Wissensbasis eingetragen.\\
Sowohl Pull Requests, als auch der Kontakt mit dem Systementwickler entsprechen dem direkten Wissenserwerb gem�� der Klassifikation in Kapitel \ref{subsec:Wissenserwerbskomponente}. Der Vorteil dabei ist, dass die Daten nicht manuell gesucht werden m�ssen, sondern direkt vom Wissenstr�ger stammen. Andererseits wird die Datenerfassung durch fehlende Git-Kenntnisse des Wissenstr�gers beschr�nkt. Um den direkten Wissenserwerb zu erleichtern, wird im weiteren Verlauf die Wissenstr�gerschnittstelle entwickelt, die es dem Wissenstr�ger erm�glicht, einen bestehenden Vendor zu aktualisieren. Nach dem erfolgreichen Absenden der �nderungen soll automatisch ein Pull Request erstellt werden. 
\subsection{Wissenstr�gerschnittstelle}\label{subsec:Wissenstr�gerschnittstelle}
Als erster Schnitt in der Automatisierung der Datenerfassung von \textit{PaaSfinder} wird eine Web-Schnittstelle zur Dateneingabe entwickelt (Wissenstr�gerschnittstelle). Mithilfe dieser soll die Aktualisierung eines bestehenden Vendors erm�glicht werden. Der Ablauf wird in Abbildung \ref{fig:update-allgemein} veranschaulicht. Das Anlegen oder L�schen eines Vendors liegt nicht im Anwendungsbereich der Schnittstelle.
\begin{figure}[H] 
	\centering
	\includegraphics[width=1.0\textwidth]{images/update.png}
	\caption{Profilaktualisierung mittels Wissenstr�gerschnittstelle}
	\label{fig:update-allgemein}
\end{figure}
Der Ablauf in Abbildung \ref{fig:update-allgemein} l�sst sich in zwei Schritte aufteilen. Dieser Abschnitt besch�ftigt sich mit dem ersten Schritt, in welchem die Daten von einem Wissenstr�ger entgegengenommen, zusammenfasst und abschickt werden. Die Verarbeitung dieser Daten erfolgt im zweiten Schritt, der im n�chsten Abschnitt besprochen wird. Allgemein besteht der erste Schritt aus folgenden Aktionen:
\begin{enumerate}
\item \textit{Vendorauswahl} (Vendor Page)
\item \textit{Aktualisieren} des Profils (Update Page)
\item \textit{�berpr�fung} der Daten (Review Page)
\item \textit{Absenden} der Daten (Submit)
\end{enumerate}
Der gesamte Prozess l�sst sich wie in Abbildung \ref{fig:activity} als Aktivit�tsdiagramm darstellen. Die verwendete Notation entspricht dem UML-Standard 2.5 \footnote{http://www.omg.org/spec/UML/2.5/PDF}. Es werden Knoten (Rechtecke) f�r Aktivit�ten und Kanten (Pfeile) f�r Verbindungen verwendet. Bei den Verbindungen werden Kontrollfluss- und Datenflussverbindungen unterschieden. Im Fall einer Datenflussverbindung gibt es einen Ein- und einen Ausgabe-Pin (kleiner Quadrat). Die Rauten kennzeichnen Fallunterscheidungen.
\begin{figure}[H] 
	\centering
	\includegraphics[width=1.0\textwidth]{images/activity.png}
	\caption{Aktivit�tsdiagramm der Aktualisierung eines Vendors}
	\label{fig:activity}
\end{figure}
Ausgehend von einer Vendor Page wird der Ablauf durch den Link \glqq{}Add an Update\grqq{} gestartet. Als N�chstes werden die Daten eingegeben. Wenn die Vendorinformationen aus Sicht des Wissenstr�gers vollst�ndig sind, klickt dieser auf \glqq{}Review\grqq{}. Sonst wird der Schritt der Dateneingabe wiederholt. Sobald der \glqq{}Review\grqq{}-Button geklickt wurde, gelangt der Nutzer zur Review Page. Hier k�nnen die Daten �berpr�ft werden. Wenn die Daten korrekt sind, k�nnen diese mit dem Klick auf \glqq{}Submit\grqq{} an den Worker gesendet werden. Andernfalls kann der Wissenstr�ger zur�ck zur Dateneingabe gehen und die Daten �berarbeiten. Zu jedem Zeitpunkt besteht die M�glichkeit, den Vorgang abzubrechen, indem das Browserfenster geschlossen wird. Die Daten werden nur dann gesendet, wenn der \glqq{}Submit\grqq{}-Button auf der Review Page geklickt wurde. Im weiteren Verlauf wird die Implementierung der Update Page und Review Page beschrieben.\\
Die Update Page umfasst zwei Aspekte. Erstens sollen die Vendordaten f�r den Wissenstr�ger geeignet dargestellt werden. Zweitens soll die Eingabe neuer und die �nderung bestehender Daten erm�glicht werden. Au�erdem sollen die �nderungen am Profil dynamisch zwischengespeichert werden. Das ist eine herausfordernde Aufgabe, da ein Vendor, abgesehen von allgemeinen Eigenschaften (z.B. \glqq{}name\grqq{}), aus komplexen und teils verschachtelten Datentypen besteht. In Abbildung \ref{fig:profil} wird dies am Beispiel von \glqq{}runtimes\grqq{} (Laufzeitumgehungen) gezeigt. Gelb steht dabei f�r ein Objekt, grau f�r eine Liste und wei� f�r einen String (Zeichenkette). Hier umfasst ein Vendor eine Liste von \glqq{}runtimes\grqq{}. Eine \glqq{}runtime\grqq{} besteht wiederum aus \glqq{}language\grqq{} und \glqq{}versions\grqq{}.
\begin{figure}[H] 
	\centering
	\includegraphics[width=0.90\textwidth]{images/vendor.png}
	\caption{Ausschnitt aus Vendor Datentyp }
	\label{fig:profil}
\end{figure}
Um diese Aufgabe zu l�sen, werden die Daten in zwei Richtungen gebunden (two-way data binding). Zu diesem Zweck wird das Open-Source Framework Knockout.js\footnote{http://knockoutjs.com} eingesetzt, eine JavaScript Bibliothek zur Erstellung dynamischer Inhalte. Das Framework basiert auf dem \ac{MVVM} Entwurfsmuster\footnote{http://knockoutjs.com/documentation/observables.html}:
\begin{itemize}
\item \textit{Model}: Daten, die unabh�ngig auf einem Server liegen (hier: verf�gbare Vendors).
\item \textit{ViewModel}: Programmcode der Daten und Operationen auf der Benutzerschnittstelle anzeigt (hier: JavaScript Klasse f�r Vendor (Modell)).
\item \textit{View}: Benutzerschnittstelle, die den Zustand des ViewModels abbildet und bei Benutzeraktionen aktualisiert (hier: Update Page).
\end{itemize}
Als Erstes werden die f�r das Modell erforderlichen Daten �ber die \textit{PaaSfinder}-\acs{API}\footnote{https://paasfinder.org/api/vendors/} geholt und das Modell initialisiert. Da das Modell komplexe Datentypen umfasst, sind explizite JavaScript-Klassen n�tig. Beispielsweise gibt es f�r \glqq{}runtime\grqq{} und \glqq{}version\grqq{} eigene Klassen mit jeweils unterschiedlichem Verhalten. Die Klasse \glqq{}runtime\grqq{} beinhaltet z.B. die Methode zum Hinzuf�gen einer neuen Version. Nachdem das Modell initialisiert wurde, wird das Data-Binding zwischen dem Modell und der View aktiviert.\\
W�hrend die Update Page die Benutzerdaten entgegennimmt, werden die Daten auf der Review Page zusammengefasst, sodass der Wissenstr�ger die Eingaben �berpr�fen kann. Die Profildaten werden mithilfe von Web Stogare\footnote{https://www.w3.org/TR/webstorage/} zwischen der Update Page und der Review Page ausgetauscht. Der W3C-Standard definiert zwei Arten der Speicherung, n�mlich Session und Local Storage. Session Storage wird ausschlie�lich innerhalb des Browserfensters verwendet. Auf die Inhalte des Local Storage kann dagegen innerhalb einer Domain zeitlich unbeschr�nkt zugegriffen werden. Aus diesem Grund wird im Weiteren das Local Storage betrachtet.\\
Auf der Update Page wird beim Klicken auf \glqq{}Review\grqq{} ein Vendor in das Local Storage gespeichert. Das Speichern bzw. Lesen basiert wie bei JSON auf dem Schl�ssel/Wert-Paar-Prinzip. Der Schl�ssel ist dabei der Name des Vendors, der bei der API benutzt wird. Da der Schl�ssel als Parameter an die Review Page geschickt wird, k�nnen die Daten problemlos gelesen werden. Es besteht ebenso die M�glichkeit, den Vendor auf der Update Page zur�ckzusetzen, indem die Vendordaten aus dem Local Storage gel�scht und erneut von der \textit{PaaSfinder}-API angefordert werden.\\
Im folgenden Beispiel wird der Vendor \glqq{}Heroku\grqq{} aktualisiert, indem eine neue Java Version (1.9) in \glqq{}runtimes\grqq{} hinzugef�gt wird.
\begin{enumerate}
\item Auf der Vendor Page auf \glqq{}Add an Update\grqq{} klicken.
	\begin{figure}[H] 
		\centering
		\includegraphics[width=1.0\textwidth]{images/ablauf/vendor-page.png}
		\caption{Vendor Page}
		\label{fig:vendor-page}
	\end{figure}
\item Eine neue Java Version (1.9) hinzuf�gen (siehe Abbildung \ref{fig:update-page}).
	\begin{figure}[H] 
		\centering
		\includegraphics[width=1.0\textwidth]{images/ablauf/update-page.png}
		\caption{Update Page}
		\label{fig:update-page}
	\end{figure} 
\item Die Daten auf der Review Page �berpr�fen (siehe Abbildung \ref{fig:vendor-page}).
	\begin{figure}[H] 
		\centering
		\includegraphics[width=1.0\textwidth]{images/ablauf/review-page.png}
		\caption{Review Page}
		\label{fig:review-page}
	\end{figure} 
\item Optional kann der Wissenstr�ger seine Kontaktdaten (Name und E-Mail-Adresse) sowie eine Nachricht zum Update hinterlassen.
	\begin{figure}[H] 
		\centering
		\includegraphics[width=1.0\textwidth]{images/ablauf/kontaktform.png}
		\caption{Kontaktformular}
		\label{fig:kontaktdatenform}
	\end{figure} 
\item Anschlie�end werden die Daten mit dem Klick auf den \glqq{}Submit\grqq{}-Button an die f�r die Daten�bermittlung zust�ndige Schnittstelle geschickt.
\end{enumerate}
Ein wichtiger Aspekt bei der Entwicklung der Wissenstr�gerschnittstelle ist die Va\-li\-die\-rung der Eingabedaten. In einigen F�llen k�nnen problematische Eingaben bereits auf der Ebene der Benutzerschnittstelle (Update Page) verhindert werden. Als Beispiel wird im Kontaktformular (Abbildung \ref{fig:kontaktdatenform}) bei der E-Mail Adresse mittels des HTLM-Attributs type=\glqq{}email\grqq{} sichergestellt, dass der Input der formalen Input-Struktur einer E-Mail Adresse entspricht. Allerdings kann dieser Ansatz nicht immer angewendet werden. So kann im Feld f�r eine Version kein type=\glqq{}number\grqq{} Attribut verwendet werden, da eine Version ein Sonderzeichen (z.B. *) oder einen Buchstaben enthalten kann.
\subsection{Worker f�r Daten�bermittlung}\label{subsec:Worker}
Der vorliegende Abschnitt befasst sich mit dem Teil der Wissenserwerbskomponente, der als Bindeglied zwischen den Wissenserfassungsmethoden und der Wissensbasis auftritt und in Abbildung \ref{fig:wissenserwerbskomponente} als Schnittstelle f�r die Daten�bermittlung bezeichnet wird. Im Rahmen dieser Arbeit wird der Begriff \glqq{}\textit{Worker}\grqq{} verwendet, der f�r die Auspr�gung dieser Schnittstelle steht. Im Folgenden wird das Konzept hinter dem Worker abstrakt skizziert und im weiteren Verlauf mit der konkreten Implementierung verdeutlicht. Der Anwendungsbereich vom Worker wird in Abbildung \ref{fig:worker} wie folgt dargestellt:
\begin{figure}[H] 
	\centering
	\includegraphics[width=1.0\textwidth]{images/anwendungsbereich_worker.png}
	\caption{Anwendungsbereich vom Worker}
	\label{fig:worker}
\end{figure}
Grunds�tzlich besteht der Worker aus zwei Komponenten, n�mlich \glqq{}\textit{UpdateService}\grqq{} und \glqq{}\textit{UpdateClient}\grqq{}. Die Aufgabe vom UpdateService besteht in der Bereitstellung einer Schnittstelle, die die Daten von au�en aufnimmt und die Daten�bermittlung an den UpdateClient delegiert. Auf der anderen Seite stellt der UpdateClient die Methoden zur Verf�gung, die f�r die Daten�bermittlung zust�ndig sind.\\
Generell l�sst sich der Ablauf gem�� der Abbildung \ref{fig:worker} folgenderma�en beschreiben. Als erstes werden die Daten von der Wissenserfassungsmethode an den UpdateService gesendet. Darauffolgend werden die Daten vom UpdateService verarbeitet und f�r den UpdateClient vorbereitet. Im n�chsten Schritt wird die Aufgabe der Daten�bermittlung an den UpdateClient delegiert. Der UpdateClient erstellt die Anfrage an die Wissensbasis und teilt die Antwort dem UpdateService mit. Anschlie�end verschickt der UpdateService die R�ckmeldung an die Wissenserfassungsmethode.\\ 
Im Weiteren wird die Wissenstr�gerschnittstelle aus dem Abschnitt \ref{subsec:Wissenstr�gerschnittstelle} als Wissenserfassungsmethode betrachtet. Die Wissensbasis stellt das Git-Repository von \textit{PaaSfinder} dar, das von einem Bot-Account auf Github \glqq{}geforkt\grqq{} wird\footnote{https://github.com/update-bot/paas-profiles}. In anderen Worten wird das urspr�ngliche Git-Repository von \textit{PaaSfinder} kopiert, sodass der Bot-Account einen schreibenden Zugriff auf die Wissensbasis hat.\\
Technisch gesehen erfolgt der Nachrichtenaustausch zwischen Wissenserfassungsmethode, dem Worker und der Wissensbasis auf Basis von \ac{HTTP}\footnote{https://www.w3.org/Protocols} und \ac{REST} Prinzip, das urspr�nglich aus der Dissertation von Fielding \cite{fielding2000} stammt. Das zentrale Konzept von REST basiert auf Ressourcen, die im globalen Raum mithilfe von \ac{URI}\footnote{https://tools.ietf.org/html/rfc3986} eindeutig identifiziert werden \cite[S.11,35]{tilkov2015}. Ein Vendor wird also als Ressource in JSON Format zun�chst zum Worker und darauffolgend zum Git-Repository geschickt.\\
Um die Daten von au�en empfangen zu k�nnen, implementiert der UpdateClient eine REST API, die in Form einer Route (\glqq{}/vendor\grqq{}) definiert wird. Die Route entspricht dem HTTP-Standardverb POST\footnote{https://tools.ietf.org/html/rfc7231\#section-4.3.3} und aktepriert die Daten in JSON Format. F�r die Implementierung der Route wurde das Framework Spark\footnote{http://sparkjava.com} verwendet. Auf der Seite vom UpdateClient werden die Nachrichten als Anfragen an die Github API\footnote{https://developer.github.com/v3} mithilfe von OkHttp\footnote{https://square.github.io/okhttp}  gesendet. In Java Pseudocode l�sst sich die Route folgenderma�en beschreiben (siehe Listing \ref{service}):
\begin{lstlisting}[basicstyle=\ttfamily, label=service,
					captionpos=b, caption={\glqq{}/vendor\grqq{} Route}]
post("/vendor", "application/json", (request, response) -> {
  JsonObject data = jsonParser.parse(request.body());
  
  Branch branch = new Branch(..);       
  client.postBranch(branch);
  
  File file = new File(..);
  client.putFile(file);

  PullRequest pullRequest = new PullRequest(..);
  client.postPullRequest(pullRequest);
});
\end{lstlisting}
Als erstes werden die Daten geparsed. Danach wird ein neuer Branch erzeugt und an den UpdateClient zum Absenden weitergegeben. Sobald der Branch auf der Github-Seite erfolgreich erstellt wurde, wird die Vendor-Datei im erstellen Branch aktualisiert. Anschlie�end wird ein Pull-Request erzeugt und vom UpdateClient an das Git-Repository geschickt. Bedauerlicherweise l�sst sich der Ablauf nicht parallelisieren, da der n�chste Schnitt die erfolgreiche Ausf�hrung des vorherigen Schrittes voraussetzt. Beispielsweise setzt die Aktualisierung der Datei die Erstellung vom Branch voraus, da die Datei im erstellten Branch aktualisiert wird.\\ 
Wenn man den oben beschriebenen Ablauf auf das Beispiel mit \glqq{}Heroku\grqq{} aus dem Abschnitt \ref{subsec:Wissenstr�gerschnittstelle} �bertr�gt, ergibt sich Folgendes. Sobald der Benutzer auf \glqq{}Submit\grqq{} klickt, werden die Daten als JSON in der POST-Anfrage an den Worker gesendet, n�mlich an die Schnittstelle vom UpdateService. Das erfolgreiche Branch bzw. Pull-Request-Erstellen wird durch den Code 201 (\glqq{}Created\grqq{}) im Response der Github-API mitgeteilt (siehe Listing \ref{branch} und \ref{pullrequest}). Beim erfolgreichen Updaten der Datei wird der Statuscode 200 (\glqq{}OK\grqq{}) zur�ckgeliefert (siehe Listing \ref{file}).
\begin{lstlisting}[basicstyle=\ttfamily, breaklines=true, label=branch,
					captionpos=b, caption={Response beim erfolgreichen Branch-Erstellen}]
{
  protocol=http/1.1,
  code=201, 
  message=Created,
  url=https://api.github.com/repos/update-bot/paas-profiles/git/refs
}
\end{lstlisting}

\begin{lstlisting}[basicstyle=\ttfamily, breaklines=true, label=file,
					captionpos=b, caption={Response beim erfolgreichen File-Update}]
{
  protocol=http/1.1,
  code=200, 
  message=OK, 
  url=https://api.github.com/repos/update-bot/paas-profiles/contents/profiles/heroku.json
}
\end{lstlisting}

\begin{lstlisting}[basicstyle=\ttfamily, breaklines=true, label=pullrequest,
					captionpos=b, caption={Response beim erfolgreichen Pull-Request-Erstellen}]
{
  protocol=http/1.1,
  code=201,
  message=Created,
  url=https://api.github.com/repos/update-bot/paas-profiles/pulls
}
\end{lstlisting}
Nach der erfolgreichen Pull-Request-Erstellung kann der Update von Heroku auf Github betrachtet werden (siehe Abbildung \ref{fig:pull-requests}). In der Detailansicht werden ebenso die �nderungen explizit gezeigt. Dabei werden die gel�schten Zeilen als rot markiert und die hinzugef�gten als gr�n (siehe Abbildung \ref{fig:pull-request-detail}).
\begin{figure}[H] 
	\centering
	\includegraphics[width=0.95\textwidth]{images/pull-requests.png}
	\caption{Pull-Requests Ansicht auf Github}
	\label{fig:pull-requests}
\end{figure}
\begin{figure}[H] 
	\centering
	\includegraphics[width=0.95\textwidth]{images/pull-request-detail.png}
	\caption{Heroku Pull Request}
	\label{fig:pull-request-detail}
\end{figure}
Schlie�lich wird im Erfolgsfall der Statuscode 200 (\glqq{}OK\grqq{}) der Wissenserfassungsmethode mitgeteilt. In Bezug auf die Fehlerbehandlung wird jeder Schritt, der sich logisch abgrenzen l�sst (Branch erstellen, Datei aktualisieren und Pull-Request erstellen), in einem eigenen try-catch-Block bei der Route in \ref{service} ausgef�hrt, sodass der Client eine aussagekr�ftige Fehlermeldung bekommt.
\newpage
\section{Ausblick}\label{sec:Ausblick}
Nachdem die Wissenstr�gerschnittstelle zur Datenkorrektur und der Worker zum Erstellen von automatischen Pull Requests betrachtet wurden, handelt es sich im vorliegenden Abschnitt um einen Ausblick in weitere Datenquellen, die einen Automatisierungspotential im Rahmen der Datenerfassung f�r \textit{PaaSfinder} aufweisen. Nach der kurzen Erl�uterung der m�glichen Datenquellen findet ein Vergleich der Quellen statt. Anschlie�end wird ein Bezug auf zuk�nftige Arbeit genommen. 
\subsection{Abgrenzung der Informationsquellen}\label{subsec:Informationsquellen}
Allgemein beschr�nkt sich die Datenerfassung wie im Kapitel \ref{sec:Umsetzung} auf die Aktualisierung der bestehenden Eintr�gen der Wissensbasis von \textit{PaaSfinder}. Dabei lassen sich folgende Informationsquellen identifizieren: 
\begin{itemize}
\item Webseiten von PaaS-Anbietern
\item Web-Feeds (z.B. News-Feeds)
\item Soziale Netzwerke 
\end{itemize}
Die Analyse der Webseiten w�re die naheliegende Vorgehensweise, neue Daten �ber einen PaaS-Anbieter zu erschlie�en. Die Wissensbasis von \textit{PaaSfinder} beinhaltet bereits die Homepage-URLs, die als Ausgangspunkt (Seed) bei einem Web-Crawler benutzt werden k�nnen. Auch die Fachliteratur bietet zahlreiche Informationen bez�glich der Umsetzung eines Web-Crawlers (z.B. \cite[S.35-50]{croft2010}). Allerdings macht das Webseiten-Crawling aus folgenden Gr�nden wenig Sinn:
\begin{itemize}
\item \textit{Webseitenstruktur}. Unter den PaaS-Anbietern gibt es keine einheitliche Sicht, wie die Informationen auf der Webseiten strukturiert werden k�nnen.
\item \textit{Datenverarbeitung (Parsen)}. Bei der kleinsten �nderung der Seitenstruktur kann die gesamte Anfrage (meist HTML-Tag-Selektoren) zusammengebrochen werden.
\item \textit{Neuheit der Information}. Im Kontext von bestehenden Daten ist es h�chstwahrscheinlich, dass die Daten bereits erfasst wurden. 
\end{itemize}
In Bezug auf Parsen der Webseiten wurde im Abschnitt \ref{subsec:webdaten} das Konzept von Wrapper angesprochen. Allerdings ist an der Stelle die Verwendung der REST-API einfacher zu nutzen und zu warten. Das Argument mit der Neuheit der Information kann umstritten werden, wenn es um die Erfassung eines noch nicht in der Wissensbasis vorhandenen PaaS-Anbieters geht. Aber selbst in diesem Fall stellt sich die Frage der Glaubw�rdigkeit eines PaaS-Profils, das auf maschinelle Weise erstellt wurde.\\
Eine weitere Kategorie der Informationsquelle stellen Web-Feeds und soziale Netzwerke dar. Aus Sicht der Datenaktualit�t sind Web-Feeds und soziale Netzwerke hervorragend geeignet. Die Daten werden in kleinen zeitlich geordneten Informationseinheiten dargestellt. Au�erdem bieten einige PaaS-Anbieter einen Service, der sich auf ein bestimmtes Thema spezialisiert. Aufgrund dessen wird im Weiteren der Bereich von Web-Feeds und sozialen Netzwerken genauer betrachtet.
\subsection{Vergleich der Informationsquellen}\label{subsec:vergleich}
Im Zuge des Vergleich wurden 30 Vendors (n=30) nach Vorhandensein der Informationsquellen ausgewertet. Als Informationsquellen wurden RSS-Feeds, Twitter und Newsletter ausgew�hlt. Ein weiteres Kriterium in der Auswertung ist das Datum des letzten Updates. Das Datum wird nach dem neuesten Eintrag zwischen den vorliegenden Informationsquellen zum Zeitpunkt 10.05.2017 bestimmt. Die Zusammenfassung der Ergebnisse wird in Tabelle \ref{tab:vergleich} dargestellt: 
\begin{table}[H] 
\centering 
\begin{tabular}{ |l|c|c|c|c| }
\hline
\textbf{Vendor} & \textbf{RSS} & \textbf{Twitter} & \textbf{Newsletter} & \textbf{Last Update} \\ \hline
\hline
Anynines & nein & ja & ja & 18.08.15 \\ \hline
App42 PaaS & nein & ja & nein & 10.05.17 \\ \hline
AppFog & nein & ja & ja & 10.05.17 \\ \hline
AppHarbor & nein & ja & nein & 22.05.16 \\ \hline
BitNami & nein & ja & nein & 10.05.17 \\ \hline
Brightbox & nein & ja & nein & 25.04.17 \\ \hline
Clever Cloud & nein & ja & nein & 26.04.17 \\ \hline
Cloudnode & nein & ja & nein & 23.04.17 \\ \hline
CloudUnit & nein & ja & nein & 10.05.17 \\ \hline
Cloudways & nein & ja & ja & 10.05.17 \\ \hline
EngineYard & nein & ja & nein & 29.04.17 \\ \hline
Flynn & nein & ja & ja & 17.10.16 \\ \hline 
fortrabbit & nein & ja & nein & 04.05.17 \\ \hline
Getup Cloud & nein & ja & nein & 09.05.17 \\ \hline
Google App Engine & nein & ja & ja & 10.05.17 \\ \hline
Heroku & ja & ja & ja & 10.05.17 \\ \hline
Jelastic & nein & ja & ja & 10.05.17 \\ \hline
Mendix & nein & ja & nein & 28.04.17 \\ \hline
mOSAIC & ja & nein & nein & 05.04.13 \\ \hline
OpenShift Container Platform & nein & ja & nein & 22.04.17 \\ \hline
Oracle Cloud PaaS & ja & ja & nein & 01.12.16 \\ \hline
OrangeScape & nein & ja & nein & 05.04.17 \\ \hline
Pagoda Box & nein & ja & nein & 06.05.16 \\ \hline
Platformer.com & ja & ja & nein & 01.04.17 \\ \hline
SAP HANA Cloud Platform & nein & ja & nein & 02.05.17 \\ \hline
Software AG Live & nein & ja & nein & 10.05.17 \\ \hline
Tsuru & nein & ja & ja & 12.04.17 \\ \hline
Voxoz & nein & ja & nein & 08.01.17 \\ \hline
WSO2 App Cloud & nein & ja & ja & 10.05.17 \\ \hline   
\end{tabular}
\caption {Vergleich der Informationsquellen (RSS, Twitter und Newsletter)} 
\label{tab:vergleich}
\end{table} 
In Bezug auf Web-Feeds spricht man von \glqq{}Content Syndication\grqq{}, was im deutschsprachigen Raum als \glqq{}Syndikation\grqq{} bezeichnet wird. Unter Syndikation wird Bereitstellung von Daten f�r �bertragung, Aggregierung und Online-Publikation\footnote{http://web.resource.org/rss/1.0/} verstanden. Ein verbreitetes Format f�r Web-Feeds ist RSS. Urspr�nglich fand RSS besondere Verbreitung bei Webbloggern \cite[S.103]{tilkov2015}. Aufgrund des einfachen Konzeptes ist RSS in k�rzer Zeit zum meistgenutzten Format zur �nderungsbenachrichtigung geworden.\\
Je nach Quelle wird die Abk�rzung RSS unterschiedlich interpretiert. In der ersten Version (RSS 1.0) stand die Abk�rzung f�r \glqq{}RDF Site Summary\grqq{}. In der Version 2.0 wird jedoch \glqq{}Really Simple Syndikation\grqq{}\footnote{https://validator.w3.org/feed/docs/rss2.html\label{rss2.0}} erw�hnt. Ein RSS 2.0 Dokument wird in XML-Format definiert und als Wurzel den Rss-Tag enth�lt, der den Namensraum der Elemente und die Version von RSS bestimmt. Im Rss-Tag wird der Channel-Tag definiert, der folgende Elemente enthalten muss (siehe \ref{rss2.0}):
\begin{itemize}
\item Title: Name des Channels (engl. Kanal)
\item Link: URL zur Webseite des entsprechenden Channels
\item Description: kurze Zusammenfassung des Channels
\end{itemize} 
Die Inhalte des Channels werden durch Items dargestellt. Laut der Spezifikation in \ref{rss2.0} entspricht ein Item einer Story (ein kurzer Artikel). Bei einem Item gibt es keine Pflichtfelder. Allerdings muss entweder Title oder Description vorhanden sein. Ein Beispiel f�r einen RSS 2.0 Feed\footnote{https://blog.heroku.com/news/feed} wird in Listing \ref{rss} dargestellt.
\begin{lstlisting}[basicstyle=\ttfamily, breaklines=true, label=rss,
					captionpos=b, caption={Beispiel eines Eintrages in RSS 2.0}]
<rss xmlns:dc="http://purl.org/dc/elements/1.1/" version="2.0">
 <channel>
  <title>Heroku</title>
  <link>http://blog.heroku.com</link>
  <description>The Heroku Blog</description>
  <item>
   <title>The Heroku-16 Stack is Now Available</title>
   <link>https://blog.heroku.com/heroku-16-is-generally-available</link>
   <pubDate>Thu, 20 Apr 2017 15:06:00 GMT</pubDate>
   <guid>https://blog.heroku.com/heroku-16-is-generally-available</guid>
   <description>
    <p>Your Heroku applications run on top of ...</p> 
   </description>
   <author>Jon Byrum</author>
  </item>
 </channel>
</rss>
\end{lstlisting}
Im Beispiel von Heroku werden au�erdem separate Feed-Channels f�r die unterschiedlichen Themenbereiche angeboten, wie es in Abbildung \ref{fig:heroku-feeds} zu sehen ist.
\begin{figure}[H] 
	\centering
	\includegraphics[width=1.0\textwidth]{images/heroku-feeds.png}
	\caption{Feed-Channels von Heroku}
	\label{fig:heroku-feeds}
\end{figure}
Der Themenbereich umfasst sowohl allgemeine (z.B. Heroku Blog) als auch spezifische Informationen (z.B. Dev Center Articles). Im Anwendungsfall von \textit{PaaSfinder} ist der Dev Center Changelog Channel besonders relevant, da der die Benachrichtigungen im technischen Bereich enth�lt (siehe Abbildung \ref{fig:heroku-changelog-feed}).
\begin{figure}[H] 
	\centering
	\includegraphics[width=1.0\textwidth]{images/heroku-changelog.png}
	\caption{Heroku Feed}
	\label{fig:heroku-changelog-feed}
\end{figure}
Neben RSS k�nnen die Daten aus sozialen Netzwerken erschlossen werden. Zu diesem Zweck stellen die Anbieter der sozialen Netzwerken (z.B. Facebook\footnote{https://developers.facebook.com/docs/graph-api/}, Twitter\footnote{https://dev.twitter.com/overview/api}) eine REST-API zur Verf�gung, die das Lesen/Schreiben der Daten erm�glicht. Im Beispiel von Heroku hat sich herausgestellt, dass sich Twitter am besten f�r Benachrichtigungen geeignet ist. Bei diesem konkreten Fall handelt es sich sogar um die gleiche Information wie in RSS-Feeds (vgl. Heroku in Abbildung \ref{fig:heroku-changelog-feed} und \ref{fig:heroku-twitter}).
\begin{figure}[H] 
	\centering
	\includegraphics[width=1.0\textwidth]{images/twitter.png}
	\caption{Heroku Twitter}
	\label{fig:heroku-twitter}
\end{figure}
Es bleibt also dem Entwickler offen, ob man sich f�r einen News-Dienst wie Twitter oder Web-Feeds entscheidet. In beiden F�llen kommt man auf die gleiche Information, die unterschiedlich repr�sentiert wird. W�hrend Twitter eine umfangreiche REST-API mit Daten in JSON Format anbietet, setzt RSS auf das XML-Format. Die RSS-Daten k�nnen mit dem HTTP-GET-Request aufgerufen werden.\\
Unabh�ngig davon, ob die Datenerfassung �ber Twitter API oder RSS erfolgt, k�nnen die Daten wiederum an die Worker-API gesendet werden. Im einfachsten Fall werden sie an das Github-Repository weitergegeben. In der fortgeschrittenen Variante k�nnen die Daten nach bestimmten Kriterien gefiltert bzw. verarbeitet werden. 
\newpage
\section{Fazit}\label{sec:Fazit}
Im folgenden Abschnitt werden die Ergebnisse der Arbeit zusammengefasst und ein Ausblick auf Richtungen in Future Work gegeben.\\
In Bezug auf den theoretischen Teil der Arbeit wurde ein Modell des automatisierten Wissenserwerbs erarbeitet (siehe Abbildung \ref{fig:wissenserwerbskomponente}). Das Modell sieht vor, dass die Datenerfassung je nach dem Anwendungsbereich unterschiedlich automatisierbar ist. Aus diesem Grund wurden drei Kategorien des Wissenserwerbs unterschieden, n�mlich indirekter, direkter und automatisierter Wissenserwerb. Dar�ber hinaus wurde die Komponente zur Daten�bermittlung definiert, die als Bindeglied zwischen den Wissenserwerbsmethoden und der Wissensbasis auftritt. \\
Im praktischen Teil wurde das Modell auf \textit{PaaSfinder} angewandt. Hinsichtlich der Zielsetzung wurde im ersten Schritt die Benutzerschnittstelle zur Aktualisierung eines Vendors entwickelt. Als n�chstes wurde ein Service implementiert, der f�r die automatische Erstellung von Pull Requests zust�ndig ist. \\
Potentielle Forschungsrichtungen wurden in Abschnitt \ref{subsec:future_work} angesprochen. Im Hinblick auf die Updates spielt der Kurznachrichtendienst Twitter eine wichtige Rolle. Twitter bietet au�erdem eine REST API an, die f�r den Zugriff auf die Daten benutzt werden kann. Im Weiteren verf�gen Blogs �ber Informationen, die f�r \textit{PaaSfinder} relevant sein k�nnen. Die Datenerfassung kann bei Blogs mithilfe von Web-Crawling erfolgen. Dabei besteht der Spielraum f�r die Anwendung des maschinellen Lernens auf die von Twitter und Blogs erfassten Daten.\\
Die Erweiterung der automatischen Tests soll in Zukunft ebenso ber�cksichtigt werden, um die Datenbank konsistent und fehlerfrei zu halten.
\newpage

% Einstellungen f�r Literaturverzeichnis
\addcontentsline{toc}{section}{\bibname}
\bibliographystyle{geralpha}
\selectlanguage{german}

% Hier das Literaturverzeichnis einbinden
\bibliography{bibliography/references}
\newpage

% Eigenst�ndigkeitserkl�rung
\makedeclaration{Bachelorarbeit}{31.03.2017}{Petr Vasilyev}

\end{document}