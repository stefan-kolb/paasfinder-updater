% Einbindung der Konfigurationsdatei
\input{includes/config.tex}

\begin{document}

% Titelblatt erstellen
\maketitle{Bachelor}{Wirtschaftsinformatik}{Automatisierung der Datenerfassung f�r Wissensdatenbanken im technischen Kontext}{Vasilyev Petr}{31.03.2017}

% Erstellung der Inhaltsverzeichnisses
\pagenumbering{Roman}

\tableofcontents
\newpage

\listoffigures
\newpage

%\listoftablesD
%\newpage

%\lstlistoflistings
%\newpage

\fancyhead[LO]{\footnotesize\sc\nouppercase{Abk�rzungsverzeichnis}}
\section*{Abk�rzungsverzeichnis}
% In Klammern steht das l�ngste Akronym!
\begin{acronym}[PaaS]
 \acro{IaaS}{Infrastructure-as-a-Service}
 \acro{PaaS}{Platform-as-a-Service}
 \acro{SaaS}{Software-as-a-Service}
\end{acronym}
\newpage
\fancyhead[LO]{\footnotesize\sc\nouppercase{\leftmark}}
\setcounter{page}{1}
\pagenumbering{arabic}

%
% Einzelne Kapitel mit \input{sections/Kapitel-File} einf�gen.
%
\section{Einf�hrung}\label{sec:Einf�hrung}
\subsection{Motivation}\label{subsec:Motivation}
Die Idee von wissensbasierten Systemen entstand aus dem Bed�rfnis, ein intelligentes System zu schaffen, das mittels spezifischen Wissens die Fachexperten bei den Problem\-l�\-sung\-en unterst�tzt \cite[S.18]{akerkar2010}. Eine Spezialisierung wissensbasierter Systeme stellen Expertensysteme dar, in denen das Wissen letztendlich von Experten entsprechender Dom�nen stammt. Ein wissensbasiertes System, bzw. ein Expertensystem, ist durch die Trennung der Wissensdarstellung eines Problembereichs und der Wissensverarbeitung gekennzeichnet \cite[S.11]{beierle2014}.\\ 
Die Beschreibung des Wissens eines wissensbasierten bzw. eines Expertensystems erfolgt in der Wissensbasis, die in Form einer Wissensdatenbank realisiert wird. Die Anfangswissensbasis wird in der Regel mithilfe manueller Datenerfassungen aufgebaut \cite[S.70]{kurbel1992}. Hierzu wird beispielsweise h�ufig die Methode des Interviews zwischen einem Wissens\-ingenieur und einem Wissenstr�ger eingesetzt \cite[S.76]{gottlob1990}. Allerdings sind die ma\-nu\-ell\-en Wissenserwerbsmethoden f�r die Weiterentwicklung der Wissensbasis hinsichtlich der Erweiterung und Aktualisierung der Wissensdatenbank eher schlecht geeignet, da sie fehleranf�llig sowie kosten- und zeitintensiv sind. Aus diesen Gr�nden gibt es Bestrebungen, den Prozess der Datenerfassung zu automatisieren. Eine Auswahl an bestehenden An\-s�tzen, die f�r die vorliegende Arbeit relevant sind, wird im Abschnitt \ref{subsec:Verwandte-Arbeiten} gegeben. Trotz der vielen Herangehensweisen gibt es kein einheitliches Konzept, das einen allgemeinen Rahmen f�r die Automatisierung der Datenerfassung bildet.
\subsection{Verwandte Arbeiten}\label{subsec:Verwandte-Arbeiten}
Der Ausgangspunkt dieser Arbeit stellt die Publikation \cite{tecuci1992} von G. Tecuci dar, die die Automatisierung der Wissenserfassung als ein Konzept der Erweiterung, Aktualisierung und Verbesserung der Wissensbasis beschreibt \cite[S.1444]{tecuci1992}. In diesem Zusammenhang wurde ein lernendes System vorgestellt, das eine Auswahl an generischen Ans�tzen des maschinellen Lernens bei der Wissenserfassung umsetzt \cite[S.1445]{tecuci1992}. Ferner wird ein Framework gebildet, das die Wissenserfassung durch maschinelles Lernen automatisiert. Dar�ber hinaus werden die gelernten Daten von einem Experten auf die Korrektheit �berpr�ft. Die Entwicklung der Wissensbasis wird in drei Phasen durchgef�hrt. In der ersten Phase wird die Anfangswissensbasis aufgebaut, die unvollst�ndig und teilweise widerspr�chlich sein kann. Die zweite Phase umfasst die inkrementelle Erweiterung und Verbesserung der Wissensbasis. Schlie�lich wird in der dritten Phase die Wissensbasis in Bezug auf Effizienz optimiert \cite[S.1444]{tecuci1992}. Die Kernaussage der Arbeit besteht darin, dass die Kooperation zwischen dem fachlichen Experten und dem Expertensystem in jeder Phase die Automatisierung der Datenerfassung deutlich erleichert. Beispielsweise k�nnen die Daten von einem Algorithmus generiert werden. Daraufforgend werden sie vom fachlichen Experten auf die formale und semantische Korrekteit �berpr�ft und in die Wissensbasis gespeichert \cite[S.1445]{tecuci1992}. In \cite{tecuci1994} wird die Entwicklung des Frameworks fortgef�hrt, wobei der Schwerpunkt im Bereich vom multi-strategischen Lernen (multistrategy leraning) liegt \cite[S.137]{tecuci1994}.\\
Neben \cite{tecuci1992} und \cite{tecuci1994} gibt es eine Reihe unterschiedlicher Ans�tze, die das maschinelle Lernen zur Automatisierung der Datenerfassung benutzen. Einige Beispiele sind \cite{castro1999}, \cite{castro2001}, \cite{webb1996}. Dabei ist die Idee der Zusammenarbeit zwischen dem Experten und dem Lernalgorithmus durchaus verbreitet. Ein Beispiel stellt die Arbeit von Castro et al. \cite{castro2001} dar. In der grundlegenden Idee beziehen sich die Autoren auf die Arbeit von \cite{tecuci1992}. Als Startpunkt wird eine unvollst�ndige Anfangswissensbasis betrachtet, die schrittweise verbessert wird, indem der Experte die Fragen vom System beantwortet. Bei der Frageerstellung wird ein Lernalgorithmus eingesetzt, der aus einer Trainingsmenge die Regeln lernt und kontinuierlich die Qualit�t der Fragen verbessert. Dabei betonen Castro et al. \cite{castro2001}, dass der Lernalgorithmus keineswegs den Wissensingenieur ersetzen kann. Vielmehr soll der Algorithmus den Wissensingenieur die Routinearbeit abnehmen und bei den schwierigeren Aufgaben unterst�tzen, indem verschiedene Varianten der Interviews vom Algorithmus vorgeschlagen werden \cite[S.308]{castro2001}.\\
Ein praxisorientierter Ansatz f�r die Autormatisierung der Wissenserfassung wird in \cite{gebus2009} thematisiert. Allgemein handelt es sich um die Transformation eines datenbasierten Systems in ein wissensbasiertes System, um die Effizient der Produktion zu steigern. Im Hinblick auf die Automatisierung der Aktualisierung und Erweiterung der unvollst�ndigen Wissensbasis beziehen sich Gebus und Leivisk{\"a} auf die Erkenntnisse aus \cite{tecuci1992}, \cite{winter1992} und \cite{su2002}. Bez�glich der Erfassung von Erfahrungswissen nehmen die Autoren den Bezug auf die Arbeit von Okamura et al \cite{okamura1991}, die Heuristik bei Probleml�sungen eingesetzt. Im praktischen Teil wird ein bereits bestehendes datenbasiertes \acf{DSS} betrachtet, das die Unternehmensf�hrung bei den Entscheidungen in Bezug auf die Produktionsoptimierung unterst�tzen soll. Allerdings werden die St�rungen in der Produktion von Anlagenbedienern (Experten) mithilfe von Erfahrungswissen intern behoben, ohne dieses Wissen weiterzugeben. Als Folge hat die Unternehmensf�hrung kein umfangreiches Bild der Produktion, was sich negativ auf die Produktion auswirkt. Aus diesem Grund erweitern Gebus und Leivisk{\"a} das System um eine Wissenstr�gerschnittstelle, um das Expertenwissen in die Datenbank zu integrieren \cite[S.94]{gebus2009}. Gebus und Leivisk{\"a} veranschaulichen damit, wie die Idee der Zusammenarbeit zwischen dem Experten und dem wissensbasierten System zur Automatisierung der Wissenserfassung im Kontext eines Unternehmens umgesetzt werden kann.\\
Mit der rasanten Entwicklung des Word Wide Web hat sich eine Forschungsrichtung ergeben, die sich mit der Daten- und Wissenserfassung aus Netzwerken unterschiedlicher Art befasst. Ein fundierter �berblick �ber die Webdatenerfassung sowie aktuelle Ans�tze und Anwendungen wurde in \cite{ferrara2014} vorgestellt. Dabei werden sowohl theoretische als auch praktische Aspekte umfassend thematisiert. In der theoretischen Hinsicht wurden zuerst allgemeine Probleme wie Automatisierungsgrad, Skalierbarkeit, Datenschutz, Unstabilit�t der Ressourcenstruktur und Trainingsmenge angesprochen \cite[S.301-302]{ferrara2014}. In Bezug auf die Ans�tze wurden Baumparadigma (z.B. \cite{dave2003} und \cite{wang1998}), Web Wrapper (z.B. \cite{sahuguet1999}) und hybrides System (z.B. \cite{crescenzi2001}) durch die Analyse zahlreicher Publikationen systematisiert. Bei der Anwendungen zur Webdatenerfassung w�chst der Trend Richtung freier Open-Source Projekte, die mit kommerziellen Anwendungen im Wettbewerb stehen \cite[S.310]{ferrara2014}. Die Autoren nennen als Beispiel die Pipes\footnote{https://en.wikipedia.org/wiki/Yahoo!\_Pipes} von Yahoo. Bedauerlicherweise ist das ein veraltetes Beispiel, da die Plattform nicht mehr unterst�tzt wird. Die Tatsache best�tigt nun die Aussage �ber die hohe Dynamik im Webtechnologiebereich. Nichtsdestotrotz gibt es einige Anwendungen, die �ber mehrere Jahre hinweg �berleben und enorm beliebt sind. Ein Beispiel stellen Feed-Services RSS/Atom das, die die Pull-Benachrichtigen �ber die �nderungen auf der Webseite, Blogs usw. erm�glichen (mehr dazu in \cite{hammersley2005} und \cite{tilkov2015}). Bezogen auf Anwendungen, die im Forschungsrahmen entstanden sind, wird nehmen die Autoren in \cite{ferrara2014} Lixito aus \cite{baumgartner2001} als Beispiel.


%Die Autoren thematisieren die Datenextraktion in unterschiedlichen Anwendungsbereichen.
%In diesem Zusammenhang wurden zahlreiche Systeme in unterschiedlichen Anwendungsgerechten wie Business Intelligence, Web-Crawling oder sogar Bioinformatik ausgewertet.
\subsection{Zielsetzung}\label{subsec:Zielsetzung}
Das Ziel der vorliegenden Arbeit ist die Erarbeitung eines allgemeinen Konzeptes zur Automatisierung der Datenerfassung. Da die komplett automatisierte Datenerfassung sehr schwierig umzusetzen ist, liegt der Schwerpunkt dieser Arbeit auf der Kombination zwischen den manuellen und maschinellen Vorgehensweisen.\\
Beim technischen Kontext wird angedeutet, dass die Umsetzung im Rahmen eines bestimmten Anwendungsbereichs erfolgt. Es wird also kein Allgemeinwissen wie in \cite{tandon2016}, sondern ein anwendungsbezogenes Wissen betrachtet. In diesem Zusammenhang wird das Konzept auf Basis eines Expertensystems entwickelt, da die Wissensbasis eines Expertensystems spezifischer ist als bei einem wissensbasierten System.\\
Die praktische Umsetzung soll auf Basis von \textit{PaaSfinder}\footnote{https://paasfinder.org} erfolgen. Bei \textit{PaaSfinder} handelt es sich um eine Web-Anwendung, die �ber eine Wissensdatenbank im Bereich \ac{PaaS} verf�gt. \ac{PaaS} geh�rt neben \ac{IaaS} und \ac{SaaS} zum Themengebiet von Cloud Computing \cite{nist2011} und soll die Anwendungsentwicklung erleichtern, indem die Laufzeit- ober Entwicklungsumgebungen von einem \ac{PaaS}-Anbieter dem Kunden vorkonfiguriert angeboten werden \cite[S.14]{lawton2008}. Da \textit{PaaSfinder} ein Open-Source Projekt ist, kann jeder der Datenbank von \textit{PaaSfinder} beitragen. Allerdings ist die Mitwirkung mit einem hohen Aufwand verbunden und setzt Informatikkenntnisse voraus. Au�erdem werden die Daten von \textit{PaaSfinder} haupts�chlich auf den Webseiten von \ac{PaaS}-Anbietern manuell erfasst. Dies stellt ein hohes Entwicklungspotential, die Erfassung von solchen Daten zu automatisieren.
\subsection{Aufbau der Arbeit}\label{subsec:Aufbau-der-Arbeit}
Die vorliegende Arbeit ist wie folgt aufgebaut. In Kapitel 2 wird der Begriff und die grundlegende Architektur eines Expertensystems erl�utert. Darauffolgend werden die Bestandteile, die f�r das Konzept relevant sind, n�her betrachtet. Anschlie�end wird schematisch das Kontext der Datenerfassung dargestellt und Automatisierungsm�glichkeiten angesprochen. In Kapitel 3 wird genauer auf die Automatisierungsmethoden bei der Datenerfassung eingegangen. Dabei werden bestehende Forschungsergebnisse vorgestellt und konzeptuell verallgemeinert. Kapitel 4 umfasst die praktische Umsetzung der Erkenntnisse, in Kapitel 2 und 3 gewonnen wurden. In Kapitel 5 wird die Implementierung evaluiert und die Ergebnisse der Arbeit zusammengefasst. Anschlie�end wird in Kapitel 7 ein Ausblick f�r zuk�nftige Forschung im betrachteten Bereich gegeben.
\newpage
\section{Grundlagen von Expertensystemen}\label{sec:Grundlagen}
\subsection{Begriffsdefinition}\label{subsec:Begriffsdefinition}
Urspr�nglich waren Expertensysteme Anwendungsprogramme, die logische Schlussfolgerungen aus einer Wissensbasis ziehen konnten. Au�erdem konnten sie �berpr�fen, ob eine Aussage aus einer vorhandenen Wissensbasis abgeleitet werden kann \cite[S.75]{greer2010}. Daher handelt es sich in der fr�heren Literatur meist um Anwendungen, die ihr Wissen in Form von logischen Ausdr�cken darstellen und in der Lage waren, neue Erkenntnisse von bestehendem Wissen abzuleiten \cite{tecuci1992}. Im Laufe der Zeit hat sich das Konzept eines Expertensystems auf andere Anwendungsbereiche ausgeweitet. Aus diesem Grund gibt es mehrere Definitionen, die im Allgemeinen �hnlich sind und im Spezifischen Merkmale des zugeh�rigen Anwendungsbereichs beinhalten.\\
Allgemein l�sst sich sagen, dass ein Expertensystem ein Computersystem (Hardware und Software) ist, das in einem bestimmten Bereich Wissen und Schlussfolgerungsf�higkeit eines menschlichen Experten nachbildet \cite[S.12]{beierle2014}. Aus Sicht der Wirtschaftsinformatik zielen Expertensysteme darauf ab, das Expertenwissen menschlicher Fachleute in der Wissensbasis eines Computers abzuspeichern und f�r eine Vielzahl von Probleml�sungen zu nutzen \cite[S.59]{mertens2012}. Im Weiteren gehen Beierle und Kern-Isberner auf die Eigenschaften ein, die ein Expertensystem aufweisen sollen \cite[S.12]{beierle2014}. Im Rahmen dieser Arbeit sind folgende Eigenschaften besonders relevant:
\begin{itemize}
\item Anwendung des Wissens eines oder mehrerer Experten, um Probleme in einem bestimmten Anwendungsbereich zu l�sen,
\item Leicht lesbare Wissensdarstellung,
\item M�glichst anschauliche und intuitive Benutzerschnittstelle,
\item Leichte Wartbarkeit und Erweiterbarkeit des Wissens im Expertensystem,
\item Unterst�tzung beim Wissenstransfer vom Experten zum System.
\end{itemize}
Hier ist es au�erdem wichtig anzumerken, dass die Begriffe \grqq{}K�nstliche Intelligenz\grqq{}, \grqq{}wissensbasiertes System\grqq{} und \grqq{}Expertensystem\grqq{} in einer engen Beziehung zueinander stehen. Haun stellt eine systematische Abgrenzung dieser Begriffe vor, die sich folgenderma�en beschreiben l�sst \cite[S.30]{haun2000}:
\begin{itemize}
\item \textit{K�nstliche Intelligenz} stellt den Oberbegriff dar und bildet den theoretischen Rahmen f�r die Entwicklung von wissensbasierten Systemen und Expertensystemen.
\item \textit{Wissensbasierte Systeme} sind eine Teilmenge der Anwendungen innerhalb des Bereichs der k�nstlichen Intelligenz. Sie wenden die Wissensverarbeitung auf ein konkretes Aufgabengebiet an und verwalten Allgemeinwissen explizit und getrennt vom Rest des Systems.
\item \textit{Expertensysteme}, die ein Teilbereich der wissensbasierten Systeme sind, stellen eine Spezialisierung von wissensbasierten Systemen dar. Sie verwalten spezifisches Expertenwissen, das von einem Experten stammt und auf praxisbezogene Probleme angewandt wird.
\end{itemize}
Graphisch l�sst sich die vorliegende Abgrenzung in Abbildung \ref{Abgrenzung} darstellen:
\begin{figure}[H] 
	\centering
	\includegraphics[width=0.55\textwidth]{images/abgrenzung.png}
	\caption{Begriffsabgrenzung, \cite[S.30]{haun2000}}
	\label{Abgrenzung}
\end{figure} 
Nach dieser Abgrenzung l�sst sich feststellen, dass der Unterschied zwischen einem wissensbasierten System und einem Expertensystem darin besteht, dass das Wissen im Endeffekt von einem Experten stammt. Allerdings ist dieses Kriterium nicht besonders aussagekr�ftig. Beierle und Kern-Isberner weisen darauf hin, dass nach diesem Kriterium viele der existierenden wissensbasierten Systeme als Expertensysteme bezeichnet werden k�nnten \cite[S.11]{beierle2014}. Als Reaktion auf fehlende Kriterien stellen die Autoren die Eigenschaften eines Experten dar, die sich folgenderma�en zusammenfassen lassen:
\begin{itemize}
\item Experten sind selten und teuer.
\item Experten sind nicht immer verf�gbar.
\item Leistungsf�higkeit der Experten ist nicht konstant, sondern kann nach Tagesverlauf schwanken.
\item Expertenwissen kann oft nicht als solches weitergegeben werden.
\item Expertenwissen kann verloren gehen.
\end{itemize}
Ein gutes Beispiel hinsichtlich der Gefahr, dass Expertenwissen verloren gehen kann, wird in \cite[S.94]{gebus2009} vorgestellt. Gebus nimmt hier Bezug auf die Mitarbeiter der sogenannten Baby-Boomgeneration. Es handelt sich um Experten, die ein umfangreiches Erfahrungswissen besitzen und bald aus Altersgr�nden das Unternehmen verlassen. Somit geht auch das Erfahrungswissen aus dem Unternehmen verloren.\\
Zusammenfassend l�sst sich sagen, dass die Entwicklung eines Expertensystems ein hohes Potenzial besitzt. Allerdings kann ein Expertensystem nicht als Ersatz f�r einen menschlichen Experten betrachtet werden. Vielmehr geht es um eine Erfassung, Darstellung und Pflege des Expertenwissens in einem Expertensystem, um die Arbeitsprozesse effizienter zu gestalten und sowohl erfahrene als auch neue Anwender in einem bestimmten Wissensbereich bei der Aufgabenabwicklung zu unterst�tzen.
\subsection{Architektur eines Expertensystems}\label{subsec:Architektur}
Beierle und Kern-Isberner betonen, dass die Trennung zwischen der Darstellung des Wissens (Wissensbasis) und der Wissensverarbeitung (Wissensverarbeitungskomponente) der wichtigste Aspekt eines Wissensbasierten Systems ist. \cite[S.11]{beierle2014}. Die Wissensbasis kann man sich als eine Art Datenstruktur vorstellen, in der das ben�tigte Wissen gespeichert wird. Die Wissensverarbeitungskomponente umfasst eine Menge von anwendungsunabh�ngigen Algorithmen, die mithilfe der Wissensbasis eine L�sung f�r ein gegebenes Problem erarbeiten. Somit stehen die Wissensbasis und die Wissensverarbeitungskomponente in einer engen Beziehung zueinander \cite[S.18]{kurbel1992}.\\
Allgemein umfasst ein Expertensystem folgende Bestandteile \cite[S.75]{greer2010}:
\begin{itemize}
\item \textit{Wissensbasis}, die Expertenwissen in Form von Fakten in einer bestimmten Sprache speichert sowie Regeln zur Wissensorganisation beinhaltet.
\item \textit{Inferenzmaschine}, die unter Ber�cksichtigung des zugrunde liegenden Wissensbedarfs die Wissensbasis
durchsucht bis das System einen Probleml�sungsvorschlag erarbeitet hat oder herausfindet, dass keiner existiert.
\item \textit{Dialogkomponente}, die eine Schnittstelle zwischen dem Nutzer und dem System darstellt. 
\item \textit{Erkl�rungskomponente}, die dem Benutzer erl�utert, warum und auf welche Weise eine bestimmte L�sung gefunden bzw. nicht gefunden wurde \cite[S.126]{haun2000}.
\item \textit{Wissensakquisitionskomponente}, die den Entwickler des Expertensystems bei der Erweiterung, �nderung und Wartung der Wissensbasis unterst�tzt.
\end{itemize}
Laut Tecuci stellen Wissensbasis und Inferenzmaschine grundlegende Bestandteile eines Expertensystems dar und bilden damit den Kern des Expertensystems \cite[S.1444]{tecuci1992}. Dialogkomponente, Erkl�rungskomponente und Wissensakquisitionskomponente geh�ren zur sogenannten Schale und sind f�r die Kommunikation zwischen dem Systemverwalter und dem Nutzer zust�ndig (siehe Abbildung \ref{expertensystem_haun}). 
\begin{figure}[H] 
	\centering
	\includegraphics[width=1.0\textwidth]{images/expertensystem_haun.png}
	\caption{Architektur eines Expertensystems nach Haun, \cite[S.126]{haun2000}}
	\label{expertensystem_haun}
\end{figure}
Im Hinblick auf die Interaktion gibt es drei Gruppen, die mit dem Expertensystem interagieren: 
\begin{itemize}
\item \textit{Nutzer}, der das Expertensystem zum L�sen eines Problems benutzt und mit der Dialogkomponente kommuniziert. Der Wissensingenieur und der Experte k�nnen ebenso als Nutzer auftreten \cite[S.758]{wachsmuth1993}.
\item \textit{Wissensingenieur}, der sich mit dem Aufbau und Wartung der Wissensbasis besch�ftigt. Unter anderem ist Wissensmodellierung ein wichtiger Aufgabenbereich eines Wissensingenieurs \cite[S.742]{wachsmuth1993}.
\item \textit{Experte}, der �ber spezifisches Erfahrungswissen verf�gt, das f�r das Expertensystem relevant ist.
\end{itemize}
Der Ablauf der Kommunikation zwischen dem Nutzer und dem Expertensystem sieht folgenderma�en aus: 
\begin{itemize}
\item Der Nutzer schickt eine Anfrage an die Dialogkomponente des Expertensystems.
\item Die Dialogkomponente �bermittelt die Anfrage an die Inferenzmaschine.
\item Die Inferenzmaschine erarbeitet eine L�sung f�r das gegebene Problem mittels der Wissensbasis und gibt das Ergebnis an die Dialogkomponente zur�ck. 
\item Anschlie�end teilt die Dialogkomponente dem Nutzer die L�sung des Problems mit. Falls keine L�sung zum Problem existiert, wird eine entsprechende Fehlermeldung angezeigt.
\end{itemize}
Auf der anderen Seite k�nnen die Inhalte der Wissensbasis von einem Wissensingenieur mithilfe der Wissensakquisitionskomponente beeinflusst werden. Der Wissenserwerb durch den Wissensingenieur ist die verbreitetste Vorgehensweise, neue Daten f�r ein wissensbasiertes System zu erschlie�en. Meistens handelt es sich um ein Interview zwischen dem Wissensingenieur und dem Experten \cite[S.76]{greer2010}, \cite[S.210]{fujihara1997}. Neben dem Interview kann der Wissensingenieur eine Recherche der verf�gbaren Wissensquellen wie Text, technische Zeichnungen oder Web-Ressourcen durchf�hren. Anschlie�end werden die Daten vom Wissensingenieur formalisiert und in die Wissensbasis gespeichert. \\
Die Wissensbasis kann in einigen F�llen von einem fachlichen Experten beeinflusst werden. Daf�r ist eine geeignete Expertenschnittstelle innerhalb der Wissensakquisitionskomponente notwendig, die den Experten erm�glicht, ihr Erfahrungswissen selbst zu formalisieren und gegebenenfalls zu warten \cite[S.743]{wachsmuth1993}. Die �berpr�fung des Dateninputs ist ebenfalls die Aufgabe der Wissensakquisitionskomponente. Dies kann mittels Durchf�hrung automatisierten Tests bei jeder �nderungsanfrage erfolgen, um die Konsistenz der Wissensbasis zu gew�hrleisten \cite[S.743]{wachsmuth1993}.\\
Um ein geeignetes Konzept der automatisierten Datenerfassung zu entwickeln, ist ein grundlegendes Verst�ndnis von der Struktur und Funktionsweise der Wissensbasis sowie der Wissensakquisitionskomponente erforderlich. Im weiteren Verlauf der Arbeit werden die Erkenntnisse �ber die Wissensbasis und die Wissensakquisitionskomponente erl�utert, die in der Forschung von Expertensystemen entstanden sind.

\subsection{Wissensbasis}\label{subsec:Wissensbasis}
Neben der Inferenzmaschine stellt die Wissensbasis den zentralen Teil eines Expertensystems, der die Daten des gesamten Systems beinhaltet \cite[S.754]{wachsmuth1993}. Im Folgenden werden der allgemeine Prozess der Wissensbasisentwicklung, der Inhalt der Wissensbasis und die M�glichkeiten der Wissensrepr�sentation thematisiert. Gheorghe Tecuci beschreibt folgende Phasen bei der Entwicklung der Wissensbasis \cite[S.1444]{tecuci1992}: 
\begin{itemize}
\item[1.] Systematische Erfassung vom Expertenwissen
\item[2.] Verfeinerung der Wissensbasis
\item[3.] Reorganisation der Wissensbasis
\end{itemize}
In der ersten Phase werden das Vokabular und die geeignete Wissensrepr�sentation festgelegt. Gebus und Leivisk{\"a} betonen, dass die Wissensrepr�sentation den entscheidenden Einfluss auf die Generierung und sp�tere Handhabung der Wissensbasis hat \cite[S.95]{gebus2009}. Die initialen Daten werden meistens im Rahmen eines Interviews zwischen dem Wissensingenieur und dem Experten erfasst \cite[S.1444]{tecuci1992}. Das Ergebnis der ersten Phase ist eine initiale Wissensbasis, die unvollst�ndig und teilweise widerspr�chlich ist. In der zweiten Phase wird die initiale Wissensbasis mithilfe der geeigneten Datenerfassungsmethoden solange erweitert und verbessert, bis sie vollst�ndig und korrekt genug ist, um ein gegebenes Problem richtig zu l�sen. In der dritten Phase wird die vollst�ndige und korrekte Wissensbasis reorganisiert, um die Effizienz der L�sungsberechnung zu steigern \cite[S.1445]{tecuci1992}. Zusammenfassend werden die Phasen in der Abbildung \ref{drei_phasen} dargestellt.  
\begin{figure}[H] 
	\centering
	\includegraphics[width=0.7\textwidth]{images/drei_phasen.png}
	\caption{Phasen der Expertensystementwicklung, \cite[S.138]{tecuci1994}}
	\label{drei_phasen}
\end{figure}
In der Abbildung \ref{drei_phasen} sieht man, dass der Autor dem Experten die gesamte Kontrolle �ber die Entwicklung der Wissensbasis zuweist. Allerdings ist diese Sichtweise nicht vollst�ndig, da im Entwicklungsprozess der Wissensingenieur und der Systementwickler beteiligt sind und dementsprechend ber�cksichtigt werden sollen.\\
In Bezug auf den Inhalt der Wissensbasis unterscheiden Beierle und Kern-Isberner folgende Wissensarten \cite[S.5]{beierle2014}:
\begin{itemize}
\item \textit{Fachspezifisches Wissen}. Dabei handelt es sich um das spezifischste Wissen, das sich
nur auf den gerade betrachteten Problemfall bezieht. Das sind z.B. Fakten, die von Beobachtungen oder Untersuchungsergebnissen stammen.
\item \textit{Regelhaftes Wissen}, das den eigentlichen Kern der Wissensbasis darstellt. Dieses Wissen kann noch genauer differenziert werden: 
	\begin{itemize}
	\item \textit{Bereichsbezogenes Wissen}, das sich auf den gesamten Problembereich beziehen. Das kann sowohl theoretisches Fachwissen als auch Erfahrungswissen sein. Anders gesagt handelt es sich um generisches Wissen.
	\item \textit{Allgemeinwissen}, das z.B. um generelle Probleml�sungsheuristiken, Optimierungsregeln oder auch allgemeines Wissen �ber Objekte und Beziehungen in der realen Welt beinhaltet.
	\end{itemize}
\end{itemize}
Unter Ber�cksichtigung der Differenzierung der Wissensarten innerhalb der Wissensbasis beschreiben die Autoren in \cite[S.18]{beierle2014} auf eigene Weise die Architektur des Expertensystems, die in der Abbildung \ref{expertensystem_beierle}  dargestellt wird.
\begin{figure}[H] 
	\centering
	\includegraphics[width=0.7\textwidth]{images/expertensystem_beierle.png}
	\caption{Expertensystem nach Beierle und Kern-Isberner, \cite[S.18]{beierle2014}}
	\label{expertensystem_beierle}
\end{figure} 
Laut Beierle und Kern-Isberner k�nnen verschiedene Wissensarten in einem wissensbasierten System je nach dem Anwendungsbereich unterschiedlich umfangreich auftreten. Ein hochspezialisiertes System kann beispielsweise �ber sehr wenig oder gar kein Allgemeinwissen verf�gen. Auf der anderen Seite kann ein wissensbasiertes System den Schwerpunkt auf das gew�hnliche Alltagswissen legen \cite[S.5-6]{beierle2014}.\\
Ein weiterer Aspekt beim Aufbau der Wissensbasis ist die Wissensrepr�sentation. Die grundlegende Aufgabe der Wissensrepr�sentation ist die Formularisierung von Wissen, um eine maschinelle Verarbeitung  erst zu erm�glichen \cite[S.22]{haun2000}. Sinz und Ferstl unterscheiden folgende Formen der Wissensrepr�sentation \cite[S.366]{sinz2013}:
\begin{itemize}
\item \textit{Regelorientierte Darstellung}, in der das Wissen in Form von WENN-DANN-Regeln beschrieben wird. Diese Darstellungsform wird beispielsweise bei Prolog-Regeln eingesetzt.
\item \textit{Objektorientierte Darstellung}, die das Konzept der Objekttypen �bernimmt und mit deklarativen Operatorbeschreibungen verbindet.
\item \textit{Constraints Darstellung}, die Modellbeschreibungen aus dem Operations Research benutzt. Dabei handelt es sich um L�sungsr�ume durch Nebenbedingungen und Zielvorgaben.   
\end{itemize}
Hinsichtlich der Wissensrepr�sentation stellen Ferstl und Sinz imperative und deklarative Paradigmen gegen�ber \cite[S.366]{sinz2013}. Ein Programm, das dem imperativen Paradigma folgt, besteht aus einer Folge von Befehlen, die nacheinander ausgef�hrt werden \cite[S.341]{sinz2013}. Bei einem deklarativen Programm handelt es sich um eine Beschreibung der Aufgabenau�ensicht. Ein deklaratives Programm hat keine festgelegten L�sungsverfahren je Aufgaben. Stattdessen wird eine L�sung zum Zeitpunkt der Aufgabendurchf�hrung mittels Inferenzmaschine abgeleitet \cite[S.361]{sinz2013}.\\
Allgemein beziehen sich die Autoren darauf, dass an ein wissensbasiertes System nur geringe Anforderungen bez�glich Vollst�ndigkeit, Widerspruchsfreiheit und Eindeutigkeit gestellt werden k�nnen. Aus diesem Grund ist das deklarative Paradigma f�r die Wissensrepr�sentation besser geeignet. Folgende Gr�nde nennen die Autoren f�r die deklarative Umsetzung der Wissensbasis \cite[S.366]{sinz2013}: 
\begin{itemize}
\item \textit{Wissensdarstellung}: Da ein Mensch das Erfahrungswissen durch assoziative Beziehungsmuster aufbaut, ist die deklarative Wissensdarstellung eher geeignet.
\item \textit{Wissensauswertung}: �nderungen von Erfahrungswissen werden normalerweise in deklarativen Form erfasst.
\item \textit{Wissensverf�gbarkeit}: Die Codewartung von einem imperativen Programm ist fehleranf�llig, kosten- und zeitintensiv, da das Erfahrungswissen h�ufig ge�ndert und aktualisiert werden muss.
\end{itemize}
Der objektorientierte Ansatz ist eine weitere M�glichkeit, das Wissen zu beschreiben. Ein Beispiel f�r die objektorientierte Implementierung wird in \cite{leung1990} vorgestellt. Die Wissensbasis wird dabei als eine Sammlung von Klassen, Objekten und Methoden definiert \cite[S.40]{leung1990}. Der gro�e Vorteil solcher Umsetzung besteht in der Modularit�t des Wissens. Das Wissen in unabh�ngige Module aufgeteilt wird. Da die einzelnen Module unabh�ngig voneinander sind, k�nnen sie getrennt getestet und modifiziert werden, ohne den Rest der Wissensbasis zu beeintr�chtigen. Dies erm�glicht hohe Flexibilit�t bei der Wissensbasiserweiterung \cite[S.43]{leung1990}.\\
Neben der Implementierung der Wissensbasis ist eine geeignete Umsetzung der Wissenserwerbskomponente erforderlich, um die Wissensbasis aktuell, m�glichst fehlerfrei und konsistent zu halten. Im Folgenden wird die Wissenserwerbskomponente in Hinsicht auf den allgemeinen Aufbau und Funktionen thematisiert.  
\input{sections/kapitel-2/wissensakquisitionskomponente.tex}
\newpage
\section{Automatisierung der Wissenserfassung}\label{sec:Datenerfassung}
Mit der Wissensakquisitionskomponente wurde bereits angedeutet, dass die Datenerfassung f�r Expertensysteme meistens nur teilweise automatisierbar ist. Die Autoren in \cite{tecuci1994} weisen ebenso darauf hin, dass manuelle und maschinelle Wissenserschlie�ung jeweils eigene St�rke haben, die gegenseitig nicht ersetzt werden k�nnen \cite[S.137]{tecuci1994}. Aus diesem Grund ist bei der Datenerfassung ein hybrides Modell sinnvoll, das die Vorteile manueller und maschineller Verfahren kombiniert. Aufgrund der Fragestellung dieser Arbeit wird es im weiteren auf automatisierte und halb-automatisiere Bestandteile der Wissensakquisitionskomponente beschr�nkt.
\subsection{Die Wissenstr�gerschnittstelle}
Bei der Wissenstr�gerschnittstelle handelt es sich um eine Methode zur Datenerfassung, die einen direkten Wissenstransfer zwischen einem Wissenstr�ger und dem Expertensystem erm�glicht. Da die Daten manuell eingegeben und maschinell verarbeitet werden, bezeichnet man die Wissenstr�gerschnittstelle als semi-automatisierte Wissenserfassungsmethode. Im Folgenden werden zwei Anwendungsbeispiele aus unterschiedlichen Bereichen betrachtet und in Bezug auf die wichtigsten Erkenntnisse bei der Entwicklung und dem Einsatz zusammengefasst. \\
Ein Beispiel des Einsatzes der Wissenstr�gerschnittstelle bei einem Elektrotechnikunternehmen wird in \cite{gebus2009} vorgestellt. In der Fallstudie wird ein bereits bestehendes \ac{DSS} untersucht, das die F�hrungskr�fte bei den Entscheidungen von nicht-strukturieren Problemen in der Produktion unterst�tzt \cite[S.94]{gebus2009}. Das Problem dabei besteht darin, dass in der Datenbank des DSS nur die Daten von Produkteigenschaften gesammelt werden. Die Produktionsprozesse werden allerdings von Experten gesteuert, die mithilfe des Erfahrungswissens St�rungen in der Produktion beseitigen. Als Folge hat die Unternehmensf�hrung einen begrenzten �berblick �ber die Situation in der Produktionsabteilung. Au�erdem besteht die Gefahr, dass das spezifische Expertenwissen verloren geht falls der Wissenstr�ger das Unternehmen verl�sst.\\
Die Zielsetzung von \cite{gebus2009} ist das Expertenwissen ins System zu integrieren und dementsprechend ein wissensbasiertes System zu schaffen. Um das Wissen aus der Produktionsabteilung zu erschlie�en, wollen Gebus und Leivisk{\"a} das System um eine Schnittstelle (User Interface) f�r Anlagenbediener erweitern. Als erster Schritt spezifizieren Gebus und Leivisk{\"a} die Nutzergruppen des Systems:
\begin{itemize}
\item Anlagenbediener (Experte), der sein Wissen zu den St�rungen mittels der Wissenstr�gerschnittstelle in die Datenbank eingibt.
\item Administrator, der das gesamte DSS verwaltet und gegebenenfalls die Systemeinstellungen im Hinblick auf die Informationen vom Experten anpasst.
\item Qualit�tsabteilung, die die St�rungsstatistik analysiert, eine Qualit�tsr�ckmeldung an die Produktion und einen Bericht an die F�hrungskraft �bermittelt.
\item F�hrungskraft, die eine umfangreiche �bersicht von der Qualit�tsabteilung erh�lt und davon ausgehend Entscheidungen zur Prozessoptimierung trifft.
\end{itemize}
Jeder Nutzergruppe wird eine geeignete Benutzerschnittstelle zur Verf�gung gestellt. Schematisch l�sst sich die Struktur und die Informationsfl�sse im DSS in der Abbildung \ref{dss_gebus} vereinfacht nachbilden.
\begin{figure}[H] 
	\centering
	\includegraphics[width=0.9\textwidth]{images/dss_gebus.png}
	\caption{Die Struktur und die Informationsfl�sse im DSS, \cite[S.98]{gebus2009}}
	\label{dss_gebus}
\end{figure} 
%Gem��
\subsection{Datenextraktion aus dem Web}
Das Web stellt eine der wichtigsten Quellen der unstrukturierten und semi-strukturierten Daten dar. Es wurden zahlreiche Methoden der Datenextraktion aus dem Web entwickelt, die in unterschiedlichen Anwendungsbereichen eingesetzt werden. Ferrara et al unterscheiden dabei zwei Kategorien der Algorithmen zur Datenextraktion, n�mlich Tree Matching Algorithmen und Algorithmen des maschinellen Lernens \cite[S.302]{ferrara2014}. Da das maschinelle Lernen wird explizit im Abschnitt \ref{subsec: Maschinelles-Lernen} angesprochen, wird im Weiteren ein Beispiel eines Tree Matching Algorithmus betrachtet.\\
Aufgrund der Struktur der Webressourcen ist die Erfassung der Daten aus dem Web nicht einfach. Ferrara et al nennen folgende Herausforderungen, die bei der Datenerfassung aus Webressourcen vorkommen \cite[S.302]{ferrara2014}:
\begin{itemize}
\item \textit{Automatisierungsgrad}: Die Erfassung der Webdaten soll oft von einem menschlichen Experten �berwacht werden, um die Genauigkeit der Daten zu gew�hrleisten.
\item \textit{Skalierbarkeit}: Bei den umfangreichen Webressourcen soll innerhalb k�rzer Zeit schnell eine gro�e Datenmenge bearbeitet werden.
\item \textit{Datenschutz}: Wenn es um die Erfassung der personenbezogenen Daten geht (bei sozialen Netzwerken wie Facebook), soll die Privatsph�re des Individuums nicht beeintr�chtigt werden.
\item \textit{�nderung der Ressourcenstruktur}: Die Struktur der Webressourcen �ndert sich oft. Die Datenerfassungsmethoden f�r das Web sollen eine gewisse Flexibilit�t besitzen, um weiterhin korrekt zu funktionieren.
\item \textit{Trainingsdaten}: Das maschinelle Lernen braucht eine Trainingsmenge der Webseiten, die manuell mit Labels annotiert werden. Dies ist eine schwierige und fehleranf�llige Aufgabe.
\end{itemize}
Ein Verfahren, das am meisten zur Datenerfassung aus dem Web bekannt ist, nutzt die semi-strukturierten Dokumente, die in \acf{HTML} beschrieben werden. Eine HTML-Seite wird in \acf{DOM}\footnote{https://www.w3.org/DOM} definiert. Die Idee vom DOM besteht darin, dass die HTML-Webseite ein Baum darstellt, der mittels HTML-Tags (z.B. Button-Tag) ausgezeichnet wird. Tags k�nnen weitere Tags beinhalten und bilden somit eine hierarchische Struktur. Diese hierarchische Baumstruktur erm�glicht effiziente Datensuche in einer HTML-Seite \cite[S.303]{ferrara2014}.\\
Da HTML ein Dialekt von \acf{XML} ist, kann \acf{XPath}\footnote{https://www.w3.org/TR/xpath} f�r die Navigation in DOM eingesetzt werden. In einem XPath-Ausdruck k�nnen beliebige Elemente einer HTML-Webseite ausgew�hlt werden. In Abbildung \ref{xpath} werden zwei Beispiele dargestellt. Im ersten Fall (A) wird genau ein Element (die erste Zelle in der ersten Reihe) ausgew�hlt. Im Beispiel (B) werden mehrere Elemente (alle Zellen der zweiten Reihe) angesprochen \cite[S.303]{ferrara2014}.
\begin{figure}[H] 
	\centering
	\includegraphics[width=1.0\textwidth]{images/xpath.png}
	\caption{XPath im Dokumentenbaum, \cite[S.304]{ferrara2014}}
	\label{xpath}
\end{figure} 
Der Hauptnachteil von XPath besteht darin, dass XPath-Ausdr�cke strikt an der DOM-Struktur gebunden sind. Wenn eine �nderung im DOM stattfindet, funktioniert der von der �nderung betroffene Ausdruck nicht mehr. Aus diesem Grund m�ssen die XPath-Ausdr�cke nach jeder Ver�nderung der HTML-Webseite manuell angepasst werden. In Bezug auf dieses Problem wurde im letzten Release von XPath\footnote{https://www.w3.org/TR/xpath20} relative XPath-Ausdr�cke eingef�hrt \cite[S.304]{ferrara2014}.\\
Eine Beispiel des Verfahrens zur Datenextraktion ist ein Web Wrapper. Unter dem Web Wrapper wird ein Verfahren verstanden, das eine oder mehrere Algorithmen zur Datensuche beinhaltet. Dabei werden die Daten in unstrukturierten oder semi-strukturieren Dokumenten erfasst und in eine strukturierte Form transformiert. Ein Wrapper-Lebenszyklus umfasst folgende Schritte \cite[S.305]{ferrara2014}:
\begin{itemize}
\item[1.]\textit{Generierung}: Definition des Wrappers.
\item[2.]\textit{Ausf�hrung}: Datenerfassung mithilfe des Wrappers.
\item[3.]\textit{Wartung}: Anpassung des Wrappers bei der �nderung der DOM-Struktur.
\end{itemize}
Nach Ferrara et al kann ein Web Wrapper unter Verwendung von regul�ren Ausdr�cken, logikbasiertem Ansatz, Baumansatz oder maschinellem Lernen generiert werden \cite[S.306]{ferrara2014}. Im Weiteren wird der Ansatz von regul�ren Ausdr�cken im Rahmen des zweiten Schritts (Ausf�hrung) verdeutlicht.\\
Regul�re Ausdr�cke erm�glichen die Erkennung von Patterns in der unstrukturierten bzw. semi-strukturierten Dokumenten mithilfe der Regeln, die z.B. in Form von Wortgrenzen oder HTML-Tags definiert werden. Der Vorteil der regul�ren Ausdr�cken besteht darin, dass der Benutzer die Elemente auf einfache Weise in einem grafischen Interface ausw�hlen kann. Die dazugeh�rigen Regeln werden dann automatisch generiert. Eine m�gliche Umsetzung des Wrappers wird in \cite{sahuguet1999} mit W4F vorgestellt. Das Tool verf�gt �ber eine Hilfsmethode, die den Benutzer bei der Auswahl der Elementen unterst�tzt. Auf Basis der ausgew�hlten Elementen werden die Regeln erstellt. Allerdings sind die Regeln in Bezug auf DOM-�nderungen nicht flexibel. Dies f�hrt dazu, dass die Regeln stets von einem menschlichen Experten angepasst werden sollen \cite[S.306]{ferrara2014}.\\
\textbf{Weiter mit Wartung ... }
\newpage
\subsection{Maschinelles Lernen beim Wissenserwerb}
Eine Anwendung des maschinellen Lernens bei der Automatisierung des Wissenserwerbs wird in \cite{castro2001} vorgestellt. Die grundlegende Idee dabei orientiert sich an \cite{tecuci1992}, indem die Wissensakquisition als ein Prozess der der Erweiterung, Aktualisierung und Verbesserung der unvollst�ndigen Wissensbasis betrachtet wird. Der vorliegende Algorithmus verwendet die Induktion und den Algorithmus mit Fuzzylogik, um neues Wissen aus einer Trainingsmenge zu erschlie�en. Wie bereits im Kapitel \ref{subsec:Wissensakquisitionskomponente} erw�hnt wurde, werden bei der Induktion allgemeine Schlussfolgerungen auf Basis von Einzelf�llen erzielt. Bei der Fuzzylogik handelt es sich um graduelle Aussagen, die mithilfe der vagen Pr�dikaten beschrieben werden \cite[S.27]{beierle2014}. Castro et al betonen, dass die Induktion und die Fuzzylogik effektiv mit dem unvollst�ndigen Wissen arbeiten, das bei der Erfassung des Expertenwissens oft der Fall ist \cite[S.308]{castro2001}.\\
Weitere Punkte:
\begin{itemize}
\item Wissensrepr�sentation  
\item Algorithmus
\item Beispiel
\end{itemize}
\newpage
\section{Implementierung der Wissensewerbskomponente}\label{sec:Implementierung}
Nachdem die theoretischen Grundlagen von der Wissenserfassungskomponente betrachtet wurden, geht es im vorliegenden Kapitel um die praktische Umsetzung am Beispiel von \textit{PaaSfinder}\footnote{https://paasfinder.org}. Zuerst wird der Zweck und das Ziel von \textit{PaaSfinder} beschrieben. Als n�chstes werden geeignete Datenerfassungsmethoden im Rahmen von \textit{PaaSfinder} ausgew�hlt und implementiert.
\subsection{Systembeschreibung}
Die Umsetzung der Wissenserwerbskomponente erfolgt am Beispiel von \textit{PaaSfinder}\footnote{{\label{foot:paasfinder}}https://paasfinder.org}, ein Expertensystem und Open-Source-Projekt im Bereich von \acf{PaaS}. Das Ziel des Systems besteht darin, zahlreiche PaaS-Anbieter vergleichbar zu machen. \textit{PaaSfinder} verf�gt bereits �ber eine Wissensdatenbank, die aufgrund der h�ufigen �nderungen auf dem PaaS-Markt aktualisiert werden muss. Ein PaaS-Eintrag wird in der Wissensbasis vom \textit{PaaSfinder} durch folgende Eigenschaften\footnote{https://github.com/stefan-kolb/paas-profiles\#profile-specification} spezifiziert:
\begin{enumerate}
\item \textbf{General Properties}: allgemeine Eingenschaften des PaaS-Anbieters (z.B. Name, URL, Status, Type, etc.)
\item \textbf{Extensible}: generelle Erweiterungsm�glichkeit nach Kundenbedarf (z.B. spezielle Laufzeitumgebung)
\item \textbf{Pricing}: verf�gbare Preismodelle
\item \textbf{Quality of Service}: die Servicequalit�t (z.B. Verf�gbarkeit)
\item \textbf{Hosting}: Art der Bereitstellung (z.B. privat)
\item \textbf{Scaling}: Skalierbarkeit (z.B. Speichererweiterung)
\item \textbf{Runtimes}: unterst�tzte Laufzeitumgebungen (z.B. Java\footnote{https://java.com})
\item \textbf{Middleware} (z.B. Tomcat\footnote{https://tomcat.apache.org})
\item \textbf{Frameworks} (z.B. Play\footnote{https://www.playframework.com})
\item \textbf{Services} (z.B. Datenspeicher)
\item \textbf{Infrastructures} (z.B. Informationen zum Standort)
\end{enumerate}
Die Struktur vom \textit{PaaSfinder} l�sst sich in der Abbildung \ref{paasfinder} wie folgt beschreiben.
\begin{figure}[H] 
	\centering
	\includegraphics[width=1.0\textwidth]{images/paasfinder.png}
	\caption{Die Struktur vom \textit{PaaSfinder}}
	\label{paasfinder}
\end{figure} 
Dem Benutzer wird \textit{PaaSfinder} als die auf einem Server laufende Web-Anwendung bereitgestellt. Die Wissensbasis und der Programmcode von \textit{PaaSfinder} befinden sich in einem Git\footnote{https://git-scm.com}-Repository. Bei Git handelt es sich um ein verteiltes System zur Projektverwaltung. Jeder, der dem Projekt beitragen m�chte, kann das Repository des gesamten Projekts lokal speichern, gew�nschte �nderungen vornehmen und anschlie�end einen Pull-Request\footnote{https://git-scm.com/docs/git-request-pull} erstellen. Der Pull-Request wird vom Projektmaster �berpr�ft und zugeh�rige �nderungen �bernommen (\glqq{}gemerged\grqq{}), falls keine Konflikte vorliegen. Wenn der Wissenstr�ger mit dem Git-System nicht vertraut ist, kann ein direkter Kontakt (beispielsweise per E-Mail) mit dem Systementwickler aufgenommen werden. Die Daten werden dann manuell vom Systemverwalter in die Wissensbasis eingetragen.\\
Sowohl der Beitrag mittels Pull-Requests als auch Kontakt mit dem Systementwickler entspricht dem direkten Wissenserwerb gem�� der Klassifikation in Kapitel \ref{subsec:Wissenserwerbskomponente}. Der Vorteil dabei ist, dass die Daten nicht manuell gesucht werden sollen, sondern gleich vom Wissenstr�ger stammen. Anderseits wird die Datenerfassung durch Git-Kenntnisse des Wissenstr�gers beschr�nkt. Um den direkten Wissenserwerb zu erleichtern, wird im weiteren Verlauf die Wissenstr�gerschnittstelle entwickelt, die es dem Wissenstr�ger erm�glicht, einen bestehenden PaaS-Eintrag zu aktualisieren. Nach dem erfolgreichen Absenden der �nderungen soll automatisch ein Pull-Request erstellt werden. 
\newpage
\subsection{Update Interface}
Roadmap:
\begin{itemize}
\item Ablauf beschreiben
\item Data-Binding Framework
\item LocalStorage
\item Submit zum Worker
\end{itemize}
\input{sections/kapitel-4/webcrawler.tex}
\subsection{Update Service zur Datenverwaltung}
Ausf�hren
\section{Evaluation}\label{sec:Evaluation}
\subsection{Test 1}
\subsection{Test 2}
\subsection{Test 3}
\section{Fazit und Ausblick}\label{sec:Fazit und Ausblick}
\newpage

% Einstellungen f�r Literaturverzeichnis
\addcontentsline{toc}{section}{\bibname}
\bibliographystyle{geralpha}
\selectlanguage{german}

% Hier das Literaturverzeichnis einbinden
\bibliography{bibliography/references}
\newpage

% Eigenst�ndigkeitserkl�rung
\makedeclaration{Bachelorarbeit}{31.03.2017}{Petr Vasilyev}

\end{document}